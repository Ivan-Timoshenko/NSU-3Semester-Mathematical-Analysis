\begin{theorem*}[Теорема о тригонометрической аппроксимации]
    Пусть $f$ - непрерывная функция на $[0, 2\pi]$, $f(0) = f(2\pi)$. Тогда $\forall \varepsilon$ $\exists$ тригонометрический многочлен $P(x)$, такой, что 
    \\ $\forall x \in [0, 2\pi] \ \abs{f(x) - P(x)} \le \varepsilon$.
    \begin{proof}
        \begin{enumerate}
            \item Заменим $f$ кусочно гладкой линейной функцией $\tilde{f}$, так, что \\ $\forall x \ \abs{f(x) - \tilde{f}(x)} \le \frac{\varepsilon}{2}$.\par
            По теореме Кантора-Гейне (функция, непрерывная на компакте, равномерно непрерывна на нем) непрерывная функция $f$ на отрезке $[0, 2\pi]$ равномерно непрерывна, то есть
            \[\forall \varepsilon > 0 \ \exists \delta > 0 \ \forall x_1, x_2 \in [0, 2\pi] \quad \abs{x_1 - x_2} < \delta \implies \abs{f(x_1) - f(x_2)} \le \varepsilon\]
            Пользуясь этим разобьем $[0, 2\pi]$ на конечное число кусочков $[a_k, a_{k+1}]$, настолько коротких, что $\forall x_1, x_2  \in [a_k, a_{k+1}] \ \abs{f(x_1) - f(x_2)} \le \frac{\varepsilon}{4}$.
            \par
            Пусть $\tilde{f}(a_i) = f(a_i) \ \forall i$. На промежутках $[a_i, a_{i+1}]$ $\tilde{f}$ - прямая между точками $(a_i, f(a_i))$ и $(a_{i+1}, f(a_{i+1}))$.
            \begin{align*}
                \forall x \in [a_i, a_{i+1}] \\
                \abs{\tilde{f}(x) - f(x)} = \abs{\tilde{f}(x) - \tilde{f}(a_i) + \tilde{f}(a_i) - f(x)} \le \\
                \le \abs{\tilde{f}(x) - \tilde{f}(a_i)} + \abs{\tilde{f}(a_i) - f(x)}
            \end{align*}
            \begin{align*}
                \abs{\tilde{f}(x) - \tilde{f}(a_i)} \le \abs{\tilde{f}(a_{i+1}) - f(a_i)} = \abs{f(a_{i+1}) - f(a_i)} \le \frac{\varepsilon}{4}
            \end{align*}
            \[\abs{\tilde{f}(a_i) - f(x)} = \abs{f(a_i) - f(a_i)} \le \frac{\varepsilon}{4}\]

            \item Функция $\tilde{f}$ принадлежит классу Фурье, $\tilde{f}(0) = f(2\pi)$. По теореме Фурье $S_n(\tilde{f}) \rightrightarrows \tilde{f}$ на $[0, 2\pi]$. Тогда $\exists n_0 \ \underset{x \in [0, 2\pi]}{\sup}\abs{S_{n_0}(\tilde{f}(x)) - \tilde{f}(x)} \le \frac{\varepsilon}{2}$. А значит $\abs{S_{n_0}(\tilde{f}(x)) - f(x)} \le \frac{\varepsilon}{2} + \frac{\varepsilon}{2}$ (доказывается хитрым движением $\pm \tilde{f}(x)$ под модуль).
        \end{enumerate}
    \end{proof}
\end{theorem*}

\begin{theorem*}[Теорема Стоуна-Вейерштрасса]
    Непрерывную на отрезке функцию можно равномерно приблизить многочленом:
    \[f:[a,b] \to \Compl(\Real) \implies \forall \varepsilon \ \exists \text{многочлен } P(x) \ \forall x \in [a,b] \ \abs{U(x) - f(x)} \le \varepsilon\]

    \begin{proof}
        \begin{enumerate}
            \item Пусть $\lambda(x):[a,b] \underset{\text{линейное}}{\to} [0, 2\pi]$. 
            \[g(t) = f(\lambda^{-1}(t)):[0, 2\pi] \to \Compl\]
            Существует тригонометрический многочлен $Q(t)$ (например, форма Фурье) такой, что $\forall t \in [0, 2\pi] \ \abs{Q(t) - g(t)} \le \varepsilon$. Теперь осталось приблизить $Q(t)$ обычным многочленом.
            \par
            \[Q(t) = q_0 + q_1\cos(t) + \hdots + q_N\cos(Nt) + q_1\sin(t) + \hdots + q_N\sin(Nt)\]
            Для $\forall k_0$ форма $\cos(k_0 t)$ равномерно приближаемая на $[0, 2\pi]$ частичными суммами своего ряда Тейлора. 
            \[\cos(k_0 t) = 1 - \frac{(k_0 t)^2}{2!} + \frac{(k_0 t)^4}{4!} + \hdots\]
            Радиус сходимости равен бесконечности, поэтому ряд сходится равномерно на любом отрезке конечной длины.
            \par
            Пусть $n_0$ настолько большое, что все частичные суммы для \\ $\cos(t), \hdots, \cos(Nt), \sin(t), \hdots, \sin(Nt)$ были $\delta$-близки к своим функциям. Т.е.
            \[\delta := \frac{\varepsilon}{
                2\Bigl(
                    \sum_{i=1}^k \abs{q_i} + \sum_{i=1}^k \abs{\tilde{q}_i}
                \Bigr)
            }\]
            Соответствующая сумма имеет вид 
            \[q_0 + \sum_{k=1}^{n_0} q_k \cos(kt) + \sum_{k=1}^{n_0} \tilde{q}_k \sin(kt),\]
            где первая сумма — это частичная сумма ряда Тейлора (до $n_0$) для $\cos(kt)$, 
            а вторая — для $\sin(kt)$. Поскольку каждая из функций $\cos(kt)$, $\sin(kt)$
            аппроксимируется многочленом от $t$, вся сумма является многочленом;
            следовательно, можно взять $P(t)$ в указанном виде.
        \end{enumerate}
    \end{proof}
\end{theorem*}

\section{Примитивный интеграл, интеграл Лебега}
\begin{definition*}
    Отрезки вида $I = {a}, (a,b), [a,b), (a, b], [a,b]$ - примитивные сегменты в $\Real$. Для них на прямой определена примитивная мера, как функция $\mu_1(I) := \abs{b-a}$. \par
    Элементарный сегмент в $\Real^n$ это $I_1 \times \hdots \times I_n$, где $I_k$ - элементарный сегмент в $\Real$ $\forall k \in {1, \hdots, n}$. Для сегментов большой размерности мера определена в виде $\mu_n(I) = \mu_1(I_1) \cdot \hdots \cdot \mu_n(I_n)$.\par
    Множество $S$ будем считать примитивным (ступенчатым) множеством, если $S$ - объединение конечного числа сегментов.
\end{definition*}

\begin{lemma*}[Лемма о дроблении]
    Для любого примитивного (стуенчатого) множества $A$ $\exists$ дизъюнктный набор сегментов (т.е. набор непересекающихся сегментов) $I_1, \hdots, I_N$ такой, что $A = \bigcup_{n=1}^{N} I_n$.

    \begin{proof}
        Пусть $A = \bigcup J_k$, $J_k = J_k^1 \times \hdots \times J_k^r$. Пусть $a_0^i < a_1^i < \hdots < a_{N_i}^i$ - концы сегментов $J_1^i, \hdots, J_m^i$. Возьмем ${a_0^i}, {a_0^i, a_1^i}, {a_1^i}, {a_1^i, a_2^i}$ и так далее - это элементарные сегменты на оси $i$-ой координаты. Искомая система дизъюнктных элементов - это декартово произведение этих сегментов по каждой координате.
    \end{proof}
\end{lemma*}

\textbf{Свойства ступенчатых множеств:}
\begin{enumerate}
    \item Если $A, B$ - ступенчатые множества, то $A \cup B$ - ступенчатое множество.
    \item Если $A, B$ - ступенчатые множества, то $A \cap B$ - ступенчатое множество.
    \begin{proof}
        Есть сегменты, которые в объединении дают $A$ и дают $B$. Возьмем их, раздробим по лемме о дроблении, посчитаем пересечение. 
    \end{proof}
    \item $A \backslash B$ - ступенчатое множество.
\end{enumerate}

\begin{definition*}
    Ступенчатые функции: \par
    Пусть $E$ - векторное пространство. Определим функцию $f$ как ступенчатую функцию на множествах $\Real^n$ и $E$ ($f \in Step(\Real^n, E)$), если $\exists$ система сегментов в $\Real^n$ $J_1, \hdots, J_k$ такая, что $f = \sum_{i=1}^{k}C_i \chi_{J_i}$ и $C_1, \hdots, C_k \in E$.
    Уточнение: $\chi_{J_i}(x)$ - характеристическая функция вида 
    \[\chi_{J_i}(x) = \begin{cases*}
        1 \ x \in J_i \\ 0 \ x \notin J_i
    \end{cases*}\]
    А $C_i$ - некоторый вектор из $E$.
\end{definition*}

\begin{statement*}
    Пусть $f,g \in Step(\Real^n, E)$. Тогда $f+j \in Step(\Real^n, E), \ \lambda \cdot f \in Step(\Real^n, E)$. Если в $E$ можно перемножать веткора ($\forall a, b \in E \ \exists a\cdot b \in E$), то $f\cdot g \in Step(\Real^n, E)$.
    \begin{proof}
        Упражнение, надо поковыряться с определением новой функции и (возможно) хар. функциии.
    \end{proof}
\end{statement*}

\begin{statement*}
    $f \in Step(\Real^n, E) \iff \exists$ дизъюнктная система отрезков, на каждом из которых $f = const$.
\end{statement*}

\begin{definition*}
    Элементарный интеграл определим как функцию $\int_{\Real^n}: Step(\Real^n, E) \to E$. Функция $f$ должна иметь вид $\sum_{i=1}^{k}C_i\cdot \chi_{J_i}$.
    Тогда
    \[f = \sum_{i=1}^{k} C_i \cdot \chi_{J_i} \implies \int_{\Real^n}f := \sum_{i=1}^{k}C_i \mu_n(J_i)\]

    При $n = 1$ набор $J_1, \hdots, J_n$ - это интервалы, полуинтервалы, отрезки и точки. 
    \[f(x) = \sum C_i \chi_{J_i} \quad \int_{\Real^n} f(x) \diff x = \int_{-\infty}^{\infty}f(x)\diff x \text{ - обычный интеграл не зависит от разбиения}\]    
\end{definition*}

\begin{theorem*}[Примитивная теорема Фубини]
    Введем удобные обозначения: $X = \Real^n, Y = \Real^m, \ X \times Y = \Real^{n+m}$.
    Пусть $f \in Step(X \times Y, E)$. Тогда:
    \begin{enumerate}
        \item $\forall x \in X$ функция от $y$, отображающая $y \to f(x,y)$ является ступенчатой на $Y$.
        \item Функция от $x$, отображающая $x \to \int_{Y} f(x,y)\diff y$ - ступенчатая на $X$.
        \item $\int_X \left(\int_Y f(x, y) \diff y\right) \diff x = \int_{X \times Y} f(x, y) \diff (x, y)$        
    \end{enumerate}

    \begin{proof}
        \begin{enumerate}
            \item По условию $\exists$ система сегментов $J_1, \hdots, J_k$ в $X \times Y$ и набор констант $C_1, \hdots, C_k$ такой, что
            \[f(x, y) = \sum_{i=1}^{k} C_i \cdot \chi_{J_i}(x,y)\]
            Каждый из сегментов $J_i$ - это $\underbrace{J_i^X}_{\text{сегмент из Х}} \times \underbrace{J_i^Y}_{\text{сегмент из Y}}$
            Тогда 
            \[\chi_{J_i}(x,y) = \chi_{J_i^X}(x) \cdot \chi_{J_i^Y}(y)\]
            \[\forall x \ f(x, y) = \sum \underbrace{C_i \cdot \chi_{J_i^X}(x)}_{d_i}\cdot \chi_{J_i^Y}(y) \text{ - это ступенчатая функция на Y}\]
            Если $x$ - фиксирован, то эта запись определяет ступенчатую функцию на $Y$.

            \item \[\int_Y f(x, y) \diff y = \sum_{i=1}^{m} C_i \cdot \chi_{J_i^X}(x) \cdot \mu_Y (J_i^Y) = \sum_{i=1}^{m}C_i \cdot \mu_Y(J_i^Y)\cdot \chi_{J_i^X}(x) \text{ - ступенчатая функция на X}\]
            \item \begin{align*}\text{Интеграл от предыдущей функции по } X = \sum_{i=1}^{m}C_i \cdot \mu_Y(J_i^Y) \cdot \mu_X(J_i^X) = \int_{X \times Y}f(x, y) \diff (x,y) \\ \text{ так как } \mu_Y(J_i^Y) \cdot \mu_X(J_i^X) = \mu_{X \times Y}(J_i^X \times J_i^Y) = \mu_{X \times Y}(J_i)
            \end{align*}
            
        \end{enumerate}
    \end{proof} 
\end{theorem*}
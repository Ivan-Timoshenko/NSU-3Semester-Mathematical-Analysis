\begin{theorem*}[Теорема о тригонометрической аппроксимации]
    Пусть $f$ - непрерывная функция на $[0, 2\pi]$, $f(0) = f(2\pi)$. Тогда $\forall \varepsilon$ $\exists$ тригонометрический многочлен $P(x)$, такой, что 
    \\ $\forall x \in [0, 2\pi] \ \abs{f(x) - P(x)} \le \varepsilon$.
    \begin{proof}
        \begin{enumerate}
            \item Заменим $f$ кусочно гладкой линейной функцией $\tilde{f}$, так, что \\ $\forall x \ \abs{f(x) - \tilde{f}(x)} \le \frac{\varepsilon}{2}$.\par
            По теореме Кантора-Гейне (функция, непрерывная на компакте, равномерно непрерывна на нем) непрерывная функция $f$ на отрезке $[0, 2\pi]$ равномерно непрерывна, то есть
            \[\forall \varepsilon > 0 \ \exists \delta > 0 \ \forall x_1, x_2 \in [0, 2\pi] \quad \abs{x_1 - x_2} < \delta \implies \abs{f(x_1) - f(x_2)} \le \varepsilon\]
            Пользуясь этим разобьем $[0, 2\pi]$ на конечное число кусочков $[a_k, a_{k+1}]$, настолько коротких, что $\forall x_1, x_2  \in [a_k, a_{k+1}] \ \abs{f(x_1) - f(x_2)} \le \frac{\varepsilon}{4}$.
            \par
            Пусть $\tilde{f}(a_i) = f(a_i) \ \forall i$. На промежутках $[a_i, a_{i+1}]$ $\tilde{f}$ - прямая между точками $(a_i, f(a_i))$ и $(a_{i+1}, f(a_{i+1}))$.
            \begin{align*}
                \forall x \in [a_i, a_{i+1}] \\
                \abs{\tilde{f}(x) - f(x)} = \abs{\tilde{f}(x) - \tilde{f}(a_i) + \tilde{f}(a_i) - f(x)} \le \\
                \le \abs{\tilde{f}(x) - \tilde{f}(a_i)} + \abs{\tilde{f}(a_i) - f(x)}
            \end{align*}
            \begin{align*}
                \abs{\tilde{f}(x) - \tilde{f}(a_i)} \le \abs{\tilde{f}(a_{i+1}) - f(a_i)} = \abs{f(a_{i+1}) - f(a_i)} \le \frac{\varepsilon}{4}
            \end{align*}
            \[\abs{\tilde{f}(a_i) - f(x)} = \abs{f(a_i) - f(a_i)} \le \frac{\varepsilon}{4}\]

            \item Функция $\tilde{f}$ принадлежит классу Фурье, $\tilde{f}(0) = f(2\pi)$. По теореме Фурье $S_n(\tilde{f}) \rightrightarrows \tilde{f}$ на $[0, 2\pi]$. Тогда $\exists n_0 \ \underset{x \in [0, 2\pi]}{\sup}\abs{S_{n_0}(\tilde{f}(x)) - \tilde{f}(x)} \le \frac{\varepsilon}{2}$. А значит $\abs{S_{n_0}(\tilde{f}(x)) - f(x)} \le \frac{\varepsilon}{2} + \frac{\varepsilon}{2}$ (доказывается хитрым движением $\pm \tilde{f}(x)$ под модуль).
        \end{enumerate}
    \end{proof}
\end{theorem*}

\begin{theorem*}[Теорема Стоуна-Вейерштрасса]
    Непрерывную на отрезке функцию можно равномерно приблизить многочленом:
    \[f:[a,b] \to \Compl(\Real) \implies \forall \varepsilon \ \exists \text{многочлен } P(x) \ \forall x \in [a,b] \ \abs{U(x) - f(x)} \le \varepsilon\]

    \begin{proof}
        \begin{enumerate}
            \item Пусть $\lambda(x):[a,b] \underset{\text{линейное}}{\to} [0, 2\pi]$. 
            \[g(t) = f(\lambda^{-1}(t)):[0, 2\pi] \to \Compl\]
            Существует тригонометрический многочлен $Q(t)$ (например, форма Фурье) такой, что $\forall t \in [0, 2\pi] \ \abs{Q(t) - g(t)} \le \varepsilon$. Теперь осталось приблизить $Q(t)$ обычным многочленом.
            \par
            \[Q(t) = q_0 + q_1\cos(t) + \hdots + q_N\cos(Nt) + q_1\sin(t) + \hdots + q_N\sin(Nt)\]
            Для $\forall k_0$ форма $\cos(k_0 t)$ равномерно приближаемая на $[0, 2\pi]$ частичными суммами своего ряда Тейлора. 
            \[\cos(k_0 t) = 1 - \frac{(k_0 t)^2}{2!} + \frac{(k_0 t)^4}{4!} + \hdots\]
            Радиус сходимости равен бесконечности, поэтому ряд сходится равномерно на любом отрезке конечной длины.
            \par
            Пусть $n_0$ настолько большое, что все частичные суммы для \\ $\cos(t), \hdots, \cos(Nt), \sin(t), \hdots, \sin(Nt)$ были $\delta$-близки к своим функциям. Т.е.
            \[\delta := \frac{\varepsilon}{
                2\Bigl(
                    \sum_{i=1}^k \abs{q_i} + \sum_{i=1}^k \abs{\tilde{q}_i}
                \Bigr)
            }\]
            Соответствующая сумма имеет вид 
            \[q_0 + \sum_{k=1}^{n_0} q_k \cos(kt) + \sum_{k=1}^{n_0} \tilde{q}_k \sin(kt),\]
            где первая сумма — это частичная сумма ряда Тейлора (до $n_0$) для $\cos(kt)$, 
            а вторая — для $\sin(kt)$. Поскольку каждая из функций $\cos(kt)$, $\sin(kt)$
            аппроксимируется многочленом от $t$, вся сумма является многочленом;
            следовательно, можно взять $P(t)$ в указанном виде.
        \end{enumerate}
    \end{proof}
\end{theorem*}

\section{Примитивный интеграл, интеграл Лебега}
\begin{definition*}
    Отрезки вида $I = {a}, (a,b), [a,b), (a, b], [a,b]$ - примитивные сегменты в $\Real$. Для них на прямой определена примитивная мера, как функция $\mu_1(I) := \abs{b-a}$. \par
    Элементарный сегмент в $\Real^n$ это $I_1 \times \hdots \times I_n$, где $I_k$ - элементарный сегмент в $\Real$ $\forall k \in {1, \hdots, n}$. Для сегментов большой размерности мера определена в виде $\mu_n(I) = \mu_1(I_1) \cdot \hdots \cdot \mu_1(I_n)$.\par
    Множество $S$ будем считать примитивным (ступенчатым) множеством, если $S$ - объединение конечного числа сегментов.
\end{definition*}

\begin{lemma*}[Лемма о дроблении]
    Для любого примитивного (стуенчатого) множества $A$ $\exists$ дизъюнктный набор сегментов (т.е. набор непересекающихся сегментов) $I_1, \hdots, I_N$ такой, что $A = \bigcup_{n=1}^{N} I_n$.

    \begin{proof}
        Пусть $A = \bigcup J_k$, $J_k = J_k^1 \times \hdots \times J_k^r$. Пусть $a_0^i < a_1^i < \hdots < a_{N_i}^i$ - концы сегментов $J_1^i, \hdots, J_m^i$. Возьмем $\{a_0^i\}, \{a_0^i, a_1^i\}, \{a_1^i\}, \{a_1^i, a_2^i\}$ и так далее - это элементарные сегменты на оси $i$-ой координаты. Искомая система дизъюнктных элементов - это декартово произведение этих сегментов по каждой координате.
    \end{proof}
\end{lemma*}

\textbf{Свойства ступенчатых множеств:}
\begin{enumerate}
    \item Если $A, B$ - ступенчатые множества, то $A \cup B$ - ступенчатое множество.
    \item Если $A, B$ - ступенчатые множества, то $A \cap B$ - ступенчатое множество.
    \begin{proof}
        Есть сегменты, которые в объединении дают $A$ и дают $B$. Возьмем их, раздробим по лемме о дроблении, посчитаем пересечение. 
    \end{proof}
    \item $A \backslash B$ - ступенчатое множество.
\end{enumerate}

\begin{definition*}
    Ступенчатые функции: \par
    Пусть $E$ - векторное пространство. Определим функцию $f$ как ступенчатую функцию на множествах $\Real^n$ и $E$ ($f \in Step(\Real^n, E)$), если $\exists$ система сегментов в $\Real^n$ $J_1, \hdots, J_k$ такая, что $f = \sum_{i=1}^{k}C_i \chi_{J_i}$ и $C_1, \hdots, C_k \in E$.
    Уточнение: $\chi_{J_i}(x)$ - характеристическая функция вида 
    \[\chi_{J_i}(x) = \begin{cases*}
        1 \ x \in J_i \\ 0 \ x \notin J_i
    \end{cases*}\]
    А $C_i$ - некоторый вектор из $E$.
\end{definition*}

\begin{statement*}
    Пусть $f,g \in Step(\Real^n, E)$. Тогда $f+j \in Step(\Real^n, E), \ \lambda \cdot f \in Step(\Real^n, E)$. Если в $E$ можно перемножать веткора ($\forall a, b \in E \ \exists a\cdot b \in E$), то $f\cdot g \in Step(\Real^n, E)$.
    \begin{proof}
        Упражнение, надо поковыряться с определением новой функции и (возможно) хар. функциии.
    \end{proof}
\end{statement*}

\begin{statement*}
    $f \in Step(\Real^n, E) \iff \exists$ дизъюнктная система отрезков, на каждом из которых $f = const$.
\end{statement*}

\begin{definition*}
    Элементарный (примитивный) интеграл определим как функцию $\int_{\Real^n}: Step(\Real^n, E) \to E$. Функция $f$ должна иметь вид $\sum_{i=1}^{k}C_i\cdot \chi_{J_i}$.
    Тогда
    \[f = \sum_{i=1}^{k} C_i \cdot \chi_{J_i} \implies \int_{\Real^n}f := \sum_{i=1}^{k}C_i \mu_n(J_i)\]

    При $n = 1$ набор $J_1, \hdots, J_n$ - это интервалы, полуинтервалы, отрезки и точки. 
    \[f(x) = \sum C_i \chi_{J_i} \quad \int_{\Real^n} f(x) \diff x = \int_{-\infty}^{\infty}f(x)\diff x \text{ - обычный интеграл не зависит от разбиения}\]    
\end{definition*}

\begin{theorem*}[Примитивная теорема Фубини]
    Введем удобные обозначения: $X = \Real^n, Y = \Real^m, \ X \times Y = \Real^{n+m}$.
    Пусть $f \in Step(X \times Y, E)$. Тогда:
    \begin{enumerate}
        \item $\forall x \in X$ функция $f(x, y)$ от $y$ (выбрали $\forall x \in X$, зафиксировали, аргументом стал только $y$), отображающая $y \to f(x,y)$ является ступенчатой на $Y$.
        \item Функция от $x$, отображающая $x \to \int_{Y} f(x,y)\diff y$ - ступенчатая на $X$.
        \item $\int_X \left(\int_Y f(x, y) \diff y\right) \diff x = \int_{X \times Y} f(x, y) \diff (x, y)$        
    \end{enumerate}

    \begin{proof}
        \begin{enumerate}
            \item По условию $\exists$ система сегментов $J_1, \hdots, J_k$ в $X \times Y$ и набор констант $C_1, \hdots, C_k$ такой, что
            \[f(x, y) = \sum_{i=1}^{k} C_i \cdot \chi_{J_i}(x,y)\]
            Каждый из сегментов $J_i$ - это $\underbrace{J_i^X}_{\text{сегмент из Х}} \times \underbrace{J_i^Y}_{\text{сегмент из Y}}$
            Тогда 
            \[\chi_{J_i}(x,y) = \chi_{J_i^X}(x) \cdot \chi_{J_i^Y}(y)\]
            \[\forall x \ f(x, y) = \sum \underbrace{C_i \cdot \chi_{J_i^X}(x)}_{d_i}\cdot \chi_{J_i^Y}(y) \text{ - это ступенчатая функция на Y}\]
            Если $x$ - фиксирован, то эта запись определяет ступенчатую функцию на $Y$.

            \item \[\int_Y f(x, y) \diff y = \sum_{i=1}^{m} C_i \cdot \chi_{J_i^X}(x) \cdot \mu_Y (J_i^Y) = \sum_{i=1}^{m}C_i \cdot \mu_Y(J_i^Y)\cdot \chi_{J_i^X}(x) \text{ - ступенчатая функция на X}\]
            \item \begin{align*}\text{Интеграл от предыдущей функции по } X = \sum_{i=1}^{m}C_i \cdot \mu_Y(J_i^Y) \cdot \mu_X(J_i^X) = \int_{X \times Y}f(x, y) \diff (x,y) \\ \text{ так как } \mu_Y(J_i^Y) \cdot \mu_X(J_i^X) = \mu_{X \times Y}(J_i^X \times J_i^Y) = \mu_{X \times Y}(J_i)
            \end{align*}
            
        \end{enumerate}
    \end{proof} 

    \begin{lemma*}[Лемма о счетном покрытии отрезка]
        Пусть $I = \bigcup_{n=1}^\infty J_n$. При этом $I, J_n$ - сегменты в $\Real$, и система $\{J_n\}$ - дизъюнктная (отрезки не пересекаются). Тогда $\mu_1(I) = \sum_{n=1}^{\infty} \mu_1(J_n)$.
        \begin{proof}
            Поскольку $\forall k \sum_{n=1}^{k} \le \sum_{n=1}^{\infty}$, то $\mu_1(I) \ge \sum_{n=1}^{\infty} \mu_1(J_n)$. \\ Но почему $\mu_1(I) \le \sum_{n=1}^{\infty} \mu_1(J_n)$?
            Будем считать, что $I$ - замкнутый отрезок (если мы добавим концы, то ничего в рассуждениях не изменится). 
            \par
            Пусть $\varepsilon > 0$, покажем, что $\mu_1(I) \le \sum_{n=1}^{\infty} \mu_1(J_n) + \varepsilon$. Каждый сегмент $J_n = \langle a_n, b_n \rangle$ удлиним на $\frac{\varepsilon}{2^n}$, положив $\tilde{J}_n = (a_n - \frac{\varepsilon}{2^{n+1}}, b_n + \frac{\varepsilon}{2^{n+1}})$, очевидно, $J_n \subseteq \tilde{J}_n$, а значит $\mu_1(\tilde{J}_n) = \mu_1(J_n) + \frac{\varepsilon}{2^n}$.
            Получим покрытие отрезка $I$ открытыми множествами $\tilde{J}_n$. 
            \par
            В силу компактности $I$ существует конечное подпокрытие $J_{n_1}, \hdots, J_{n_m}$:
            \[I \subset \underbrace{J_{n_1} \cup \hdots \cup J_{n_m}}_{\text{конечная система}} \subset \bigcup_{n_1}^{n_m}\tilde{J}_n\]
            \[\mu_1(I) \le \sum_{n_1}^{n_m}\tilde{J}_n \le \sum_{n=1}^{\infty}\left[ \mu_1(J_n) + \frac{\varepsilon}{2^n} \right] = \sum_{n=1}^{\infty} \mu_1(J_n) + \varepsilon\]
            Поскольку это неравенство выполнено $\forall \varepsilon > 0$, то верно, что $\mu_1(I) \le \sum_{n=1}^{\infty} \mu_1(J_n)$.
        \end{proof}
        Что интересно, в $\mathbb{Q}$ лемма неверна, поскольку там любой сегмент это счетное объединение точек.
    \end{lemma*}
\end{theorem*}

\begin{definition*}
    Интегральная оценка:
    \par
    Пусть $f:\Real^k \to E$ - произвольная функция. Число $C \in [0, \infty)$ называется интегральной оценкой функции $f$, если существует возрастающая последовательность ступенчатых функций $\Real^k \to \Real$, $\varphi_1 \le \varphi_2 \le \hdots \forall x$ такая, что 
    \begin{enumerate}
        \item $\forall x \in \Real^k \ \lim_{n\to \infty} \varphi_n(x) \ge \abs{f(x)}$
        \item $\forall n$ элементарный $\int_{\Real^k} \varphi_n \le C$.
    \end{enumerate}
\end{definition*}

\begin{definition*}
    Пусть $M \subset \Real^n$. Будем говорить, что $M$ - пренебрежимое множество в $\Real^n$(множество меры нуль), если $\forall \varepsilon > 0 \ M$ можно покрыть счетным набором сегментов, сумма мер которых $\le \varepsilon$.
\end{definition*}

\begin{definition*}
    Интегральная норма
    \[\abs{\abs{f}}_{L_1}:=\inf \text{множества всех интегральных оценок}, \ f \in [0, \infty)\]
\end{definition*}

\begin{definition*}
    $f: \Real^k \to E$ любая интегрируема по Лебегу, если существует последовательность ступенчатых функций $f_1, \hdots, f_n, \hdots : \Real^k \to E$ таких, что $\abs{\abs{f - f_n}}_{L_1} \underset{n \to \infty}{\to} 0$
    Интеграл от $f$ в этом случае:
    \[\int_{\Real} f\diff x := \lim_{n \to \infty} \int_{\Real} f_n \diff x\]
\end{definition*}

\begin{statement*}[Принцип исчерпывания для элементарных множеств]
    Пусть $S_k$ для $k = 1, 2, \hdots$ и $S$ - элементарные множества. Если $S \subset \bigcup_{k=1}^{\infty} S_k$, то $\mu(S) \le \sum_{k=1}^{\infty} \mu(S_k)$.

    На самом деле это кусок из доказательства леммы о счетном покрытии отрезка.
\end{statement*}

\begin{statement*}[Следствие]
    Если $S_k$ - дизъюнктны ($S_k \cap S_l \neq \varnothing$ если $k \neq l$) и $S = \bigcup_{k=1}^{\infty}S_k$, то $\mu(S) = \sum_{k=1}^{\infty}\mu(S_k)$.
\end{statement*}

\begin{lemma*}[Принцип исчерпывания для вложенных множеств]
    Пусть множество $U$ и последовательность $U_1 \subset U_2 \subset \hdots$ $\in Step$, причем $U = \bigcup_{k = 1}^{\infty} U_k$, тогда $\lim_{i \to \infty} \mu(U_i) \ge \mu(U)$.
    \begin{proof}
        Пусть $\tilde{U}_1 := U_1, \tilde{U}_2 := U_2 \backslash U_1, \hdots, \tilde{U}_n := U_n \backslash U_{n-1}$ и так далее. Поскольку $U_n = \tilde{U}_1 \cup \hdots \cup \tilde{U}_n$, то $\bigcup \tilde{U}_n = \bigcup U_n$. Тогда по классическому принципу исчерпывания:
        \[\sum_{n=1}^{\infty} \mu(\tilde{U}_n) \ge \mu(\tilde{U})\]
        Сумма слева - классическая сумма ряда, т.е. предел последовательности частичных сумм, с учетом, что $\sum_{n=1}^{k} \mu(\tilde{U}_n) = \mu(\bigcup_{n=1}^{k} \tilde{U}_n)$:
        \[\sum_{n=1}^{\infty} \mu(\tilde{U}_n) = \lim_{k \to \infty} \sum_{n=1}^{k} \mu(\tilde{U}_n) = \lim_{k \to \infty} \mu(\bigcup_{n=1}^{k} \tilde{U}_n) = \mu(U_n)\]
    \end{proof}
\end{lemma*}

\begin{statement*}[Интеграл элементарной функции как мера подграфика]
    Пусть ступенчатая функция $f: \Real^n \to \Real$ и $f \ge 0 (\forall x \ f(x) \ge 0)$. Пусть $U_f = \{ (x, y) \mid x \in \Real^n, \ 0 \le y \le f(x) \}$. Тогда:
    \[\int_{\Real^n} f(x) \diff x = \mu_{n+1}(U_f)\]
    Т.е. интеграл $n$-мерной ступенчатой функции является мерой размерности $n+1$ для ее подграфика.
    \begin{proof}
        Представим $f = \sum_{k=1}^{m} \lambda_k \chi(S_k)$, где $S_k$ - дизъюнктные сегменты в $\Real^n$. Тогда из того, что $f(x) \ge 0$ следует, что $\lambda_k \ge 0$. По определению:
        \[\int_{\Real^n}f = \sum_{k=1}^{m}\lambda_k \cdot \mu_n(S_k)\]
        Заметим, что $\lambda_k \cdot \mu_n(S_k) = \mu_{n+1}(S_k \times [0, \lambda_k])$, ведь $S_k \times [0, \lambda_k]$ - сегмент размерности $n+1$, это буквально $U_f$ по определению.
        А сумма $\sum_{k=1}^{m} \mu_{n+1}(S_k \times [0, \lambda_k]) = \sum_{k=1}^{m}\mu_{n+1}(U_f)$.
    \end{proof}
\end{statement*}

\begin{statement*}[Принцип исчерпывания для последовательности элементарных функций]
    Пусть даны $\varphi_1 \le \varphi_2 \le \hdots$ и $\varphi$ - ступенчатые функции. Если $\forall x \ \lim_{k \to \infty} \varphi_k (x) \ge \varphi(x) (\star)$, то \[\lim_{k \to \infty} \int_{\Real^n} \varphi_k \ge \int_{\Real^n} \varphi\]

    \begin{proof}
        Предположим что $\varphi_1 \ge 0$ и $\varphi \ge 0$. Тогда $\forall k \ \varphi_k \ge 0$ в силу возрастания последовательности. Заметим, что из условия $(\star)$ следует, что $\bigcup_{k=1}U_{\varphi_k} \supset U_{\varphi}$. 
        В самом деле, если $(x, y) \in U_\varphi$ (точка принадлежит подграфику), то $0 \le y < \varphi(x)$ и $\exists K \mid 0 \le y < \varphi_K(x)$, т.е. $(x, y) \in U_{\varphi_k}$.
        Заметим также, что из $\varphi_1 \le \varphi_2 \le \hdots$ следует, что $U_{\varphi_1} \subset U_{\varphi_2} \subset \hdots$. Теперь используя интеграл как меру подграфика:
        \[\int_{\Real^n}\varphi_k = \mu_{n+1}(U_{\varphi_k}), \ \int_{\Real^n}\varphi = \mu_{n+1}(U_{\varphi})\]
        Осталось воспользоваться принципом исчерпывания для вложенных множеств. 
\newline
        А теперь рассмотрим общий случай, когда $\varphi_1, \varphi$ могут быть и меньше нуля. Прибавим ко всем $\varphi_i$ и $\varphi$ функцию $\psi \in Step$, так, чтобы стало $\begin{cases*}
            \varphi_1 + \psi \ge 0 \\ \varphi + \psi \ge 0
        \end{cases*}$ В качестве $\psi(x)$ можно взять $\psi = \max(-\varphi_1, -\varphi_1)$ - а это ступенчатая функция.
        Рассмотрим теперь $\begin{cases*}
            \tilde{\varphi}_n := \varphi_n + \psi \ge 0 \\ 
            \tilde{\varphi} := \varphi + \psi \ge 0
        \end{cases*}$, тогда по первой части доказательства получаем, что 
        $\lim_{n \to \infty} \int_{\Real^n} \tilde{\varphi}_n \ge \int_{\Real^n} \tilde{\varphi}$
        Заметим, что 
        \[\int \tilde{\varphi}_n = \int \varphi_n + \int \psi, \ \int \tilde{\varphi} = \int \varphi + \int \psi\]
        В обеих частях есть $\int \psi$, который еще и не зависит от $n$. Избавившись от него получаем необходимое утверждение.
    \end{proof}
\end{statement*}

\begin{statement*}[Свойства интегральной нормы]
    Перечислим свойства интегральной нормы:
    \begin{enumerate} 
        \item $\abs{\abs{f}} = \abs{\abs{ \ \abs{f} \ }}$
        \item $\forall f \ \abs{\abs{f}} \ge 0$
        \item $\abs{\abs{\lambda f(x)}} = \abs{\lambda}\cdot \abs{\abs{f}}$
        \item Неравенство бесконечноугольника:\\
        Если сумма $\sum_{k=1}^{f_k}$ существует $\forall x$, то
        \[\abs{\abs{\sum_{k=1}^{\infty} f_k}} \le \sum_{k=1}^{\infty}\abs{\abs{f}}\]
        \begin{proof}
            Докажем четвертое свойство. Можно считать, что $\forall k \ f_k < \infty$, иначе справа в нашем утверждении будет $\infty$ и нечего там доказывать.\\
            Пусть $\varepsilon > 0$, у нас существует каскад $\sigma_{11} \le \sigma_{12} \le \hdots$ для $f_1$ такой, что $\lim_{k \to \infty} \sigma_{1k}(x) \ge \abs{\abs{f_1(x)}}$, но при этом $\forall k$ выполнено $\int_{\Real^n} \sigma_{1k}(x) \le \abs{\abs{f(x)}} + \frac{\varepsilon}{2^1}$. \\
            Для $f_2$ существует каскад $\sigma_{21} \le \sigma_{22} \le \hdots$, такой, что $\lim_{k \to \infty} \sigma_{2k}(x) \ge \abs{\abs{f_2(x)}}$, но $\forall k \ \int_{\Real^n} \sigma_{2k} \le \abs{\abs{f_2}} + \frac{\varepsilon}{2^2}$ и так далее для всех $f_i$.\\
            Построим каскад для функции $g:= \sum_{k=1}^{\infty}f_k$. Пусть $\rho_1 := \sigma_{11},\  \rho_2 := \sigma_{21} + \sigma_{22}, \ \hdots, \rho_k = \sum_{i=1}^{k}\sigma_{ik}$.
            Помним, что все $\rho$ и $\sigma$ - ступенчатые функции. А еще отметим, что $\rho_i$ образуют возрастающую последовательность
            $\rho_1 \le \rho_2 \le \hdots$.
            \par
            Покажем, что $\lim_{k \to \infty} \rho_k(x) \ge \abs{\abs{g(x)}}$. 
            Для этого сначала установим, что \\ $\forall m \ \lim_{k \to \infty} \rho_k(x) \ge \abs{\abs{ \sum_{k=1}^{m} f_k(x)}} (\star)$.

            \[\forall k \ge m \ \rho_k(x) \ge \sigma_{1k}(x) + \hdots +\sigma_{mk}(x)\]
            Поэтому $\lim_{k \to \infty} \rho_k(x) \ge \lim_{k \to \infty} \left[ \sigma_{1k}(x) + \hdots + \sigma_{mk}(x) \right] = \underbrace{\lim_{k \to \infty} \sigma_{1k}(x)}_{\le \abs{\abs{f_1}}} + \hdots + \underbrace{\lim_{k \to \infty} \sigma_{mk}(x)}_{\abs{\abs{f_m}}} \ge \abs{\abs{f_1 + \hdots + f_m}}$ 
            В искомом неравенстве $(\star)$ теперь перейдем к пределу по $m \to \infty$ и по неравенству пределов получим, что 
            \[\lim_{k \to \infty} \rho_k(x) \ge \abs{\abs{\sum_{k=1}^{\infty}f_k(x)}}\]
            Это нам и нужно было. 
            \par
            Поскольку для $f_i$ мы вводили $\sigma_{ik}$ с условием, что $\forall k \ \int_{\Real^n} \sigma_{ik} \le \abs{\abs{f_i}} + \frac{\varepsilon}{2^i}$, то:
            \begin{equation}
                \label{infiangle}
                \int_{\Real^n} \rho_k(x) = \underbrace{\int_{\Real^n} \sigma_{1k}}_{\le \abs{\abs{f_1}} + \frac{\varepsilon}{2}} + \underbrace{\int_{\Real^n} \sigma_{2k}}_{\le \abs{\abs{f_2}} + \frac{\varepsilon}{2^2}} + \hdots + \underbrace{\int_{\Real^n} \sigma_{kk}}_{\le \abs{\abs{f_k}}+ \frac{\varepsilon}{2^k}} \le \abs{\abs{f_1}} + \hdots \le \abs{\abs{f_k}} + \varepsilon 
            \end{equation}
            Итак, нашелся каскад $\rho_k$, дающий интегральную оценку вида \eqref{infiangle} для функции $g = \sum_{k=1}^{\infty}f_k$.
            То есть:
            \[\forall \varepsilon \ \abs{\abs{\sum_{k=1}^{\infty} f_k}} \le \sum_{k=1}^{\infty} \abs{\abs{f_k}} +\varepsilon\]
            В силу общности выбора $\varepsilon$ получаем искомое.
        \end{proof}
    \end{enumerate}
\end{statement*}

\begin{definition*}
    Функция $f:\Real^n \to E$ интегрируема по Лебегу ($f\in L_1(\Real^n, E)$), если $\exists$ последовательность $\varphi_n \in Step(\Real^n, E)$ (т.е. $\abs{\abs{f - \varphi_n}} \underset{n \to\infty}{\to} 0$). В это случае:
    \[\int_{\Real^n} f := \lim_{k \to \infty} \int_{\Real^n} \varphi_k\]
\end{definition*}
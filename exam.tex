\documentclass[a4paper]{article}

\usepackage{ulem}
\usepackage{cmap}
\usepackage[T2A]{fontenc}
\usepackage[utf8]{inputenc}
\usepackage[english,russian]{babel}
\usepackage{amsthm}
\usepackage{amssymb}
\usepackage{amsmath}
\usepackage{mathtools}
\usepackage{indentfirst}
\usepackage{fullpage}
\usepackage{titlesec}
\usepackage{multicol}
\usepackage{parskip}
\usepackage{graphicx}
\usepackage{tikz}
\usepackage{wrapfig}
\usepackage{breqn}

\newcommand{\open}{\underset{op}{\subset}}
\newcommand{\Real}{\mathbb{R}}
\newcommand{\Compl}{\mathbb{C}}
\newcommand{\Nat}{\mathbb{N}}
\newcommand{\scalar}[2]{\langle #1, #2 \rangle}
\newcommand{\diff}{\mathop{}\!d}
\newcommand{\abs}[1]{\left| #1 \right|}

\newtheoremstyle{definition}{3pt}{3pt}{\upshape}{}{\bfseries}{.}{.5em}{}
\theoremstyle{definition}
\newtheorem{definition}{Опр.}
\newtheorem*{definition*}{Опр.}


\newtheoremstyle{statement}{3pt}{3pt}{\upshape}{}{\bfseries}{}{.5em}{}
\theoremstyle{statement}
\newtheorem{statement}{Утв.}
\newtheorem*{statement*}{Утв.}

\newtheoremstyle{lemma}{3pt}{3pt}{\upshape}{}{\bfseries}{}{.5em}{}
\theoremstyle{lemma}
\newtheorem{lemma}{Лемма}
\newtheorem*{lemma*}{Лемма}

\newtheoremstyle{note}{3pt}{3pt}{\upshape}{}{\bfseries}{:}{.5em}{}
\theoremstyle{note}
\newtheorem*{note}{Замечание}

\newtheorem{theorem}{Теорема}
\newtheorem*{theorem*}{Теорема}

\newtheoremstyle{example}{3pt}{3pt}{\upshape}{}{\bfseries}{:}{.5em}{}
\theoremstyle{example}
\newtheorem*{example}{Пример}

\title{Вопросы к экзамену по математическому анализу, 3 семестр}
\date{}
\author{Тимошенко Иван, 24123}


\begin{document}
    \maketitle
    \section{Вопросы}
    \begin{enumerate}
        \item \sout{Производная вдоль вектора.}
        \item \sout{Частные производные.}
        \item \sout{Дифференциал. }
        \item \sout{Дифференцируемость функции в точке. }
        \item \sout{Достаточный признак дифференцируемости. }
        \item Необходимые условия экстремума. \textbf{TODO}
        \item Пример разрывной в точке функции, имеющей в ней производную по любому направлению. 
        \item Дифференцируемость суммы, произведения, композиции. 
        \item Градиент вещественной функции. 
        \item Векторные поля. 
        \item Потенциал векторного поля. 
        \item Многократная дифференцируемость. 
        \item Теорема о равенстве смешанных производных, "контрпример". 
        \item Высшие дифференциалы.  
        \item Пример не коммутирующих дифференциальных операторов.  
        \item Теорема о разложении Тейлора. 
        \item Гессиан вещественной функции. Достаточные условия экстремума. 
        \item Функции класса $C^k$ на открытом множестве (и на множествах, не  
        обязательно открытых). 
        \item Теорема об открытости отображения вида $f(x) = x +\varphi(x)$, где $\varphi$ — 
        сжимающее отображение. 
        \item Теорема о локальной обратимости, "контрпримеры". 
        \item Лемма о локальном наложении. 
        \item Теорема о неявной функции. 
        \item Теорема об обратной функции. 
        \item Многообразия в $\Real^n$. 
        \item Примеры двумерных многообразий: сфера, тор, цилиндр, лист Мёбиуса. 
        \item Теорема Брауэра об инвариантности области (формулировка). \\ Гомеоморфизмы и $C^r$-изоморфизмы. 
        \item Многообразия с краем. лемма о крае полупространства.  
        \item Лемма об изоморфизме многообразий. лемма об открытых частях  
        многообразия. 
        \item Теорема о крае многообразия.  
        \item Теоремы о регулярных решениях. 
        \item Касательное пространство. Дифференциал как отображение  
        касательных пространств, корректность определения, свойства. 
        \item Круг и квадрат гладко не изоморфны. 
        \item Теорема о регулярном дополнении. 
        \item Метод множителей Лагранжа, достаточные условия экстремума. 
        \item Лемма о локальном вложении.
        \item Поточечная и равномерная сходимость последовательности функций. 
        \item Критерий Коши равномерной сходимости для последовательностей и  
        для рядов. 
        \item Признак Вирштрасса равномерной сходимости ряда. 
        \item Признак Абеля. Признак Дирихле. 
        \item Непрерывность дзета-функции при x>1. 
        \item Непрерывность равномерного предела последовательности  
        непрерывных функций. 
        \item Интегрирование равномерного предела. Пример "расползающаяся  
        куча". 
        \item Теорема о пределе производных. 
        \item Теорема о дифференцировании функционального ряда. 
        \item Теорема Абеля. 
        \item Теорема о сумме степенного ряда. 
        \item Ортогональность тригонометрической системы функций. 
        \item Вещественная и комплексная форма записи ряда Фурье. 
        \item Лемма об интегралах периодической функции. 
        \item Формула Дирихле. 
        \item Теорема Фурье. 
        \item Теоремы Вейерштрасса о тригонометрической и о полиномиальной  
        аппроксимации. 
        \item Равенство Парсеваля. 
        \item Изопериметрическое неравенство. 
        \item Примитивный интеграл. 
        \item Мера сегмента. 
        \item Ступенчатые функции. 
        \item Интеграл ступенчатой функции, корректность и элементарная теорема  
        Фубини. 
        \item Лемма о счетном покрытии интервала.
    \end{enumerate}
    
    \newpage
    
    \section{Ответы}
    \subsection{Производная вдоль вектора}    
        Стандартный контекст в котором работаем:
        \[f:U \subset \mathbb{R}^{n} \to \mathbb{R}^{k}, \quad p \in U, \quad p = (p_1, p_2, \dots, p_n) \]
        
        \begin{definition*}
            Прозводная вдоль вектора $v$:
            \[\frac{\partial f}{\partial v}(p) := \lim_{t \to 0} \frac{f(p + tv) - f(p)}{t}\]
            Если $v = e_i = (0, \ldots, 0, \underset{i}{1}, 0, \ldots, 0)$, то $\frac{\partial f}{\partial v} = \frac{\partial f}{\partial x_i} = f'_{x_i}$
        \end{definition*}

        \textbf{Пример:}
        \[f(x) = 
        \begin{cases} 
            \frac{x^2y}{x^4 + y^2} \quad $при$(x, y) \neq (0, 0) \\
            0 \quad $при$ (x, y) = (0,0)
        \end{cases}\]
        По любому вектору $v = (v_1, v_2)$ у функции есть производная в $(0, 0)$:
        \[\lim_{t \to 0} \frac{f(tv) - f(0, 0)}{t} = \frac{t^3v_1^2v^2}{t^5v_1^4 + t^3v_2^2} = \frac{v_1^2v_2}{t^2v_1^2 v_2^2} 
        \underset{t \to 0}{=} 
        \begin{cases}
            0 \quad v_2 = 0\\
            \frac{v_1^2}{v_2} \quad v_2 \neq 0
        \end{cases}\]

    \subsection{Частные производные.}
        \begin{definition*}
            Частная производная по координате $x_i$ это:
            \[\frac{\partial f}{\partial x_i}(p) = \lim_{t \to 0} \frac{f(p_1,\hdots, p_{i-1}, p_i + t, p_{i+1}, \hdots, p_n) - f(p_1,\hdots,p_n)}{t}\]
        \end{definition*}

        Пример для $f(x, y) = x^y$:
        \[\frac{\partial f}{\partial x}(x, y) = yx^{y-1}; \quad \frac{\partial f}{\partial y}(x, y) = x^y\ln(x)\]
    
    \subsection{Дифференциал.}
        \begin{definition*}
            Дифференциал функции $f$ в точке $p$ - линейное отображение $df(p): \mathbb{R}^{n} \to \mathbb{R}^{k}$,
            такое, что    \[f(x) = f(p) + df(p)<x-p> + \alpha(x) \quad (\alpha(x) \underset{x \to p}{=} o(x-p)) \]
            При сдвиге точки $p$ на вектор $h$: \[f(p+h) = f(p) + df(p)<h> + o(|h|)\]
        \end{definition*}

    \subsection{Дифференцируемость функции в точке.}    
        Пусть $f:U \in \mathbb{R}^{n} \to \mathbb{R}^{k}$, $p \in U$. 
        Функция $f$ дифференциируема в $p$, если:
        \begin{enumerate}
            \item $p \in Int(U) \quad (\exists \epsilon > 0 \quad B_\epsilon(p) \subset U)$
            \item $\exists$ дифференциал функции (линейное отображение)
                $f$ в точке $p$ \quad $df(p): \mathbb{R}^{n} \to \mathbb{R}^{k}$
                такое, что \[f(x) = f(p) + df(p)<x-p> + \alpha(x) \quad (\alpha(x) \underset{x \to p}{=} o(x-p)) \]
        \end{enumerate}
    
    \subsection{Достаточный признак дифференциируемости.}
        \begin{theorem*}[Достаточный признак дифференциируемости]
            Если все частные производные непрерырвны в $p$, то $f$ дифференциируема в $p$ ($f \in D(p)$).
        \end{theorem*}

        Пример:
            $f(x, y) = x^y$ дифференциируема во всех точках $(x_0, y_0)$, где $x_0 > 0$.
            \[\frac{\partial f}{\partial x} = yx^{y-1} \qquad \frac{\partial f}{\partial y} = x^y\ln(x)\]
            Частные производные непрерывны, значит и функция непрерывна.

    \subsection{Необходимые условия экстремума.}
        \begin{theorem*}[Необходимое условие экстремума функции одной переменной - условие Ферма]    
            Пусть $f \in D(p)$ (дифференциируема в p). Если p - экстремум, то $f'(p) = 0$.
        \end{theorem*}

        \begin{note}
            НО например для $f(x) = x^{3} \quad f'(0) = 0$, но $f(x)$ не дифференциируема в 0.
            \textbf{ЧТО ЗА ОБМАН}
        \end{note}
    
        \begin{note}
            Необходимое условие экстремума выполнено лишь для точек во внутренности области определения, точки на границе необходимо проверять отдельно.
        \end{note}

        \begin{theorem*}[Необходимое условие экстремума функции многих переменных]
            Вектор частных производных первого порядка по переменным равен нулевому вектору:
            \[\left( \frac{\partial f}{\partial x_1}, \hdots, \frac{\partial f}{\partial x_n}  \right) = (0, \hdots, 0)  \]
        \end{theorem*}
        
\end{document}

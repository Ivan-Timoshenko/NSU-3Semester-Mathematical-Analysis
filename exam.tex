\documentclass[a4paper]{article}

\usepackage{ulem}
\usepackage{cmap}
\usepackage[T2A]{fontenc}
\usepackage[utf8]{inputenc}
\usepackage[english,russian]{babel}
\usepackage{amsthm}
\usepackage{amssymb}
\usepackage{amsmath}
\usepackage{mathtools}
\usepackage{indentfirst}
\usepackage{fullpage}
\usepackage{titlesec}
\usepackage{multicol}
\usepackage{parskip}
\usepackage{graphicx}
\usepackage{tikz}
\usepackage{wrapfig}
\usepackage{breqn}

\newcommand{\open}{\underset{op}{\subset}}
\newcommand{\Real}{\mathbb{R}}
\newcommand{\Compl}{\mathbb{C}}
\newcommand{\Nat}{\mathbb{N}}
\newcommand{\scalar}[2]{\langle #1, #2 \rangle}
\newcommand{\diff}{\mathop{}\!d}
\newcommand{\abs}[1]{\left| #1 \right|}

\newtheoremstyle{definition}{3pt}{3pt}{\upshape}{}{\bfseries}{.}{.5em}{}
\theoremstyle{definition}
\newtheorem{definition}{Опр.}
\newtheorem*{definition*}{Опр.}


\newtheoremstyle{statement}{3pt}{3pt}{\upshape}{}{\bfseries}{}{.5em}{}
\theoremstyle{statement}
\newtheorem{statement}{Утв.}
\newtheorem*{statement*}{Утв.}

\newtheoremstyle{lemma}{3pt}{3pt}{\upshape}{}{\bfseries}{}{.5em}{}
\theoremstyle{lemma}
\newtheorem{lemma}{Лемма}
\newtheorem*{lemma*}{Лемма}

\newtheoremstyle{note}{3pt}{3pt}{\upshape}{}{\bfseries}{:}{.5em}{}
\theoremstyle{note}
\newtheorem*{note}{Замечание}

\newtheorem{theorem}{Теорема}
\newtheorem*{theorem*}{Теорема}

\newtheoremstyle{example}{3pt}{3pt}{\upshape}{}{\bfseries}{:}{.5em}{}
\theoremstyle{example}
\newtheorem*{example}{Пример}

\title{Вопросы к экзамену по математическому анализу, 3 семестр}
\date{}
\author{Тимошенко Иван, 24123}


\begin{document}
    \maketitle
    \section{Вопросы}
    \begin{enumerate}
        \item \sout{Производная вдоль вектора.}
        \item \sout{Частные производные.}
        \item \sout{Дифференциал. }
        \item \sout{Дифференцируемость функции в точке. }
        \item \sout{Достаточный признак дифференцируемости. }
        \item \sout{Необходимые условия экстремума.}
        \item \sout{Пример разрывной в точке функции, имеющей в ней производную по любому направлению.}  
        \item \sout{Дифференцируемость суммы, произведения, композиции. }
        \item \sout{Градиент вещественной функции.} 
        \item \sout{Векторные поля. } \textbf{TODO: Объяснялка что такое векторное поле и его потенциал через ветер или ландшафт}
        \item \sout{Потенциал векторного поля.}
        \item \sout{Многократная дифференцируемость.} 
        \item \sout{Теорема о равенстве смешанных производных, "контрпример".}
        \item \sout{Высшие дифференциалы.}
        \item Пример не коммутирующих дифференциальных операторов.  
        \item \sout{Теорема о разложении Тейлора.} 
        \item \sout{Гессиан вещественной функции. Достаточные условия экстремума.}
        \item \sout{Функции класса $C^k$ на открытом множестве (и на множествах, не  
        обязательно открытых). }
        \item \sout{Теорема об открытости отображения вида $f(x) = x +\varphi(x)$, где $\varphi$ - сжимающее отображение. }
        \item Теорема о локальной обратимости, "контрпримеры". 
        \item Лемма о локальном наложении. 
        \item Теорема о неявной функции. 
        \item Теорема об обратной функции. 
        \item Многообразия в $\Real^n$. 
        \item Примеры двумерных многообразий: сфера, тор, цилиндр, лист Мёбиуса. 
        \item Теорема Брауэра об инвариантности области (формулировка). \\ Гомеоморфизмы и $C^r$-изоморфизмы. 
        \item Многообразия с краем. лемма о крае полупространства.  
        \item Лемма об изоморфизме многообразий. лемма об открытых частях  
        многообразия. 
        \item Теорема о крае многообразия.  
        \item Теоремы о регулярных решениях. 
        \item Касательное пространство. Дифференциал как отображение  
        касательных пространств, корректность определения, свойства. 
        \item Круг и квадрат гладко не изоморфны. 
        \item Теорема о регулярном дополнении. 
        \item Метод множителей Лагранжа, достаточные условия экстремума. 
        \item Лемма о локальном вложении.
        \item Поточечная и равномерная сходимость последовательности функций. 
        \item Критерий Коши равномерной сходимости для последовательностей и  
        для рядов. 
        \item Признак Вирштрасса равномерной сходимости ряда. 
        \item Признак Абеля. Признак Дирихле. 
        \item Непрерывность дзета-функции при x>1. 
        \item Непрерывность равномерного предела последовательности  
        непрерывных функций. 
        \item Интегрирование равномерного предела. Пример "расползающаяся  
        куча". 
        \item Теорема о пределе производных. 
        \item Теорема о дифференцировании функционального ряда. 
        \item Теорема Абеля. 
        \item Теорема о сумме степенного ряда. 
        \item Ортогональность тригонометрической системы функций. 
        \item Вещественная и комплексная форма записи ряда Фурье. 
        \item Лемма об интегралах периодической функции. 
        \item Формула Дирихле. 
        \item Теорема Фурье. 
        \item Теоремы Вейерштрасса о тригонометрической и о полиномиальной  
        аппроксимации. 
        \item Равенство Парсеваля. 
        \item Изопериметрическое неравенство. 
        \item Примитивный интеграл. 
        \item Мера сегмента. 
        \item Ступенчатые функции. 
        \item Интеграл ступенчатой функции, корректность и элементарная теорема  
        Фубини. 
        \item Лемма о счетном покрытии интервала.
    \end{enumerate}
    
    \newpage
    
    \section{Ответы}
    \subsection{Производная вдоль вектора}    
        Стандартный контекст в котором работаем:
        \[f:U \subset \mathbb{R}^{n} \to \mathbb{R}^{k}, \quad p \in U, \quad p = (p_1, p_2, \dots, p_n) \]
        
        \begin{definition*}
            Прозводная вдоль вектора $v$:
            \[\frac{\partial f}{\partial v}(p) := \lim_{t \to 0} \frac{f(p + tv) - f(p)}{t}\]
            Если $v = e_i = (0, \ldots, 0, \underset{i}{1}, 0, \ldots, 0)$, то $\frac{\partial f}{\partial v} = \frac{\partial f}{\partial x_i} = f'_{x_i}$
        \end{definition*}

        \textbf{Пример:}
        \[f(x) = 
        \begin{cases} 
            \frac{x^2y}{x^4 + y^2} \quad $при$(x, y) \neq (0, 0) \\
            0 \quad $при$ (x, y) = (0,0)
        \end{cases}\]
        По любому вектору $v = (v_1, v_2)$ у функции есть производная в $(0, 0)$:
        \[\lim_{t \to 0} \frac{f(tv) - f(0, 0)}{t} = \frac{t^3v_1^2v^2}{t^5v_1^4 + t^3v_2^2} = \frac{v_1^2v_2}{t^2v_1^2 v_2^2} 
        \underset{t \to 0}{=} 
        \begin{cases}
            0 \quad v_2 = 0\\
            \frac{v_1^2}{v_2} \quad v_2 \neq 0
        \end{cases}\]

    \subsection{Частные производные.}
        \begin{definition*}
            Частная производная по координате $x_i$ это:
            \[\frac{\partial f}{\partial x_i}(p) = \lim_{t \to 0} \frac{f(p_1,\hdots, p_{i-1}, p_i + t, p_{i+1}, \hdots, p_n) - f(p_1,\hdots,p_n)}{t}\]
        \end{definition*}

        Пример для $f(x, y) = x^y$:
        \[\frac{\partial f}{\partial x}(x, y) = yx^{y-1}; \quad \frac{\partial f}{\partial y}(x, y) = x^y\ln(x)\]
    
    \subsection{Дифференциал.}
        \begin{definition*}
            Дифференциал функции $f$ в точке $p$ - линейное отображение $df(p): \mathbb{R}^{n} \to \mathbb{R}^{k}$,
            такое, что    \[f(x) = f(p) + df(p)<x-p> + \alpha(x) \quad (\alpha(x) \underset{x \to p}{=} o(x-p)) \]
            При сдвиге точки $p$ на вектор $h$: \[f(p+h) = f(p) + df(p)<h> + o(|h|)\]
        \end{definition*}

    \subsection{Дифференцируемость функции в точке.}    
        Пусть $f:U \in \mathbb{R}^{n} \to \mathbb{R}^{k}$, $p \in U$. 
        Функция $f$ дифференциируема в $p$, если:
        \begin{enumerate}
            \item $p \in Int(U) \quad (\exists \epsilon > 0 \quad B_\epsilon(p) \subset U)$
            \item $\exists$ дифференциал функции (линейное отображение)
                $f$ в точке $p$ \quad $df(p): \mathbb{R}^{n} \to \mathbb{R}^{k}$
                такое, что \[f(x) = f(p) + df(p)<x-p> + \alpha(x) \quad (\alpha(x) \underset{x \to p}{=} o(x-p)) \]
        \end{enumerate}
    
    \subsection{Достаточный признак дифференциируемости.}
        \begin{theorem*}[Достаточный признак дифференциируемости]
            Если все частные производные непрерырвны в $p$, то $f$ дифференциируема в $p$ ($f \in D(p)$).
        \end{theorem*}

        Пример:
            $f(x, y) = x^y$ дифференциируема во всех точках $(x_0, y_0)$, где $x_0 > 0$.
            \[\frac{\partial f}{\partial x} = yx^{y-1} \qquad \frac{\partial f}{\partial y} = x^y\ln(x)\]
            Частные производные непрерывны, значит и функция непрерывна.

    \subsection{Необходимые условия экстремума.}
        \begin{theorem*}[Необходимое условие экстремума функции одной переменной - условие Ферма]    
            Пусть $f \in D(p)$ (дифференциируема в p). Если p - экстремум, то $f'(p) = 0$.
        \end{theorem*}

        \begin{note}
            НО например для $f(x) = x^{3} \quad f'(0) = 0$, но 0 - не экстремум нашей функции $f(x)$.
        \end{note}
    
        \begin{note}
            Необходимое условие экстремума выполнено лишь для точек во внутренности области определения, точки на границе необходимо проверять отдельно.
        \end{note}

        \begin{theorem*}[Необходимое условие экстремума функции многих переменных]
            Вектор частных производных первого порядка по переменным равен нулевому вектору:
            \[\left( \frac{\partial f}{\partial x_1}, \hdots, \frac{\partial f}{\partial x_n}  \right) = (0, \hdots, 0)  \]
        \end{theorem*}
    
    \subsection{Пример разрывной в точке функции, имеющей в ней производную по любому направлению}
        Рассмотрим функцию $f = \begin{cases*}
            \frac{x^2 y}{x^4 + y^2} \ (x, y) \neq 0 \\ 0 \ (x,y) = (0,0)
        \end{cases*}$. В (0, 0) у нее есть производная по любому направлению:
        \[v = (v_1, v_2) \quad \lim_{t \to 0} \frac{t^3 v_1^2 v_1}{t^4 v_1^4 + t^2 v_2^2} = \lim_{t \to 0} \frac{tv_1^2 v_2}{t^2 v_1^4 + v_2^2} = \frac{v_1^2}{v_2}\]
        Если компонента $v_2 = 0$, то производная $= 0$, если же $v_2 \neq 0$, то производная равна $\frac{v_1^2}{v_2}$.
        Рассмотрим направление $\begin{cases*} x(t) = t \\ y(t) = t^2 \end{cases*}$. Если двигаться по нему к точке $(0,0)$ (т.е. $t\to 0$) функция будет стремиться к:
        \[f(t, t^2) = \lim_{t \to 0} \frac{t^4}{t^4 + t^4} = \frac{1}{2}\]
        Совсем не похоже на 0, т.е. функция разрывна в 0. Но мы показали, что у нее есть производная по этому направлению (равна 1 кстати).

    \subsection{Дифференцируемость суммы, произведения, композиции.}
        Дифференцируемость суммы, произведения и композиции удовлетворяет классическим правилам дифференциируемости, это несложно доказать, но мне лень.        

    \subsection{Градиент вещественной функции}
        В случае, если функция $f$ отображает $\mathbb{R}^n \to \mathbb{R}$, то матрица Якоби принимает вид $1 \times n$ и называется \textbf{градиентом функции}.
        \begin{definition*}
            Градиентом функции называется вектор 
            \[D_f = 
            \begin{pmatrix}
                \frac{\partial f}{\partial x_1}, & \dots, & \frac{\partial f}{\partial x_n}
            \end{pmatrix}
            \]
        \end{definition*}
        
        \begin{definition*}
            Функция дифференциируема в точке, если 
            \begin{itemize}
                \item $f:U \to \mathbb{R}^k$ и $p \in Int(U)$
                \item $f(x) = f(p) + df(p)\langle x-p \rangle + \alpha(x),$ где $\alpha(x) = o(x-p).$
            \end{itemize}
        \end{definition*}
        Если $k = 1$, то лин. отображение $df(p):\mathbb{R}^n \to \mathbb{R}$
        можно задать как $df(p)\langle v \rangle = \langle\nabla f(p);  v \rangle$
        - скалярное произведение градиента функции на вектор, причем 
        $\nabla f(p) = \left(\frac{\partial f}{\partial x_1}(p), \dots, \frac{\partial f}{\partial x_n}(p)\right)$ - вектор частных производных в точке $p$.

        \begin{statement*}
            Градиент функции задает направление, при движении в котором функция растет быстрее всего.
            \begin{proof}
                Рассмотрим функцию $f$ в точке $p$, вектор $v$ единичной длины будет задавать произвольное направление.
                \begin{equation*}\frac{f(p+tv) - f(p)}{t} \underset{t \to 0}{\to}
                    \frac{\partial f}{\partial v} = df(p)\langle v \rangle = \langle \nabla f(p); v \rangle 
                    = |\nabla f(p)| \cdot |v| \cdot cos(\varphi), \text{где $\varphi $ - угол между $\nabla f$ и $v$.}
                \end{equation*}
                Поскольку $|\nabla f(p) = const, |v| = 1$, то для максимизации надо выбрать такое $\varphi$, 
                чтобы $cos(\varphi)$ был максимален, т.е. вектора $v$ и $\nabla f$ параллельны и $\nabla f$ задает наибольшую скорость роста.
            \end{proof}
        \end{statement*}

        \begin{statement*}
            $\nabla f(p)$ ортогонален поверхности уровня $\Omega = \{x | f(x) = c\}$.
            \begin{proof}
                Пусть $f(p) = c \ (p \in \Omega)$. Пусть $x_n \in \Omega$, покажем, что $cos(\nabla f(p), \overrightarrow{x_n-p}) \underset{n \to \infty}{\to} 0:$
                \begin{eqnarray*}
                    f(x_n) = f(p) = c \implies 0 = f(x_n) - f(p) = df(p)\langle x_n-p \rangle + o(x_n - p) = 
                    \langle \nabla f(p); x_n - p \rangle + o(x_n - p).
                \end{eqnarray*}
                Значит $0 \underset{n \to \infty}{=} \langle \nabla f(p); \frac{x_n - p}{|x_n - p|} \rangle + o(1)$, т.е.
                $\langle \nabla f(p); \frac{x_n - p}{|x_n - p|} \rangle \to 0 $. Тогда:
                \begin{equation*}
                    \langle \nabla f(p); \frac{x_n - p}{|x_n - p|} \rangle = |\nabla f(p)|\cdot \left|\frac{x_n-p}{|x_n-p|}\right|\cdot cos(\alpha) \to 0, 
                    \ \text{т.е.} \alpha \underset{n \to \infty}{\to} \frac{\pi}{2}. 
                \end{equation*} 
            \end{proof}
        \end{statement*}

    \subsection{Векторные поля.}
        \begin{definition*}
            Функция $f:\mathbb{R}^n \to \mathbb{R}^n$ называется векторным полем.
        \end{definition*}

    \subsection{Потенциал векторного поля.}
        \begin{definition*}
            Потенциалом векторного поля $F$ (если он есть) называется \textbf{скалярная} функция $U:W \to \mathbb{R}$, 
            такая, что $\nabla U = F$. Если потенциал существует, то F называется потенциальным полем.
        \end{definition*}

    \subsection{Многократная дифференцируемость.}
        \begin{definition*}
            Рекурсивное: $f: U \subset \mathbb{R}^n \to \mathbb{R}^m \ k$ раз дифференциируема в точке $p$ ($f \in D^k(p)$), если:
            \begin{enumerate}
                \item $f$ дифференциируема во всех точках некоторой окрестности точки $p$;
                \item Все частные производные $\frac{\partial f}{\partial x_1}, \dots, \frac{\partial f}{\partial x_n}$ дифференциируемы $k-1$ раз в точке $p$.
            \end{enumerate}  
        \end{definition*}

        \begin{statement*}
            Если $\begin{cases}
                f \in D^k(p): \mathbb{R}^n \to \mathbb{R}^k\\
                g \in D^k(p): \mathbb{R}^n \to \mathbb{R}^k
            \end{cases}$ тогда $h(x) = f(x) \cdot g(x) \in D^k(p)$

            \begin{proof}
                \begin{equation*}
                    \frac{\partial h}{\partial x_i}(x) = \frac{\partial f}{\partial x_i}(x)\cdot g(x) +
                    f(x) \cdot \frac{\partial g}{\partial x_i}
                \end{equation*}
                Так как $\frac{\partial f}{\partial x_i}(x) \in D^{k-1}(p), \ g(x) \in D^k(p), \ f(x) \in D^k(p), \ \frac{\partial g}{\partial x_i} \in D^{k-1}(p)$,
                то $\frac{\partial h}{\partial x_i} \in D^{k-1}(p)$.
            \end{proof}
        \end{statement*}

    \subsection{Теорема о равенстве смешанных производных, "контрпример".}
        \begin{theorem*}[о вторых производных]
            Пусть $f:U \subset \mathbb{R}^n \to \mathbb{R}, \ f \in D^2(p)$. Тогда $\frac{\partial^2 f}{\partial x \partial y} = \frac{\partial^2 g}{\partial y \partial x}$.
            \begin{proof}
                Можно считать, что $n = 2$, так как при заданной функции $f(x_1, x_2, \dots)$ можно в качестве $f$ рассмотреть 
                сужение $f$ на плоскость $Ox_1x_2$, т.к. при дифференциировании по $x_1$ или $x_2$ остальные переменные не изменяются.

                \[f = f(x, y) \in D^2(p), \quad p = (x_0, y_0, \dots)\]
                Считаем, что $p = 0$ и что $\frac{\partial f}{\partial x}(0) = 0, \ \frac{\partial f}{\partial y}(0) = 0$.
                Чтобы показать почему так можно считать введем $f_1$:
                \[f_1(x, y) := f(x,y) - f'_x(0, 0)\cdot x - f'_y(0,0)\cdot y\]
                \[\frac{\partial f_1}{\partial x}(0, 0) = \frac{\partial f}{\partial x}(0, 0) - f'_x(0,0)\]
                \[\frac{\partial f_1}{\partial y}(0,0) = \frac{\partial f}{\partial y}(0,0) - f'_y(0,0)\]
                
                Дальше считаем, что $f = f_1$ и $f(0,0) = 0$. Линейно приблизим производную в (0,0), взяв первые два члена ряда Тейлора с учетом $\frac{\partial f}{\partial x}(0,0) = f(0,0) = 0$:
                \begin{equation*}
                    \frac{\partial f}{\partial x }(x, y) =f(0,0) + a_{11}\cdot x + a_{12}\cdot y + \alpha_1(x, y), \ \text{где }
                    \alpha_1(x, y) = o(x, y)
                \end{equation*}
                По условию $\frac{\partial f}{\partial x}, \frac{\partial f}{\partial y} \in D(0)$, поэтому имеем право сказать
                $a_{11} = f_{xx}(0), \ a_{12} = f_{xy}(0)$.
                \[\frac{\partial f}{\partial y}(x, y) = f(0,0) + a_{21}\cdot x + a_{22}\cdot y + \alpha_2(x, y), \ 
                a_{21} = f_{yx}(0), \ a_{22} = f_{yy}(0)\]

                \newpage
                \begin{minipage}[t]{0.45\textwidth}        
                    \begin{tikzpicture}[remember picture, overlay]
                        \node[anchor=north west, yshift=5pt, xshift=10pt] at (current page.north west) {
                            \includegraphics{График_для_теоремы_Шварца.png}
                        };
                    \end{tikzpicture}
                \end{minipage}
                \begin{minipage}[t]{0.5\textwidth}
                    Рассмотрим точку $(s,s)$ вблизи нуля. Для нее 
                    $f(s,s) - f(0,0) = (f(s,s) - f(s, 0)) + (f(s,0) - f(0,0)) = (II) + (I)$. \newline
                    И в то же время $f(s, s) =  (\text{IV}) + (\text{III})$
                \end{minipage}
                \\[100pt]
                \[\text{II} = f(s,s) - f(s,0) = \int_{y=0}^{s}\frac{\partial f}{\partial y}(s,t)dt
                = \int_{t=0}^{s}a_{21}s + a_{22}t + \alpha_2(s,t)dt = a_{21}s^2 + \frac{a_{22}s^2}{2} + \varepsilon_1(s),\]
                причем $\varepsilon_1(s) = \int_{t=0}^{s}\alpha_2(s,t)dt$. Аналогично для I:
                \[\text{I} = f(s,0) - f(0,0) = \int_{t=0}^{s}\frac{\partial f}{\partial x}(t,0)dt = \int_{t=0}^{s}a_{11}t+a_{12}\cdot0 + \alpha_1(t)dt = 
                \frac{a_{11}}{2}s^2 + \varepsilon_2(s), \ \varepsilon_2(s) = \int_{t=0}^{s}\alpha_1(t,0)dt\]
                
                Итого: $f(s,s) - f(0,0) = \text{I} + \text{II} = s^2 \left( a_{21} + \frac{a_{11}}{2} + \frac{a_{22}}{2}+ \frac{\varepsilon_1(s) + \varepsilon_2(s)}{s^2}\right)$, что на самом деле равно $\text{III} + \text{IV} = 
                \\ = s^2\left(a_{12} + \frac{a_{11}}{2} + \frac{a_{22}}{2} + \frac{\varepsilon_3(s) + \varepsilon_4(s)}{s^2}\right)$
                
                \begin{equation}
                    \label{eq}
                    a_{21} + \frac{a_{11} + a_{22}}{2} + \frac{\varepsilon_1(s) + \varepsilon_2(s)}{s^2} = a_{12} + \frac{a_{22} + a_{11}}{2} + \frac{\varepsilon_3(s) + \varepsilon_4(s)}{s^2}
                \end{equation}
                При малых $s$:
                \[\varepsilon_3(s) = \int_{t=0}^{s}\alpha_2(0, t)dt, \quad \varepsilon_3(s) = \int_{t=0}^{s}\alpha_1(t,s)dt\]
                Осталось показать, что $\varepsilon_{1,2,3,4} \underset{s \to 0}{=} o(s^2)$.
                Пусть $\varepsilon > 0$. Вспомним, что 
                \[\varepsilon_1(s) = \int_{t = 0}^{}\alpha_2(s,t)dt, \quad \alpha_2(x,y) \underset{x,y \to 0}{=} o(x,y)\]
                То есть, в некотором круге $V$ точки $(0,0)$ выполнено $\forall (x,y) \in V \quad \alpha_2(x,y) \leq \varepsilon \cdot \sqrt{x^2 + y^2}$.
                Для $s$ таких, что $(s,s) \in V \quad \alpha_2(x,y) \leq \varepsilon \cdot \sqrt{2} \cdot s$, при $\left| x \right| \leq s, \ \left| y \right| \leq s$.
                Тогда 
                \[\left| \varepsilon_1(s)\right| = \left| \int_{t=0}^{s}\alpha_2(s,t)dt \right| \leq \int_{t=0}^{s} \left|\alpha_2(s,t)\right|dt 
                \leq \int_{t=0}^{s}\varepsilon\cdot s\sqrt{2}\cdot dt = \varepsilon s^2\sqrt{2}\].
                Итак, мы доказали, что $\varepsilon_1(s) \underset{s \to 0}{=} o(s^2)$, аналогичным образом показываем для $\varepsilon_{2,3,4}$
                Тогда в равенстве \eqref{eq} $\frac{\varepsilon_1(s) + \varepsilon_2(s)}{s^2} \to 0$ и $\frac{\varepsilon_3(s) + \varepsilon_4(s)}{s^2} \to 0$, а значит $a_{21} = a_{12}$,
                то есть $f_{xy}(0) = f_{yx}(0)$.
            \end{proof}
        \end{theorem*}

        \begin{example}
            \textbf{Контрпример:}\par
            Возьмем функцию $f = \begin{cases*}
                \frac{xy(x^2-y^2)}{x^2+y^2} \ (x, y) \neq (0, 0)\\ 0 \ (x, y) = (0,0)
            \end{cases*}$. 
            Если $x^2 + y^2 \neq 0$:
            \[\frac{\partial f}{\partial x}(x, y) = \frac{x^4 y + 4x^2 y^3 - y^5}{(x^2 + y^2)^2} = g(x, y)\]
            Если $x^2 + y^2 = 0$, то поскольку $f(0, y) = f(x, 0) = 0$:
            \[\frac{\partial f}{\partial x} = \frac{\partial f}{\partial y} = 0\]
            \[\frac{\partial}{\partial x}\left[\frac{\partial f}{\partial x}\right](0,0) = \lim_{x \to 0} \frac{\frac{\partial f}{\partial x}(x, 0) - \frac{\partial f}{\partial x}(0,0)}{x} = 0\]
            \[\frac{\partial}{\partial y}\left[\frac{\partial f}{\partial x}\right](0,0) = \lim_{y \to 0} \frac{\frac{\partial f}{\partial x}(0, y) - \frac{\partial f}{\partial x}(0,0)}{y} = \lim_{y \to 0} \frac{(-y)^4}{(0^2 + y^2)^2} = -1\]
            В силу симметричности функции (но с минусом) $\frac{\partial f}{\partial y} = - \frac{\partial f}{\partial x} = \frac{-y^4 x - 4y^2 x^3 + x^5}{(x^2+y^2)^2}$.
            \[f_{yy}(0,0) = 0 \ \frac{\partial}{\partial x}\left[\frac{\partial f}{\partial y}\right](0,0) = 1\]
            Однако функция $g$ не дифференциируема в $(0,0)$:
            \[\frac{\partial g}{\partial x}(0,0) = 0 \ \frac{\partial g}{\partial y}(0,0) = -1\]
            Если предположить, что да, то будет выполнена дифференицальная формула:
            \[g(x,y) \underset{(x, y) \to 0}{=} 0\cdot  x - 1\cdot y + o(x, y)\]
            \[\begin{cases*} y = kt \\ x = t\end{cases*} \quad
            g(x, y) = \frac{t^5(k+4k^3 - k^5)}{t^4(1+k^2)^2} = \frac{t(k+4k^3 - k^5)}{(1+k^2)^2}\]
            Если, например, $k=1$, то $g(t, t) = t\cdot 1$, а если $g$ дифференциируема, должно быть $g(t, t) = -t + o(t)$.
            Значит, предположение не выполняется и $g$ - не дифференциируема.
            \par
            Все дело в условии того, что функция дважды дифференциируема в нуле, в контрпримере как раз при взгляде на наше определение многократной дифференцируемости
            требуется, чтобы производные были дифференциируемы на один раз меньше, чем функция. А мы только что показали, что одна из производных не дифференциируема в нуле.

        \end{example}

    \subsection{Высшие дифференциалы.}
        Пусть $f \open \Real^m \to E$, $f \in D^k(p)$. Тогда $\diff^k f(p)$ - $k$-линейная форма (линейная по каждой из $k$ переменных). 
        \[\diff^k f(p)\langle v \rangle := \sum_{\abs{\mu} = k} \frac{D^\mu f(p)}{\mu!}\cdot \overline{v}^\mu\]
        \[k = 1: \quad \diff f(p) \langle v \rangle = \sum_{\abs{\mu} = 1} \frac{1}{\mu!}D^\mu f(p) \cdot v^\mu = \sum_{i=1}^{m} \frac{\partial f}{\partial x_i}(p) v_i \]
        \[k = 2 \implies \abs{\mu} = 2: \quad \diff^2 f(p) \langle v \rangle = \sum \frac{1}{2!}\left( \frac{\partial^2 f}{\partial x_1^2 } + \hdots + \frac{\partial^2 f}{\partial x_m^2} \right) + \sum_{i < j} \frac{\partial^2 f}{\partial x^i \partial x^j}\]

        \textbf{СПРАВКА:}
        Численный вектор $\mu = (i_1, \hdots, i_m), i_k \ge 0$ называется мультииндексом длины $m$.
        Свойства:
        \begin{itemize}
            \item $\mu!:= i_1! \cdot \hdots \cdot i_m!$
            \item $x^\mu:= x_1^{i_1}\cdot x_2^{i_2} \cdot \hdots \cdot x_m^{i_m}$
            \item $\abs{\mu}:= \sum_{j=1}^{m}i_j$
        \end{itemize}

    \subsection{Пример не коммутирующих дифференциальных операторов.}

    \subsection{Теорема о разложении Тейлора.}
        \begin{theorem*}[разложение Тейлора]
            $\exists!$ многочлен $A(x)$ степени $\leq k$ такой, что $f(x) - A(x) \underset{x \to p}{=} o(x-p)^k$    
            \[A(x) = f(p) + f'(p)(x-p) + \frac{f''(p)}{2!}(x-p)^2 + \hdots + \frac{f^{(k)}(p)}{k!}(x-p)^k\]
        \end{theorem*}

        \begin{theorem*}[разложение Тейлора для нескольких переменных]
            Пусть $f: \mathbb{R}^m \to E, \ f \in D^k(p)$, тогда $\exists!$ многочлен $A(x) \ deg(A) \leq k$,
            такой, что $f(x) - A(x) \underset{x \to p}{=} o(\left| x - p\right|^k)$:
            \[A(x) = f(p) + \frac{df(p) \langle x-p \rangle}{1!} + \frac{d^f(p)\langle x-p \rangle}{2!} + \hdots + \frac{d^kf(p)\langle x-p \rangle}{k!}\]
            \begin{proof}
                \textbf{Единственность:} Пусть есть два таких многочлена $A(x), B(x)$. Введем $C(x) := A(x) - B(x) = o(\left| x - p\right|^k)$. И докажем вспомогательное утверждение:
                \begin{statement*}
                    $degC \leq k \ C(x) \underset{x \to p}{=} o(\left|x-p\right|^k)$, тогда $C \equiv 0$.
                    \begin{proof}
                        \begin{enumerate}
                            \item Фиксируем $v \in \mathbb{R}^m$ и рассмотрим $h(t) = C(p+tv)$ - многочлен одной переменной.
                            По условию $h(t) = o(t^k)$ для одной переменной (доказывали это в первом семестре), т.е. $h(t) \equiv 0$. В частности, при $t = 1 \ h(t) = C(p+v) = 0$.
                            \item Поскольку 1. выполняется $\forall v$, то $C(p+v) = 0 \  \forall v$.
                        \end{enumerate}
                    \end{proof}
                    Тогда в силу доказанного утверждения получаем единственность.
                \end{statement*}
                \par
                \textbf{Существование:}
                Введем $g(x) = f(x) - A(x), \ f:\mathbb{R}^m \to E$. $g(p) = 0$ и все производные до порядка $k$
                включительно равны $0$ в $p$, $g \in D^K(p)$. Необходимо доказать, что из этого следует, что $g(x) = o(\left| x - p\right|^k)$.
                \\
                Пусть $\varepsilon > 0$, надо показать, что $\left| g(x) \right| < \varepsilon\cdot \left| x - p\right|^k$ в некоторой $U$ - окрестности точки $p$.
                Пусть $\varepsilon_{k-1}(x)$ - какая-то производная порядка $k-1$ функции $g$, $\varepsilon_{k-1}$ определена в некотором шаре $V_p$ с центром в $p$.
                \begin{equation}
                    \label{eq1}
                    \varepsilon_{k-1}(p) = 0, \ \frac{\partial \varepsilon_{k-1}}{\partial x_i}(p) = 0 \ \forall i = 1 \hdots m
                \end{equation}
                Поэтому имеется маленький шар $U \subset V_p$ в котором выполнено:
                \begin{equation}
                    \label{eq2}
                    \forall x \in U \ \left| \varepsilon_{k-1}\right|(x) \leq \varepsilon\cdot \left| x - p \right|
                \end{equation}
                В самом деле, $\varepsilon_{k-1}(x) = \varepsilon_{k-1}(p) + d\varepsilon_{k-1}(p)\langle x - p \rangle + o(\left| x - p \right|)$, причем первое слагаемое равно нулю из того, что "все производные до порядка $k$ включительно равны $0$ в $p$", а второе - из уравнения (\ref{eq}).
                Значит $\varepsilon_{k-1}(x) = o(\left| x - p\right|) \implies \varepsilon_{k-1}(x) \leq \varepsilon\left| x - p \right|$
                Итак, ясно, что существует шарик $U$, в котором все производные $k-1$ порядка 
                имеют оценку (\ref{eq2})
                Пусть $\varepsilon_{k-2}$ - какая-то производная функции $g$ порядка $k-2$. Все ее первые частные производные по доказанному в шаре $U$ оцениваются в $\varepsilon\cdot \left| x - p\right|$.
                По лемме о степенной оценке приращения для $\varepsilon_{k-2}$ выполнено в шаре $U$:
                \[\left| \varepsilon_{k-2}(x)\right| \leq \left| \frac{\varepsilon \left| x - p\right|^2}{2}\right|\]
                Для $k-3, k-4, \hdots$ аналогично. 
                \[\left| g(x) \right| = \left| g^{(k-k)}(x) \right| \leq \varepsilon \cdot \frac{\left| x- p \right| ^ k}{k!} \leq \varepsilon\cdot \left| x - p\right|^k
                \]
            \end{proof}
        \end{theorem*}

    \subsection{Гессиан вещественной функции. Достаточные условия экстремума.}        
        \begin{definition*}
            Матрица Гессе (обозначается $H$ или $H(f)$, у нас часто $\diff^2 f(x_1, \hdots, x_n)$) - это матрица, у которой элемент на пересечении $i$-й строки и $j$-го столбца равен второй частной производной функции $f$ по переменным $x_i, x_j$:
            \[H_{ij} = \frac{\partial^2 f}{\partial x_i \partial x_j}\]
            \[H(f) = \begin{pmatrix}
                \frac{\partial^2 f}{\partial x_1^2} & \frac{\partial^2 f}{\partial x_1 \partial x_2} & \cdots & \frac{\partial^2 f}{\partial x_1 \partial x_n} \\
                \frac{\partial^2 f}{\partial x_2 \partial x_1} & \frac{\partial^2 f}{\partial x_2^2} & \cdots & \frac{\partial^2 f}{\partial x_2 \partial x_n} \\
                \vdots & \vdots & \ddots & \vdots \\
                \frac{\partial^2 f}{\partial x_n \partial x_1} & \frac{\partial^2 f}{\partial x_n \partial x_2} & \cdots & \frac{\partial^2 f}{\partial x_n^2}
            \end{pmatrix}\]
        \end{definition*}

        \begin{theorem*}[Достаточное условие локального экстремума функции многих переменных]    
            Пусть $f \in D^2(p), \ f:\mathbb{R}^m \to R$ и $df(p) > 0$. Тогда:
            \begin{enumerate}
                \item $d^2f(p) > 0$ - строгий локальный минимум
                \item $d^2f(p) < 0$ - строгий локальный максимум
                \item Если $d^2f(p)$ знаконеопределен, т.е. $\exists u \in \mathbb{R}^m \ d^2f(p)\langle u \rangle > 0$ и $\exists v \in \mathbb{R}^m \ d^2f(p)\langle v \rangle < 0$, то $p$ - седловая точка.
            \end{enumerate}

            \begin{proof}
                
                Докажем пункт 3:
                \begin{proof}
                    Пусть $df(p) = 0$ и существуют вектора $u$ и $v$, такие, что $d^2f(p)\langle u \rangle > 0, \ d^2f(p)\langle v \rangle < 0$.
                    
                    Введем функцию $h(t) = f(p + tu)$. Тогда $h'(0) = df(p)\langle u \rangle = 0, \  h''(0) = d^2f(p)\langle u \rangle > 0$. Значит у функции $h$ в точке $0$ строгий минимум 
                    (по достаточному условию экстремума для одной переменной). Аналогично вдоль $p + tv$ функция имеет строгий максимум, значит $p$ - седловая точка.
                \end{proof}
                
                Докажем пункт 1:
                \begin{proof}
                    Пусть $d^2f(p) > 0$, то есть $\forall v \neq 0 \ d^2 f(p)\langle v \rangle > 0$. Сфера $S^{m-1} = \{v \in \mathbb{R}^m \quad \left| v \right| = 1\}$ - компактна (замкнута и ограничена). $d^2f(p):S^{m-1} \to \mathbb{R}$ - однородный многочлен второго порядка.
                    Так как $d^2f$ - непрерывная функция на компакте, то у нее $\exists \min = C > 0$, т.е.\\ $\forall v \in S^{m-1} \quad d^2f(p)\langle v \rangle \geq C$.
                    \begin{statement*}
                        Тогда $\forall v \neq 0 \ d^2f(p)\langle v \rangle \geq C\cdot \left| v \right|^2$
                        \begin{proof}
                            \[\forall v \neq 0 \quad d^2f(p)\langle v \rangle = d^f(p)\langle \left| v \right| \cdot \frac{v}{\left| v \right|} \rangle = 
                            \left| v \right|^2\cdot d^2f(p)\langle \frac{v}{\left| v \right|} \rangle \geq C\cdot \left| v \right|^2\]
                        \end{proof}
                    \end{statement*}

                    Значит 
                    \[
                        f(x) = f(p) + df(p)\langle x -p \rangle + \frac{d^2f(p)\langle x - p \rangle}{2!} + \alpha(x)\left| x - p\right|^2, \ \alpha(x) \underset{x \to p }{\to} O(1)
                    \]
                    \[f(x) \geq f(p) + 0 + \frac{C}{2!}\cdot \left| x - p \right|^2 + \alpha(x)\left| x - p\right|^2\]
                    Существует окрестность $U$ точки $p$, такая, что $\left| \alpha(x) \right| \leq \frac{C}{3} \ \forall x \in U$. Тогда для $\forall x \in U$:
                    \[f(x) \geq f(p) + \frac{C}{2!}\left| x - p \right|^2 - \frac{C}{3}\left| x - p \right|^2 = f(p) + \frac{C}{6}\left| x - p\right| ^2\]
                    То есть $f(x) - f(p) \geq \frac{C}{6}\left| x - p \right|^2 > 0 \implies$ в $U \ f(p) < f(x) \ \forall x \in U$. Пункт 1 доказан. 
                \end{proof}
                Пункт 2 доказывается аналогично пункту 1.
            \end{proof}
        \end{theorem*}


    \subsection{Функции класса $C^k$ на открытом множестве (и на множествах, не обязательно открытых).}
        Символ $\open$ обозначает "открыто в". Контекст:
        \[U \open \mathbb{R}^m, \ f:U \to \mathbb{R}^k, \ f \in C^r(U), \ r \geq 0\]

        \begin{definition*}
            Отображение $f$ называется $r$-гладким, если все ее частные производные до порядка $r$ непрерывны на $U$.
        \end{definition*}

        Пусть $X$ - не обязательно открыто в $\mathbb{R}^m$.
        \begin{definition*}
            $f \in C^r(X)$, если $f = \tilde{f}$ - сужение на $\tilde{X} \supset X$, $\tilde{f}: \tilde{X} \open \mathbb{R}^m \to \mathbb{R}^k$ - $C^r$-гладкая на $\tilde{X}$.
        \end{definition*}

        \begin{statement*}
            Пусть $f:X \subset \mathbb{R}^m \to \mathbb{R}^k, \ g:X \to \mathbb{R}^k$ - $C^r$ отображения. Тогда $f+g \in C^r(X)$
            \begin{proof}
                Пусть $f = \tilde{f}, \ g = \tilde{g}$ и т.д. по определению $r$-гладкости:
                \[\tilde{f}:U \to \mathbb{R}^m, \tilde{g}: V \to \mathbb{R}^m, \ U, V \open \mathbb{R}^m, \ X \subset U, X \subset V\]
                Введем $U \cap V = W \open \mathbb{R}^m$. На $W$ заданы оба отображения и ясно, что $f+g = \tilde{f} + \tilde{g}$.
            \end{proof}
        \end{statement*}

        \begin{statement*}
            Композиция:
            \[X \overset{f}{\to} \mathbb{R}^k \supset Y \overset{g}{\to} \mathbb{R}^m\]
            Если $f \in C^r$ и $g \in C^r$, то $g \circ f \in C^r$.
            \begin{proof}
                Область определения $\mathrm{dom}(g \circ f) = \{x \in X | \ f(x) \in Y\} = X \cap f^{-1}(\mathrm{dom}(g))$
                \[\begin{cases}
                    f \in C^r \implies f = \tilde{f}, \ \tilde{f}:\mathbb{R}^m \underset{op}{\supset}\tilde{X} \to \mathbb{R}^k -  C^r\text{-гладкое.} \\
                    g \in C^r \implies g = \tilde{g}, \ \tilde{g}:\mathbb{R}^k \underset{op}{\supset}\tilde{Y} \to \mathbb{R}^m -  C^r\text{-гладкое.} \\
                \end{cases}\]
                \[\mathrm{dom(\tilde{g}\circ \tilde{f})} = \mathrm{dom}(\tilde{f}) \cap \tilde{f}^{-1}(\mathrm{dom}(\tilde{g})) = \tilde{X} \cap \tilde{f}^{-1}(\tilde{Y})\]
                $\tilde{X}$ - открытое, $\tilde{f}^{-1}(\tilde{Y})$ - открытое, как прообраз открытого множества $\tilde{Y}$ при непрерывном отображении.
                \newline
                Ясно, что $g \circ f = \tilde{g} \circ \tilde{f}$ - сужение $\mathrm{dom}(g \circ f)$.
            \end{proof}
        \end{statement*}


    \subsection{Теорема об открытости отображения вида $f(x) = x +\varphi(x)$, где $\varphi$ - сжимающее отображение.}
        \begin{theorem*}
            Пусть $U \open \mathbb{R}^m$, а $f:U \to \mathbb{R}^m$ такое, что отображение $f(x) - x = \lambda(x)$
            сжимающее, то есть $\forall x_1, x_2 \in U \left| \lambda(x_1) - \lambda(x_2) \right| \leq \lambda < 1$. Тогда:
            \begin{enumerate}
                \item $f(U) \open \mathbb{R}^m$
                \item Сужение $f: U \to f(U)$ обратимо и обратное отображение - липшицево с константой $\frac{1}{1- \lambda}$.        
            \end{enumerate}

            \begin{proof}
                Докажем пункт 2:
                \newline
                $f$ инъективно: $x_1 \neq x_2 \implies f(x_1) \neq f(x_2)$
                \[\begin{cases}
                    f(x_1) - x_1 = \lambda(x_1) \\
                    f(x_2) - x_2 = \lambda(x_2) 
                \end{cases} \implies
                \begin{cases}
                    f(x_1) = \lambda(x_1) + x_1 \\
                    f(x_2) = \lambda(x_2) + x_2
                \end{cases}\]

                Тогда
                \begin{align*}
                    \left| f(x_1) - f(x_2) \right|
                    &= \left| \lambda(x_1) - \lambda(x_2) + (x_1 - x_2) \right| \\
                    &\leq \left| \lambda(x_1) - \lambda(x_2) \right| + \left| x_1 - x_2 \right| \\
                    &\leq \lambda \left| x_1 - x_2 \right| + \left| x_1 - x_2 \right| \\
                    &= (\lambda + 1) \left| x_1 - x_2 \right|.
                \end{align*}

                В силу неравенства треугольника:
                \[(1 - \lambda) \left| x_1 - x_2 \right| \leq \left| f(x_1) - f(x_2) \right| \leq (1 + \lambda)\left| x_1 - x_2\right|\]
                Инъективность есть, а сужение $f: U \to f(U)$ - биективно, значит обратимо. Поймем, что обратное отображение будет $\frac{1}{1 - \lambda}$ липшицево.
                Пусть $\begin{cases}
                    y_1 = f(x_1) \\ \ y_2 = f(x_2) 
                \end{cases} \in f(U) \quad \begin{cases}
                    x_1 = g(y_1) \\ 
                    x_2 = g(y_2)
                \end{cases}$
                \[\left| y_1 - y_2 \right| \geq (1 - \lambda)\left|g(y_1) - g(y_2) \right| \quad \implies \quad \frac{1}{1 - \lambda}\left| y_1 - y_2 \right| \geq g(y_1) - g(y_2)\] 
                Пункт 2 доказан.
                \newline
                Пусть теперь $q \in f(U)$. Рассмотрим $p \in U \mid q = f(p)$. $U$ открыто, а значит $\exists \varepsilon > 0 \mid B_\varepsilon(p) \subset U$, \newline 
                где $B_\varepsilon(p)$ - открытый шар радиуса $\varepsilon$ с центром в точке $p$. Мы покажем, что множество $f(U)$ содержит шар с центром в $q$ радиуса $(1- \lambda)\varepsilon$. \newline
                Пусть $y \in B_{\varepsilon(1-\lambda)}(q)$, т.е. $\left| q - y \right| < (1 - \lambda)\varepsilon$. 
                Надо показать, что $\exists x \text{ такой, что } \left| p - x \right| < \varepsilon, \ f(x) = y$.
                Воспользуемся теоремой о неподвижной точке сжимающего отображения. Перепишем условие:
                \[f(x) = y \implies y - f(x) = 0 \implies y-f(x) + x = x\]
                Положим
                \[\varphi(x) = y - f(x) + x = y - \lambda(x)\]
                Заметим, что $\varphi(x)$ является сжимающим и покажем, что $\varphi$ переводит $B_\varepsilon(p)$ в себя. 
                \[x \in B_\varepsilon(p) \implies \left| x - p \right| \leq \varepsilon \implies \left| \varphi(x) - \varphi(p) \right| \leq \lambda\varepsilon\]
                \[\left| \varphi(x) - p \right| = \left| (\varphi(x) - \varphi(p)) + (\varphi(p) - p)\right| \leq \left| \varphi(x) - \varphi(p) \right| + \left|\varphi(p) - p\right|\]
                Первое слагаемое, как мы уже доказали, не превышает $\lambda \varepsilon$. Преобразуем второе:
                \[\left| \varphi(p) - p \right| = \left| y - f(p) + p - p \right| = \left| y - f(p) \right| = \left| y - q\right| \leq (1- \lambda)\varepsilon\]
                Тогда:
                \[\left| \varphi(x) - \varphi(p) \right| + \left|\varphi(p) - p\right| \leq \lambda\varepsilon + (1 - \lambda) \varepsilon = \varepsilon\]
                Значит $\left|\varphi(x) - p\right| \leq \varepsilon$ и $\varphi(x) \in B_\varepsilon(p) \mid x \in B_\varepsilon(p).$
            \end{proof}
        \end{theorem*}
\end{document}



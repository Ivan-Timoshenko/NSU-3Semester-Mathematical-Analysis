\section{Поточечная и равномерная сходимость}


Пусть $X$ - некое множество, $Y$ - $\mathbb{R}$ или $\mathbb{R}^n$ или вообще любое метрическое пространство.
Даны последовательность функций $f_n: X \to Y, n = 1, 2, \hdots$, и функция $f:X \to Y$.

\begin{definition}
    Последовательность функций сходится на $X$ к функции $f$, поточечно, если $\forall x \in X$ последовательность значений $f_n(x)$ сходится к $f$ при $n \to \infty$.
    \[\forall x \in X \quad f_n(x) \underset{n \to \infty}{\to} f(x)\]
\end{definition}

\begin{statement}
    Из непрерывной функций $f_n$ не всегда следует непрерывность функции $f$
    \begin{proof}
        Пусть $X = \left[0, 1\right]$, $f_n(x) = x^n$. Тогда $f_1(x) = x, f_2(x) = x^2, f_3(x) = x^3$ и так далее.
        В это случае    
        \[\underset{x \in \left[0, 1\right]}{\lim_{n \to \infty}} = \begin{cases}
            0 \ x < 1 \\ 
            1 \ x = 1
        \end{cases}\]
        Видно, что $f(x)$ разрывна, хотя все $f_n$ непрерывны.
    \end{proof}
\end{statement}

\begin{definition}
    Пусть $f_n: X \to \mathbb{R}$, $f:X \to \mathbb{R}$.\\
    Последовательность $f_n$ сходится к $f$ \textbf{равномерно} на множестве $X$, если 
    \begin{equation}
        \label{uniform_convergence}
        \underset{x \in X}{\sup} \left|f_n(x) - f(x)\right| \underset{n \to \infty}{\to} 0
    \end{equation}

    \par
    "Оригинальное"  определение равномерной сходимости:
    
    \begin{equation}
        \label{original_uniform_convergence}
        \forall \varepsilon > 0 \ \exists n_0 \mid \forall x \in X \ \forall n \geq n_0 \quad \left| f_n(x) - f(x) \right| \leq \varepsilon
    \end{equation}

    \begin{proof}
        Докажем эквивалентность двух определений:\par
        \eqref{uniform_convergence} $\implies$ \eqref{original_uniform_convergence}:\\
        Пусть $\varepsilon > 0$, надо подобрать $n_0$. По условию \eqref{uniform_convergence}:
        \[\exists n_0 \mid \forall n \geq n_0 \quad \underset{x \in X}{\sup} \left| f_n(x) - f(x) \right| \leq \varepsilon\]
        Но $\forall x \in X \ \sup \geq \left| f_n(x) - f(x) \right|$, значит 
        \[\forall x \in X \ \left|f_n(x) - f(x) \right| \leq \sup \left| f_n(x) - f(x) \right| \leq \varepsilon\]
    \end{proof}
\end{definition}

\begin{definition}
    Последовательность $f_n(x)$ поточечно сходится к $f(x)$ при любых $x$, если 
    \[\forall \varepsilon > 0 \ \forall x \in X \ \exists n_0 \in \mathbb{N} \ \forall n \geq n_0 \ \left| f_n(x) - f(x)\right| \leq \varepsilon\]

    \par
    \eqref{original_uniform_convergence} $\implies$ \eqref{uniform_convergence}: Надо доказать, что $\forall \varepsilon \ \exists n_0 \mid \forall n \geq n_0 \ \underset{x \in X}{\sup} \left| f_n(x) - f(x) \right| \leq \varepsilon$.
    
    По условию \eqref{original_uniform_convergence}:
    \[\exists n_0 \mid \forall n \geq n_0 \ \forall x \in X \ \left| f_n(x) - f(x) \right| \leq \varepsilon \implies sup \left| f_n(x) - f(x) \right| \leq \varepsilon\]
\end{definition}

Равномерную сходимость $f_n(x)$ к $f(x)$ будем обозначать $f_n \underset{n \to \infty}{\rightrightarrows} f$


\begin{theorem}[Критерий отсутствия равномерной сходимости]
    $f_n$ не сходится равномерно к $f$ $\iff$ $\exists x_n \in X$ такая, что $f_n(x_n) - f(x_n) \nrightarrow 0$
\end{theorem}

\begin{theorem}[о пределе пределов]
    Пусть $X \subseteq \mathbb{R}$ (или $\forall$ метрического пространства). \\ Пусть $\begin{cases} f: X \to \mathbb{R} \\ f_n: X \to \mathbb{R} \end{cases}$, $p$ - предельная точка $X$.

    Предположим, что $f_n(x) \underset{x \to p}{\to} y_n$ и $y_n \underset{n \to \infty}{\to} y$. 
    Если $f_n$ \textbf{равномерно} сходится к $f$ ($f_n \rightrightarrows f$) на $X$, то $f(x) \to y$ при $x \to p$.
    \[\lim_{n \to \infty} (\lim_{x \to p} f_n(x)) = \lim_{x \to p}(\lim_{n \to \infty} f_n(x))\] 

    \begin{proof}
        Пусть $\varepsilon > 0$, надо доказать, что у точки $p$ имеется окрестность $U$, такая, что $\forall x \in X \cap U \ \left| f(x) - y \right| \leq \varepsilon$.
        \[\left| f(x) - y \right| = \left| f(x) - f_n(x) + f_n(x) - y_n + y_n - y \right| \leq \left| f(x) - f_n(x) \right| + \left| f_n(x) - y_n \right| + \left| y_n - y \right|\] 
        Рассмотрим каждое слагаемое в отдельности:\\
        1:
         \[\exists n_0 \mid \forall n \geq n_0 \quad \left| f(x) - f_n(x) \right| \leq \frac{\varepsilon}{3}\]
        
        3:
            \item \[\exists n_1 \mid \forall n \geq n_1 \quad \left| y_n - y \right| \leq \frac{\varepsilon}{3}\]
            Пусть $n_2 = \max{n_0, n_1}$. Тогда $\forall n \geq n_2$ выполнено и 1, и 3.\\
        2:
            \[f_{n_2}(x) \underset{x \to p}{\to} y_{n_2} \implies \exists U \text{ - окрестность точки $p$, такая, что } \forall x \in U \cap X \ \left| f_{n_2}(x) - y_{n_2} \right| \leq \frac{\varepsilon}{3}\]
        
        Для таких $x$ из $U \cap X$ выполнено:
        \[\left| f(x) - y\right| \leq \left| f(x) - f_{n_2}(x) \right| + \left| f_{n_2}(x) - y_{n_2} \right| + \left| y_{n_2} - y \right| \leq \frac{\varepsilon}{3} + \frac{\varepsilon}{3} + \frac{\varepsilon}{3} = \varepsilon\]
    \end{proof}
\end{theorem}

\begin{theorem*}[Следствие]
    Если все $f_n$ непрерывны в $p \in X$ и $f_n \rightrightarrows f$ на $X$, то $f$ тоже непрерывна в точке $p$.\\
    Нет, тот контрпример показывал, что просто непрерывности $f_n$ недостаточно, а следствие утверждает, что непрерывности + равномерной сходимости уже достаточно.
    \begin{proof}
        Надо показать, что $f(x) \underset{x \to p}{\to}f(p)$. Заметим, что $f_n(x) \to f(x)$ при $n \to \infty$ в силу обычной поточечной сходимости. Теперь по теореме:
        \[f(x) = \lim_{n \to \infty} f_n(x) \implies \lim_{x \to p} f(x) = \lim_{x \to p} (\lim_{n \to \infty} f_n(x)) \overset{\text{по теореме}}{=} \lim_{n \to \infty} (\lim_{x \to p} f_n(x)) = \lim_{n \to \infty} f_n(p) = f(p)\]
    \end{proof}

\end{theorem*}

\begin{theorem*}[Еще одно следствие]
    Если все $f_n$ непрерывны \textbf{на $X$} и $f_n \rightrightarrows$, то $f$ тоже непрерывна на $X$.
    \begin{proof}
        Нечего тут доказывать: определение непрерывности в точке расширяем до определения непрерывности на множестве.
    \end{proof}
\end{theorem*}

\begin{theorem*}[Критерий Коши равномерной сходимости последовательности функций]
    Пусть $X$ - множество, $f_n: X \to \mathbb{R}$. 
    \[\exists \text{функция } f: X \to \mathbb{R} \text{ такая, что } f_n \rightrightarrows f \text{ на $X$} \implies \forall \varepsilon \ \exists n_0 \mid \begin{array}{c} \forall x \in X \\ \forall k, l \geq n_0\end{array} \ \left| f_k(x) - f_l(x) \right| \leq \varepsilon\]

    \begin{proof}
        "$\impliedby$":\par
        Пусть $x \in X$. Тогда для числовой последовательности $f_n(x)$ выполнен обычный критерий Коши сходимости последовательности, поэтому $\exists \lim_{n \to \infty}f_n(x) =:f(x)$ (обозначим $f(x)$ равной этому пределу).
        \\
        Осталось показать, что $f_n$ сходится к $f$ равномерно. Пусть $\varepsilon > 0$. По условию Коши выполнено:
        \[\exists n_0 \mid \forall k, l \geq n_0 \ \left| f_k(x) - f_l(x) \right| \leq \varepsilon\]
        С помощью поточечной сходимости $f_l$ к $f_x$ можем перейти к пределу по $l \to \infty$:
        \[\forall x \in X \ \forall k \geq n_0 \ \left| f_k(x) - f(x) \right| \leq \varepsilon\]
        "$\impliedby$" доказано.

        "$\implies$": \par
        Пусть $\varepsilon > 0$. По условию \[\exists k \geq n_0 \ \forall x \in X \left| f_k(x) - f(x) \right| \leq \frac{\varepsilon}{2}\]
        Имеем: если $k, l \geq n_0$, то $\forall x$ выполнено:
        \[\left| f_k(x) - f_l(x) \right| \leq \left| f_k(x) - f(x) \right| + \left| f(x) - f_l(x) \right| \leq \frac{\varepsilon}{2} + \frac{\varepsilon}{2} = \varepsilon\]

    \end{proof}
\end{theorem*}

\begin{statement*}[Следствие]
    Пространство непрерывных функций на множестве $\left[a, b\right] \subset \mathbb{R}$ полно в $\sup$-норме. $C(x)$ - множество непрерывных функций на $x = \left[a, b\right] \to \mathbb{R}$, $\parallel f  \parallel_c  = \underset{x \in X}{\sup}$
    \begin{proof}
        Полнота: если последовательность функций удовлетворяет критерию Коши, то есть предел $f_n \to f$ в $\sup$-норме $\implies \underset{x \in X}{\sup} \left| f_n(x) - f(x) \right| \underset{n \to \infty}{\to} 0$.\\
        Условие Коши для последовательностей $f_n \in C\left[a, b\right]$ как раз и означает, что $f_n \rightrightarrows f$ - какой-то функции. Функции $f_n$ непрерывны и $\rightrightarrows f \implies f \in C\left[a, b\right]$

    \end{proof}
\end{statement*}

\begin{theorem*}[Дини]
    Пусть $X$ - компакт, $f_n: X \to \mathbb{R}, f:X \to \mathbb{R}, \ f_n, f \in C(X)$. \\
    Если $f_n$ сходится к $f$ на $X$ поточечно, то сходимость равномерна.

    \begin{proof}
        Ключ к доказательству: $\forall x$ последовательность $\left| f_n(x) - f(x) \right|$ монотонно убывет к $0$. \\
        Пусть $\varepsilon > 0$. Надо понять, что:
        \[\exists n_0 \mid \forall n \geq n_0 \ \forall x \in X \ \left| f_n(x) - f(x) \right| \leq \varepsilon\]
        Положим $U_n \subset X$, $U_n = \left\{ x \in X \mid \left| f_n(x) - f(x) \right| \right\} < \varepsilon$.
        \begin{enumerate}
            \item $\bigcup_{n=1}^{\infty} U_n = X$. В самом деле, $\forall x \forall n$ начиная с какого-то $n_0$ выполнено $\left| f_n(x) - f(x) \right|\leq \varepsilon$ (поточечная сходимость из условия) и $x \in U_n \ \forall n \geq n_0$.
            \item Все $U_n$ открыты в $X$ (по функциональному признаку - $\left| f_n(x) - f(x) \right|$ непрерывна и у нас есть неравенство).
            \item $X$ компактно $\implies$ существует конечное подпокрытие $U_{n_1} \cup \hdots \cup U_{n_k} = X$.
        \end{enumerate}
        Заметим, $U_n \subset U_{n+1}$, так как $\left| f_{n+1} - f(x) \right| \leq \left| f_n(x) - f(x) \right|$ в силу ключа.
        \[x \in U_n \implies \left| f_n(x) - f(x) \right| < \varepsilon\]
        Так как \[\left| f_n(x) - f(x) \right| \geq \left| f_{n+1}(x) - f(x) \right| \implies \text{ тоже $<\varepsilon$} \implies x \in U_{n+1}, U_{n+1} \supset U_n\]
        Значит, $\exists$ какое-то $n_0 \mid U_{n_0} = X$. Итак, $\forall x \left| f_{n_0}(x) - f(x) \right| < \varepsilon$ и для всех $n \geq n_0$ тоже.
    \end{proof}
\end{theorem*}


\subsection{Функциональные ряды}

\begin{definition}
    Функциональный ряд $\label{stand_func_series} \sum_{n = 1}^{\infty} f_n(x)$ сходится поточечно, если последовательность функций $S_n(x) = \sum_{k=1}^n f_k(x)$ сходится поточечно (т.е. сходится при $\forall x$).
    \par
    Ряд функций сходится на множестве $X$ равномерно к функции $f$, если $S_n \underset{n \to \infty}{\rightrightarrows} f$ на множестве $X$.
\end{definition}

\begin{theorem*}[Критерий Коши равномерной сходимости функционального ряда]
    Ряд $\eqref{stand_func_series}$ сходится равномерно тогда и только тогда, когда:
    \[\forall \varepsilon \ \exists n_0 \mid \forall k, l: l > k \geq n_0 \ \forall x\in X \ \left| \sum_{i=k}^{l} f_n(x) \right| < \varepsilon\]

    \begin{proof}
        Сумма $\sum_{k}^{l} f_k(x) = S_l(x) - S_{k-1}(x)$ - задача сведена к критерию Коши для последовательности фукнций частичных сумм (числовой последовательности).
    \end{proof}
\end{theorem*}

\begin{theorem*}[Признак сравнения]
    Пусть числовой ряд $\sum a_n$, где $a_n \geq 0$ сходится, если функции $f_n: X \to \mathbb{R}$ таковы, что $\underset{x \in X}{\sup} f_n(x) \leq a_n$ (эквивалентно $\forall x \in X \mid f_n(x) \leq a_n$).
    Тогда функциональный ряд $\sum f_n$ сходится на $X$ равномерно.

    \begin{proof}
        Покажем, что для ряда $\sum f_n$ выполнено условие Коши:\\
        Пусть $\varepsilon > 0$. Ряд $\sum a_n$ сходится $\implies$ для этого числового ряда выполнено условие Коши числовых рядов:
        \[\exists n_0 \forall k > l \geq n_0 \ \left| \sum_{m=k}^{l} a_m\right| \leq \varepsilon\]
        Тогда $\forall x \in X$ имеем:
        \[\left| \sum_{m=k}^{l} f_m(x)\right| \leq \sum_{m=k}^{l} \left| f_m(x) \right| \leq \sum_{m=k}^{l}a_m \leq \varepsilon\]
        Значит условие Коши выполнено для функционального ряда $\implies$ он сходится равномерно.
    \end{proof}      
\end{theorem*}

\begin{theorem*}[Необходимое условие равномерной сходимости функционального ряда]
    Ряд $\sum f_n$ сходится на множестве $X$ равномерно $\implies$ $\underset{x \in X}{\sup} \left| f_n(x) \right| \underset{n \to \infty}{\to} 0$
    \begin{proof}
        \[\underset{x \in X}{\sup} \left| f_n(x) \right| = \underset{x \in X}{\sup} \left| \sum_{m=n}^{n} f_n(x) \right|\]
        По условию Коши это стремится к $0$ при $n \to \infty$.
    \end{proof}
\end{theorem*}

\begin{statement*}
    Пусть ряд $S(x) = \sum f_n(x)$ равномерно сходится на $X$. Если все $f_n$ непрерывны, то $f$ - непрерывна.
    \begin{proof}
        Все $S_n(x) = (f_1(x) + \hdots + f_n(x))$ - непрерывные функции. $S_n \rightrightarrows f$ и применяем теорему о непрерывности предельной функции для последовательности (теорема о пределе пределов).
    \end{proof}
\end{statement*}

\begin{theorem*}[об интеграле]
    Пусть $(a, b)$ - ограниченный промежуток, $f_n$ - интегрируемые на $(a,b)$ функции.\\
    Если последовательность $f_n$ равномерно сходится на $(a,b)$, то последовательность $\int_{a}^{b} f_n$ сходится. Если при этом предельная функция $f$ ($f_n \rightrightarrows f$) интегрируема, то $\int_{a}^{b} f_n \underset{n \to \infty}{\to} \int_{a}^{b} f$.
    \begin{proof}
        Используем критерий Коши:\\
        Пусть $\varepsilon > 0$, тогда $\exists n_0 \mid \forall k,l \geq n_0$ выполнено:
        \[\underset{x \in (a,b)}{\sup} \left| f_k(x) - f_l(x) \right| \leq \frac{\varepsilon}{\left| a- b \right|}\]
        Для таких $k$ и $l$ выполнено:
        \[\left| \int_{a}^{b} f_k(x) - \int_{a}^{b} f_l(x) \right| = \left| \int_{a}^{b} f_k(x) - f_l(x) \right| \leq \int_{a}^{b} \left| f_k(x) - f_l(x) \right| \leq \int_{a}^{b} \frac{\varepsilon}{b-a} = \varepsilon\]
        То есть для числовой последовательности $\int_{a}^{b} f_n(x)$ выполнено условие Коши и она имеет предел.

        \par
        Если $f_n \rightrightarrows f$ интегрируема, то ясно, что $\int_{a}^{b} f_n - \int_{a}^{b} f \underset{n \to \infty}{\to} 0$. Важно, что $(a, b)$ ограничен.
    \end{proof}
\end{theorem*}

\subsection{Дифференциирование}
\begin{theorem}
    Пусть ограниченный отрезок $(a,b) \subset \mathbb{R}$ и $f_n$ дифференцируемы на $(a,b)$ и таковы, что 
    \begin{enumerate}
        \item $f'_n \underset{n \to \infty}{\rightrightarrows} h$ на $(a,b)$.
        \item $\exists p \in (a,b)$ такая, что последовательность $f_n(p)$ сходится.
    \end{enumerate}
    Тогда $\exists$ функция $f$ (дифференцируема на $(a,b)$) и такая, что $f_n \rightrightarrows f$ и $f' = h$.
    \par
    Теорема утверждает, что предел производных последовательности функций равен производной предела.

    \begin{proof}
        Положим $g_n(x) = \frac{f_n(x) - f_n(p)}{x-p}$, область определения $g_n$ это промежуток $(a,b) \backslash \{p\}$.\\
        Покажем, что последовательность $g_n$ равномерно сходится на своей области определения:\\
        Применим критерий Коши:
        \[g_k(x) - g_l(x) = \frac{(f_k(x) - f_l(x)) - (f_k(p) - f_l(p))}{x-p}, \text{ обозначим } f_{kl}(x) = f_k(x) - f_l(x), f_{kl}(p) = f_k(p) - f_l(p)\]
        Пусть $\varepsilon > 0$, имеем: 
        \[\exists n_0 \mid \forall k,l \geq n_0 \ \underset{x \in (a,b)}{\sup} \left| f'_k(x) - f'_l(x) \right| \leq \varepsilon, \text{ то есть, } \left| f'_{kl}(x) \right| \leq \varepsilon\]
        Тогда по теореме о среднем (теорема о приращениях):
        \[\forall x \left| f_{kl}(x) - f_{kl}(p) \right| \leq \varepsilon \cdot \left| x - p \right|\]
        Значит 
        \[g_k(x) - g_l(x) \leq \varepsilon \ \forall x \in (a,b)\{p\} \ \forall k,l \geq n_0\]
        Согласно критерию Коши последовательность функций $g_n$ сходится равномерно к $g$. Покажем, что $f_n(x)$ равномерно сходится.
        \[f_n(x) = g_n(x)(x-p) + f_n(p) \ \forall x \neq p\]
        Так как $g_n \rightrightarrows g$ и $f_n(x) \rightrightarrows f(x)$, то можем определить $f(x):=g(x)(x-p) + f_n(p)$.
        \[f_n(x) - f(x) = \left[g_n(x) - g(x)\right](x-p) + f_n(p) - f(x)\]
        Применим условие Коши к обоим слагаемым: пусть $\varepsilon > 0$, тогда
        \[\exists n_0 \mid \forall k,l \geq n_0 \ \left| g_k(x) -g_l(x) \right| \leq \frac{\varepsilon}{2(a-b)}\]
        и 
        \[(f_l(x) - f_l(p)) - (f_k(x) - f_k(p)) \leq \frac{\varepsilon}{2}\]
        Тогда 
        \[\left| f_l(x) - f_k(x) \right| \leq \frac{\varepsilon(x-p)}{2(b-a)} + \frac{\varepsilon}{2} \leq \varepsilon\]
        Итак, $f_n \rightrightarrows f$ на области определения.
        \par
        \[\lim_{n \to \infty} \lim_{x \to p}g_n(x) = \lim_{x \to p}\lim_{n \to \infty} g_n(x) \text{ по теореме о пределе пределов}\]
        Заметим, что 
        \[g_n(x) = \frac{f_n(x) - f_n(p)}{x-p} \implies \lim_{n \to \infty} \lim_{x \to p}g_n(x) = \lim_{n\to \infty} f'_n(p)\] 
        \[\lim_{x \to p}\lim_{n \to \infty}g_n(x) = \lim_{x \to p}\frac{f(x) - f(p)}{x-p} = f'(p)\]
        Эти пределы равны по теореме о пределе пределов, значит $\lim_{n \to \infty}f'_n(p) = f'(p)$.
    \end{proof}
\end{theorem}

\begin{statement*}[Следствие 1]
    Пусть ряд $\sum f_n$ сходится равномерно на ограниченном промежутке $(a,b)$.\\
    Если $f_n$ интегрируема и $f$ ($f = \sum f_n$) интгерируема, то 
    \[\int_{a}^{b} f = \int_{a}^{b} \sum f_n = \sum \int_{a}^{b} f_n\]
    \begin{proof}
        Сумма $\sum f_n$ - это предел частичных сумм $S_n(x) = \sum_{k = 1}^{n}f_k(x)$, $S_n(x) \rightrightarrows f$ и применяем теорему об интеграле.
    \end{proof}
\end{statement*}

\begin{statement*}[Следствие 2]
    Пусть ряд $\sum f_n(x)$ сходится хотя бы в одной точке $p \in (a,b)$ и ряд производных $\sum_{n=1}^{\infty} f'_n(x)$ сходится равномерно на $(a,b)$.\\
    Тогда исходный ряд сходится равномерно на всем $(a,b)$, а его сумма дифференцируема на $(a,b)$, и $(\sum f_n)' = \sum (f'_n)$.
    \begin{proof}
        Теорема о дифференцируемости, примененная к частичным суммам.
    \end{proof}
\end{statement*}

Далее будем рассматривать степенной ряд $f(x) = \sum_{n=1}^{\infty} C_n x^n$.

\begin{theorem}[Формула Коши-Адамара]
    Радиус сходимости степенного ряда можно вычислить по формуле Коши-Адамара:
    \[R = \frac{1}{{\overline{\lim}}_{n\to \infty} \sqrt[n]{\left| C_n \right|}}\]
    \begin{itemize}
        \item При $\left| x \right| < R$ - сходится.
        \item При $\left| x \right| > R$ - расходится.
        \item При $\left| x \right| = R$ - нужно более тонкое исследование.        
    \end{itemize}

    \begin{proof}
        Доказывается признаком Коши (в лекциях по ТФКП должно быть).
    \end{proof}
\end{theorem}

\begin{theorem*}
    Пусть радиус сходимости степенного ряда $\sum C_n z^n$ равен $r$. Тогда $\forall r_0 < r$ ряд сходится равномерно на множестве $\left| z \right| \leq r_0$. 

    \begin{proof}
        Выберем $r_1$ так, что $r_0 < r_1 < r$. По условию $r = \frac{1}{\overline{\lim} \sqrt[n]{\left| C_n \right|}}$. Поскольку $r_1 < r$, при больших $n$ выполнено
        $\sqrt[n]{\left| C_n \right|} < \frac{1}{r_1}$, то есть $\left| C_n \right| < \frac{1}{r_1^n}$.\\
        Если $\left| z \right| \leq r_0$, то $\left| C_n z^n \right| < \left| C_n r_0^n \right| < \left( \frac{r_0}{r_1} \right)^n$ - геометрическая прогрессия, причем $\frac{r_0}{r_1} < 1$. 
        Числовой ряд $\sum \frac{r_0}{r_1}$ сходится, значит по признаку сравнения для равномерной сходимости степенной ряд сходится равномерно на $\left| z \right| \leq r_0$.
    \end{proof}

    \textbf{Следствие:}
    Внутри (на Int) круга сходимости степенной ряд задает $\infty$-дифференцируемую функцию, при этом $\left( \sum C_n z^n \right)' = \sum nC_nz^{n-1}$.

    \begin{proof}
        Пусть $\left| z \right| < r$. Выберем $r_0 > z$, но $r_0 < r$. В круге радиуса $r_1$ сходимость ряда $\sum C_n z^n$ равномерна, более того (именно это нам и надо):\\
        ряд $\sum nC_n z^{n-1}$, полученный дифференцированием, тоже сходится равномерно (для доказательства этой фразы надо посчитать радиус сходимости нового ряда, он будет такой же, как у исходного).
        По теореме о дифференцировании функционального ряда выполнено $\left( \sum C_n x^n \right)' = \sum (nC_n z^{n-1})$ в точке $X$.
    \end{proof}
\end{theorem*}

\begin{theorem*}[Признак Дирихле равномерной сходимости ряда функций]
    Пусть ряд $\sum u_n(x)v_n(x)$ таков, что 
    \begin{enumerate}
        \item $\forall x \in X$ последовательность $u_n(x)$ монотонна по $n$ и $u_n \rightrightarrows 0$ на $X$.
        \item Частичные суммы ряда $\sum v_n(x)$ равномерно ограничены (единая константа для всех частичных сумм):
        \[\exists C < \infty \mid \forall k \in \mathbb{N} \forall x \in X \left| \sum_{n=1}^{k} v_n(x) \right| \leq C\]
    \end{enumerate}
    Тогда $\sum u_n(x) v_n(x)$ сходится равномерно на $X$.
    \\
    Справка (большая):\par
    В курсе матанализа был признак Абеля-Дирихле (А-Д) для числового ряда $\sum u_n v_n$ сходится, если 
    \begin{enumerate}
        \item Частичные суммы ряда $\sum v_n$ ограничены
        \item $u_n \to 0$ монотонно
    \end{enumerate}
    В основе доказательства признака Абеля-Дирихле для числовых рядов - неравенство Абеля:\\
    Пусть $b_1 \geq b_2 \geq \hdots \geq b_m \geq 0$ и числа $a_1, \hdots, a_m$ - какие-то. Тогда:
    \[\left| a_1b_1 + a_2b_2 + \hdots + a_mb_m \right| \leq b_1\cdot \left( \underset{k=1}{\overset{m}{\max}} \left| S_k \right| \right), \text{ где }S_k = a_1 + \hdots + a_k\]
    \begin{proof}
        Докажем неравенство Абеля (для прикола):
        \[a_1 = S_1, \ a_2 = S_2 - S_1, \ a_k = S_k - S_{k-1}\]
        \begin{align*}
            &\left| a_1b_1 + \dots + a_mb_m \right| = \\
            &\quad = \left| S_1b_1 + (S_2 - S_1)b_2 + \dots + (S_m - S_{m-1})b_m \right| = \\
            &\quad = \left| S_1(b_1-b_2) + S_2(b_2 - b_3) + \dots + S_{m-1}(b_{m-1} - b_m) + S_m b_m \right| \leq \\
            &\quad \le \sum_{k=1}^{m-1} \left| S_k \right| (b_k - b_{k+1}) + \left| S_m \right| b_m  \le\\
            &\quad \le \max_{1 \le k \le m} \left| S_k \right| \Bigl( (b_1 - b_2) + (b_2 - b_3) + \dots + (b_{m-1} - b_m) + b_m \Bigr) =\\
            &\quad = \max_{1 \le k \le m} \left| S_k \right| \cdot b_1
        \end{align*}
    \end{proof}
    
    \begin{proof}
        Докажем признак Дирихле для рядов функций:\\
        Воспользуемся критерием Коши: \par
        Пусть $\varepsilon > 0$, $u_n$ сходится равномерно к $0$ на $X$, значит $\exists n_0 \mid \forall n \ge n_0 \ \forall x \in X \left| u_n(x) \right| \le \frac{\varepsilon}{2C}$ ($C$ - ограничивающая константа для $v_n$).
        Пусть $X_- = \{ x \mid u_n \searrow \}, X_+ = \{ x \mid u_n \nearrow\}$. \par
        По неравенству Абеля:\\
        $\forall x \in X_- \ \forall l \ge k > n_0$ выполнено 
        \[\left| \sum_{k}^{l} u_n(x) v_n(x) \right| \le u_k\cdot S, \text{ где } S = max \sum_{n=k}^{\tilde{l}} v_n(x), \ \tilde{l} \le l\]
        \[\left| \sum_{k}^{\tilde{l}} \right| = \left| \sum_{1}^{\tilde{l}} - \sum_{1}^{\tilde{l} - 1} \right| \le \left| \sum_{1}^{\tilde{l}} \right| + \left| \sum_{1}^{k} \right| \le C + C \text{ (по второму условию из признака)}\]
        Итак, $\sum_{k}^{l} u_n(x) v_n(x) \le u_k\cdot 2C \le \frac{\varepsilon}{2C} \cdot 2C = \varepsilon$. Условие Коши выполнено и ряд сходится равномерно на $X_-$. Для $X_+$ аналогично, значит на $X$ тоже равномерно.
    \end{proof}   
\end{theorem*} 

\begin{statement*}
    Ряд $\sum \frac{sin(nx)}{n}$ сходится равномерно на $x \in \left[ \delta, 2\pi - \delta \right], \ \delta > 0$.
    
    \begin{proof}
        Для таких $X$ выполнено $\left| \sum_{n=0}^{k} \frac{\sin(nx)}{n} \right| \le \frac{2}{\left| 1 - e^{i\delta} \right|}$, то есть для ряда $\sum \frac{\sin(nx)}{n}$ выполнено условие (2) из признака А-Д. 
        Поскольку $u_n(x) \rightrightarrows 0$ и монотонна по $n$, значит условие (1) тоже выполнено. Тогда по признаку А-Д ряд $\sum \frac{sin(nx)}{n}$ сходится равномерно.
    \end{proof}
\end{statement*}

\begin{theorem*}[Признак Абеля]
    Пусть ряд $\sum u_n(x)v_n(x)$ таков, что 
    \begin{enumerate}
        \item Любая последовательность $u_n(x)$ для каждого $x$ монотонна и все $u_n(x)$ равномерно ограничены:
        \[\exists M < \infty \mid \forall n \ \forall x \in X \ \left| u_n(x) \right| \le M\]
        \item Ряд $\sum v_n(x)$ \textbf{равномерно} сходится 
    \end{enumerate}
    Тогда $\sum u_n(x)v_n(x)$ сходится равномерно. 
    \begin{proof}
        Этот признак не следует напрямую из признака Дирихле. \\
        Воспользуемся тем же Критерием Коши: \par
        Пусть $\varepsilon > 0, X_-, X_+$ такие же, как раньше. Применим К.К. к ряду $\sum v_n(x)$ и получим:
        \[\exists n_0 \mid \forall k, l \ge n_0 \ \forall x \in X \ \left| \sum_{n=k}^{l} v_n(x)\right| \le \frac{\varepsilon}{C}\]
        Константа $C$ такая, что ограничивает все $u_n(x) \ \forall n \forall x$.
        \begin{align*}
            &\forall k, l \ge n_0 \ \forall x \in X_- \quad \left| \sum_{k}^{l} u_n(x)v_n(x) \right| \overset{\text{нер-во Абеля}}{\le} \\
            & \le u_k \cdot \left| \text{самое большое из чисел }v_k(x), (v_k(x) + v_{k+1}), \hdots, (v_k(x) + \hdots + v_l(x)) \right| \le C \cdot \frac{\varepsilon}{C} \le \varepsilon
        \end{align*}
        Выполнено условие Коши, значит ряд $\sum u_nv_n$ сходится равномерно на $X_-$. Для $X_+$ аналогично.
        \par
        В рассуждении использовалось то, чего не было в условии: мы предполагали (применяя неравенство Абеля), что $u_n \ge 0$. Этого может не быть, как спасти ситуацию?\\
        Пусть $\tilde{u_n}(x) = u_n(x) + C \in \left[0, 2C \right] \ \forall n \ \forall x \in X$
        \[\sum u_n(x) v_n(x) = \sum (\tilde{u_n}(x) - C) v_n(x) \le \sum \tilde{u_n}(x)v_n(x) - C \cdot \sum v_n(x)\]
        Для первого слагаемого доказательство, приведенное выше работает, а второе ряд сходится равномерно. Осталось показать, что $\sum \lambda a_n(x) + \gamma b_n(x)$ равномерно сходится (упражнение, но кажется достаточно расписать через определение).

    \end{proof}
\end{theorem*}

\begin{theorem*}[Теорема Абеля о степенных рядах]
    Пусть числовой ряд $\sum_{n=1}^{\infty} a_n$ сходится, $a_n \in \Compl$. 
    Тогда на $\left[ 0, 1 \right]$ степенной ряд $f(t) = \sum a_n t^n$ сходится равномерно и, значит, $f(t)$ непрерывна как функция от $t$ на отрезке $\left[0, 1\right]$.

    \begin{proof}
        Согласно признаку Абеля $\sum u_n(t)v_n(t)$ сходится равномерно, если ряд $\sum v_n(t)$ сходится равномерно, последовательность функций $u_n(t)$ равномерно ограничена константой $C$ и для любого $t$ последовательность $u_n(t)$ монотонна по $n$.\par

        Положим $v_n(t) = a_n; \ u_n(t) = t^n$.Тогда у нас выполнено условие признака Абеля:
        \begin{itemize}
            \item $\left| u_n(t) \right| \le 1 \ \forall n \ \forall t \in \left[ 0, 1 \right]$
            \item если $t \in [ 0, 1 )$, то $t^n \underset{\text{монотонно}}{\searrow} 0$
            \item если $t = 1$, то $t^n \equiv 1$ тоже монотонна!
        \end{itemize}

        Значит $f(t)$ сходится на $t \in \left[ 0, 1\right]$ равномерно, а сумма равномерно сходящегося ряда из непрерывных функций тоже непрерывна.
    \end{proof}
\end{theorem*}

\begin{definition}[\textbf{Метод суммирования Абеля}]
    Пусть есть ряд $\sum a_n$ (неважно, сходящийся или нет). Он сходится методом суммирования Абеля, если 
    \[\exists \lim_{t \to 1 \mid_{\left[ 0, 1 \right]}} (\sum_{n=1}^{\infty} a_n t^n) \text{ (конечный)}\]
\end{definition}

\begin{theorem*}
    Пусть ряды $\sum a_n, \sum b_n$ сходятся и ряд $\sum c_n$ получен из них произведением Коши:
    \[c_n = \sum_{i+j=n} a_i b_j; \ c_0 = a_0 b_0; \ c_1 = a_1 b_0 + a_0 b_1; c_2 = a_2 b_0 + a_1 b_1 + a_0 b_2\]
    Если ряд $\sum c_n$ сходится, то его сумма равна $(\sum a_n) \cdot (\sum b_n)$.

    \begin{proof}
        Пусть $A(t) = \sum_{n=0}^{\infty} a_n t^n, \ B(t) = \sum_{n=0}^{\infty} b_n t^n$. Перемножим эти ряды по Коши:
        \[c_n(t) =  \sum_{i+j=n} a_i t^i \cdot b_j t^j = \sum_{i+j=n} a_i b_j t^{i+j = n} = c_n t^n\]
        По условию все три ряда сходятся: $\sum a_n = A,\ \sum b_n = B,\ \sum c_n = C$. 
        Значит при $\left| t \right| < 1$ ряды $\sum a_n t^n, \ \sum b_n t^n,\ \sum c_n t^n$ сходятся абсолютно (потому что мажорируются убывающей геометрической прогрессией).
        По теореме из прошлого, произведение $A(t)\cdot B(t) = C(t)$. \\
        При $t \nearrow 1$ по теореме Абеля:
        \begin{itemize}
            \item $A(t) \to A(1) = A$
            \item $B(t) \to B(1) = B$
            \item $C(t) \to C(1) = C$
        \end{itemize}
        Значит, при переходе к пределу $A\cdot B = C$.

    \end{proof}
\end{theorem*}

Долг из прошлого:
\begin{theorem*}[Теорема Мертенса (упрощенная, но достаточная для нас)]
    Пусть ряд $A = \sum a_i, \ B = \sum b_i$ сходятся абсолютно, тогда ряд 
    $\sum c_n$ сходится ($c_n = \sum_{i+j=n} a_i b_j$) и его сумма равна $A\cdot B$.
    
    \begin{proof}
        Применим критерий Коши к ряду $\sum c_n$ и покажем, что $\sum_{n=k}^{l} \left| c_n \right| \underset{k, l \to 0}{\to} 0$.
        \[\sum_{k}^{l} \left| c_n \right| = \left| \sum a_i b_j \right|, \ i+j \in \left[ k,l \right]\]
        \[\left|\sum_{n=k}^{l} c_n \right| = \left| \sum_{n=k}^{l} \sum_{i+j=n} a_i b_j \right| \le \sum_{n=k}^{l} \sum_{i+j=n} \left| a_i \right|\cdot \left| b_j \right| \le \sum_{i, j = 0}^{l} - \sum_{i, j = 0}^{\left[\frac{k}{2}\right] - 1} \left| a_i \right|\cdot \left| b_j \right|\]
        \[\sum_{i,j = 0}^{l} \left|a_i \right| \cdot \left| b_j \right| = \sum_{i=0}^{l} \sum_{j=0}^{l} \left| a_i \right|\cdot \left| b_j \right| = S_1(l) \cdot S_2(l)\]
        В этой записи $S_1(l)$ - частичная сумма ряда $\sum \left| a_i \right|$, $S_2(l)$ - частичная сумма ряда $\sum \left| b_i \right|$, $S_1(l) \to \sum \left| a_n \right|, \ S_2(l) \to \sum \left| b_n \right|$.
        В квадрате на рисунке, который я не могу сюда приложить, но его можно посмотреть в ваших лекциях, площадь квадрата с квадратным вырезом в левом верхнем углу равна 
        \[= S_1(l) \cdot S_2(l) - S_1(\left[ \frac{k}{2} \right] - 1) \cdot S_2(\left[ \frac{k}{2} \right] - 1) \underset{k, l \to \infty}{\to} 0\]
    \end{proof}
\end{theorem*}
\section{Аналитические функции}

Пусть $w = f(z) = u(x, y) + iv(x,y)$, где $z = x + iy$ - функция, определенная в области $D \in \mathbb{C}$.

\begin{definition}
    Функция $f(z)$ называется дифференцируемой (моногенной), в точке $z \in D$, 
    если 
    \[\exists! \lim_{\Delta z \to 0} \frac{\Delta f}{\Delta z} = \lim_{\Delta z \to 0} \frac{f(z + \Delta z) - f(z)}{\Delta z}, \text{ где $z+ \Delta z \in D$}\]
    Этот предел называется производной функции $f$ и обозначается $f'(z)$
\end{definition}


\begin{note}
    Если этот предел существует, то он не зависит от того, как $\Delta z$ стремится к $0$.

    \begin{proof}
        \textbf{Шаг 1}\\
        Рассмотрим $\Delta z = \Delta x \to 0 \ (\Delta y = 0)$:
        \[f'(z) = \lim_{\Delta x \to 0} \left[\frac{u(x + \Delta x, y) - u(x, y)}{\Delta x} + i \frac{v(x + \Delta x, y) - v(x, y)}{\Delta x}\right] = \frac{\partial u}{\partial x} + i \frac{\partial v}{\partial x}\]
        \textbf{Шаг 2}\\
        Пусть теперь $\Delta z = i \Delta y \to 0 \ (\Delta x = 0)$:
        \[f'(z) = \lim_{\Delta y \to 0} \left[\frac{u(x, y+\Delta y) - u(x, y)}{i\Delta y} + i\frac{v(x, y+\Delta y) - v(x, y)}{i\Delta y}\right] = \frac{\partial v}{\partial y} - i \frac{\partial u}{\partial y}\]
        Для доказательства замечания нам необходимо, чтобы эти производные были равны, тогда:
        \[\begin{cases*}
            \frac{\partial u}{\partial x} = \frac{\partial v}{\partial y}\\
            \frac{\partial v}{\partial x} = -\frac{\partial u}{\partial y}
        \end{cases*} \text{ - условие Коши-Римана}\]
    \end{proof}
\end{note}

\begin{statement}
    Если $\exists f'(z)$ в точке $z \in D$, то выполнено условие Коши-Римана, но обратное утверждение не верно.
    \begin{proof}
        Слева направо ($\implies$) доказали в предыдущем замечании.
        Докажем справа налево ($\impliedby$).
        Построим контрпример:
        \[f(z) = \begin{cases*}
            e^{-\frac{1}{z^4}} \ z \neq 0 \\
            0 \ z = 0
        \end{cases*}\]
        Заметим, что если $f(0) = 0$, то $f'(0) = \lim_{z \to 0}\frac{f(z)}{z}$
        
        Пусть $x \to 0, y = 0$:
        \[\lim_{x \to 0}\frac{f(x)}{x} = \lim_{x \to 0}\frac{e^{-\frac{1}{x^4}}}{x} = u_x + iv_x = 0 \implies u_x = 0, v_x = 0\]
        Пусть $y \to 0, x = 0$:
        \[\lim_{y \to 0}\frac{f(iy)}{iy} = \lim_{y \to 0}\frac{e^{-\frac{1}{(iy)^4}}}{iy} = v_y + iu_y = 0 \implies v_y = 0, u_y = 0\]
        То есть $u_x = v_y = 0, v_x = -u_y = 0$. Значит условие Коши-Римана выполнено в $z = 0$,
        но с другой стороны, $f(z)$ не то, что не дифференцируема, она разрывна в $z = 0$:
        \par
        Пусть $z = (1+ i)x \to 0$.
        \[\lim_{x \to 0}\frac{f((1+i)x)}{(1+i)x} = \lim_{x \to 0}\frac{e^{-\frac{1}{(1+i)^4 x^4}}}{(1+i)x} = \lim_{x \to 0}\frac{e^{\frac{1}{4x^4}}}{(1+i)x} = \infty\]
        $\infty \neq 0 \implies$ функция $f(z)$ не непрерывна в 0 $\implies$ $\nexists f'(0)$.

        
    \end{proof}
\end{statement}

\begin{statement}
    Если функции $u(x,y), v(x,y)$ дифференцируемы в $z$ и выполнено условие Коши-Римана, то $\exists f'(z)$.
    \begin{proof}
        Так как $u, v$ - дифференцируемы в $z$, то
        \begin{equation}
            \begin{aligned}
            \Delta u &= \frac{\partial u}{\partial x}\Delta x + \frac{\partial u}{\partial y}\Delta y + o(\left| \Delta z \right|) \\
            \Delta v &= \frac{\partial v}{\partial x}\Delta x + \frac{\partial v}{\partial y}\Delta y + o(\left| \Delta z \right|)
            \end{aligned}
            \text{ где $\left|\Delta z \right| = \sqrt{\Delta x^2 + \Delta y^2}$}
            \label{deltauv}
        \end{equation} 
        Обозначим: 
        \begin{equation}
            \begin{aligned}
                \frac{\partial}{\partial z} = \frac{1}{2}\left(\frac{\partial}{\partial x} - i\frac{\partial}{\partial y} \right) \\
                \frac{\partial}{\partial \overline{z}} = \frac{1}{2}\left(\frac{\partial}{\partial x} + i\frac{\partial}{\partial y}\right)
            \end{aligned}
            \label{operators}
        \end{equation}
        Заметим, что 
        \begin{equation}
            \Delta x = \frac{1}{2}\left(\Delta z + \Delta \overline{z}\right), \ \Delta y = \frac{1}{2i}\left(\Delta z - \Delta \overline{z}\right)
            \label{rewrited}
        \end{equation}
        Тогда перепишем $\eqref{deltauv}$:
        \[\Delta f = \Delta u + i \Delta v = \left( \frac{\partial u}{\partial x} + i \frac{\partial v}{\partial x}\right)\Delta x + \left(\frac{\partial u}{\partial y} + i\frac{\partial v}{\partial y}\right)\Delta y + o(\Delta z) \overset{\text{по } \eqref{operators} \text{ и }\Delta x, \Delta y}{=}\frac{\partial f}{\partial z}\Delta z + \frac{\partial f}{\partial \overline{z}}\Delta \overline{z} + o(\Delta z)\]
        \begin{equation*}
        \begin{aligned}
            \frac{\partial f}{\partial z} &= \frac{1}{2}\left(\frac{\partial}{\partial x} - i\frac{\partial}{\partial y}\right)(u+iv) = \frac{1}{2}\left[\left(\frac{\partial u}{\partial x} + \frac{\partial v}{\partial y}\right) + i\left(\frac{\partial v}{\partial x} - \frac{\partial u}{\partial y}\right)\right] \\
            \frac{\partial f}{\partial \overline{z}} &= \frac{1}{2}\left(\frac{\partial}{\partial x} + i\frac{\partial}{\partial y}\right)(u+iv) = \frac{1}{2}\left[\left(\frac{\partial u}{\partial x} - \frac{\partial v}{\partial y}\right) + i\left(\frac{\partial v}{\partial x} + \frac{\partial u}{\partial y}\right)\right]
        \end{aligned}
    \end{equation*}
    Это выражение - формальная частная производная по $z$ и $\overline{z}$. Подставив эти    формулы в $\eqref{rewrited}$ получим
    \[\Delta f = \frac{\partial f}{\partial z}\Delta z + \frac{\partial f}{\partial \overline{z}}\Delta \overline{z} + o(\left| \Delta z \right|)\]
    Заметим, что $\frac{\partial f}{\partial \overline{z}} = 0 \iff$ выполнено условие Коши-Римана.Разделим предыдущее выражение на $\Delta z \neq 0$:
    \[\frac{\Delta f}{\Delta z} = \frac{\partial f}{\partial z} + \frac{o(\Delta z)}{\Delta z}\]
    Поскольку $\frac{o(\Delta z)}{\Delta z} \underset{\Delta z \to 0}{\to} 0$, то 
    \[\exists \lim_{\Delta z \to 0} \frac{\Delta f}{\Delta z} = \frac{\partial f}{\partial z} = f'(z)\]
\end{proof}
\end{statement}

\[\lim_{\Delta z \to 0} \frac{\Delta f}{\Delta z} = f'(z) \implies \frac{\Delta f}{\Delta z} = f'(z) + \eta, \ \lim_{\Delta z \to 0} \eta = 0\]
Значит $\Delta w = \Delta f = f'(z)\Delta z + \eta \cdot \Delta z$, где $f'(z)\Delta z$ - линейная часть приращения функции относительная $\Delta  z$, она же главная часть приращения, она же дифференциал функции.

Обозначим $dw = df(z) = f'(z)\Delta z$. В частности, если $f(z) = z$, то $df(z) = dz = \Delta z \implies df(z) = f'(z)dz$ или $f'(z) = \frac{df(z)}{dz}$.

\begin{definition}
    Функция $f(z)$ называется аналитичной в области $D$, если $\forall z \in D \ \exists f'(z)$.
\end{definition}

\begin{definition}
    Функция $f(z)$ называется аналитичной в точке $z_0 \in D$ ($D$ - область), если $f(z)$ аналитична в некоторой открытой окрестности точки $z_0$.
\end{definition}

\begin{theorem*}
    Сумма степенного ряда $S(z) = \sum_{k=1}^{\infty} C_k z^k$ аналитична в круге его сходимости $\left| z \right| < R$, 
    причем $S'(z) = \sum k C_k z^{k-1}$. 
    \begin{proof}
        Пусть ряд
        \begin{equation}
            \label{series}
            S_0(z) = \sum_{k = 1}^{\infty} kC_k z^{k-1}
        \end{equation}
        Заметим, что радиус сходимости ряда $S'_0(z)$ тоже равен $R$:
        \[\overline{\lim_{k \to \infty}} \sqrt[k]{\left| kC_k\right|} = \lim_{k \to \infty} \sqrt[k]{k} \cdot \overline{\lim} \sqrt[k]{\left| C_k\right|} = R \text{ - по теореме Коши-Адамара}\]
        \[S_0(0) = C_1; \ S_0(z) = \sum kC_k z^{k-1} = \frac{1}{z}\sum kC_k z^k\]
        Заметим, что
        \[k = (1+ (k^{\frac{1}{k}}-1))^k = 1 + k(k^{\frac{1}{k}} - 1) + \frac{k(k-1)}{2}(k^{\frac{1}{k}}-1)^2 + \hdots + (k^{\frac{1}{k}} - 1)^k \implies k > \frac{k(k-1)}{2}(k^{\frac{1}{k}}-1)^2\] \[\implies \left| k^2 - 1\right| < \sqrt{\frac{2}{\sqrt{k-1}}} < \varepsilon\]
        Пусть $z$ - произвольная точка круга $\left| z \right| < R$ и $\Delta z: \left| z + \Delta z \right| < R$.

        \[\left| \frac{S(z+\Delta z) - S(z)}{\Delta z} - S_0(z)\right| \overset{?}{<} \varepsilon\]
        \begin{equation}
            \label{module}
            \left| \frac{S(z + \Delta z) - S(z)}{\Delta z} -S_0(z) \right| \leq \left| \sum \left[ \frac{C_k(z + \Delta z)^k - C_k z^k}{\Delta z} - kC_k z^k \right] \right| = \left| \sum C_k \left[ \frac{(z + \Delta z)^k}{\Delta z}  \right] \right|
        \end{equation}

        Заметим, что первое слагаемое можно расписать как 
        
        \begin{multline}
            \label{binom}
            \frac{(z+\Delta z)^k}{\Delta z} = \frac{(z+\Delta z)^{k-1}(z+\Delta z)}{\Delta z} = \\ = (z+\Delta z)^{k-1} + \frac{z(z+\Delta z)^{k-1}}{\Delta z} \overset{\text{так же}}{=} (z+\Delta z)^{k-1} + z(z+\Delta z)^{k-2} + \hdots + z^{k-1} + \frac{z^k}{\Delta z}
        \end{multline}
    

        Тогда второе слагаемое из \eqref{module} сокращается с последним слагаемым \eqref{binom} и получаем
        \begin{multline*}
            \left| \sum C_k \left[ (z + \Delta z)^{k-1} + z(z + \Delta z)^{k-2} + \hdots + z^{k-1} - kz^{k-1} \right] \right| \overset{\text{дважды нер-во } \Delta\text{-ника}}{\leq} \\
            \leq \left| \sum_{k = 1}^{N} C_k\left[(z+\Delta z)^{k-1} + z(z +\Delta z)^{k-2} + \hdots + z^{k-1} - kz^{k-1}\right]\right| + \\
            \left| \sum_{k = N-1}^{\infty} C_k \left[ (z+ \Delta z)^{k-1} + z(z+ \Delta z)^{k-2} + \hdots + z^{k-1} \right]\right| + \left| \sum_{k=N-1}^{\infty} kC_k z^{k-1}\right|
        \end{multline*}
        Возьмем число $r: 0 < r < R$ и $\left| z + \Delta z \right| < r$. Из аболютной сходимости ряда \eqref{series} в круге $\left| z \right| < R \implies \forall \varepsilon > 0 \ \exists N = N(r, \varepsilon)$
        Введем новый ряд 
        \[\sum_{k=N+1}^{\infty} k\left| C_k \right| r^{k-1} < \frac{\varepsilon}{3}\]
        При таком $N$ второй и третий модули $< \frac{\varepsilon}{3}$ по критерию Коши абсолютной сходимости ряда \eqref{series}. А в первом модуле конечная сумма, которая стремится к 0 при $\Delta z \to 0$, то есть по определению предела, первый модуль тоже $< \frac{\varepsilon}{3}$
    \end{proof}
\end{theorem*}


\begin{note}
    Сумма степенного ряда $S'(z) = \sum kC_kz^{k-1}$ тоже аналитична в круге $\left| z \right| < R$, причем 
    \[S''(z) = \sum k(k-1)C_k z^{k-2}\]
    \[S^{(n)} = \sum k(k-1)\hdots(k-n+1)C_k z^{k-n} \overset{\text{при } z = 0}{\implies} C_n = \frac{S^{(n)}(z)}{n!}, \ n = 1, 2, \hdots\]
\end{note}
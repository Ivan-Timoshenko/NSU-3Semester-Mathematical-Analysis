\section{Дифференциирование функций}

    
\begin{definition}
    Функция $f(x)$ дифференциируема в точке $p \in U$, если:
    \begin{enumerate}
        \item $f$  определена в некоторой окрестности точки $p$ ($p \in Int(U)$)
        \item $\exists \lim_{\Delta h \to 0} \frac{f(p + \Delta h) - f(p)}{\Delta h} \in \mathbb{R} $ (и этот предел равен $f'(p)$)
    \end{enumerate}
\end{definition}

\subsection{Экстремумы}
\medskip
\textbf{Необходимое условие экстремума:}\\
Пусть $f \in D(p)$ (дифференциируема в p). Если p - экстремум, то $f'(p) = 0$.

\begin{note}
    НО например для $f(x) = x^{3} \quad f'(0) = 0$, но $f(x)$ не дифференциируема в 0.
\end{note}
\begin{note}
Необходимое условие экстремума выполнено лишь для точек во внутренности области определения,
точки на границе необходимо проверять отдельно.
\end{note}


\textbf{Достаточное условие экстремума:}\\
Пусть $f \in D^{2}(p)$ (дважды дифференциируема в $p$) и $f'(p) = 0$.В таком случае если
\begin{itemize}
    \item $f''(p) < 0$ - точка $p$ является локальным максимумом и экстремумом.
    \item $f''(p) > 0$ - точка $p$ является локальным минимумом и экстремумом.
\end{itemize} 

\medskip
Пусть $f:U \in \mathbb{R}^{n} \to \mathbb{R}^{k}$, $p \in U$. 
Функция $f$ дифференциируема в $p$, если:
\begin{enumerate}
    \item $p \in Int(U) \quad
        (\exists \epsilon > 0 \quad B_\epsilon(p) \subset U)$
    \item $\exists$ дифференциал функции (линейное отображение)
    $f$ в точке $p$ \quad $df(p): \mathbb{R}^{n} \to \mathbb{R}^{k}$
    такое, что    \[f(x) = f(p) + df(p)<x-p> + \alpha(x) \quad (\alpha(x) \underset{x \to p}{=} o(x-p)) \]
    При сдвиге точки $p$ на вектор $h$: \[f(p+h) = f(p) + df(p)<h> + o(|h|)\]
\end{enumerate}

\subsection{Частные производные}
\medskip
Стандартный контекст в котором работаем:
\[f:U \subset \mathbb{R}^{n} \to \mathbb{R}^{k}, \quad p \in U, \quad p = (p_1, p_2, \dots, p_n) \]

\begin{definition}
    Частная производная по координате $x_i$ это:
    \[\frac{\partial f}{\partial x_i}(p) = \lim_{t \to 0} \frac{f(p_1,.., p_i + t,.., p_n) - f(p_1,\ldots,p_n)}{t}\]
\end{definition}

Пример для $f(x, y) = x^y$:
\[\frac{\partial f}{\partial x}(x, y) = yx^{y-1}; \quad \frac{\partial f}{\partial y}(x, y) = x^y\ln(x)\]

\begin{definition}
    Прозводная вдоль вектора $v$:
    \[\frac{\partial f}{\partial v}(p) := \lim_{t \to 0} \frac{f(p + tv) - f(p)}{t}\]
    Если $v = e_i = (0, \ldots, 0, \underset{i}{1}, 0, \ldots, 0)$, то $\frac{\partial f}{\partial v} = \frac{\partial f}{\partial x_i} = f'_{x_i}$
\end{definition}

Пример:
\[f(x) = \begin{cases}
    \frac{x^2y}{x^4 + y^2} \quad $при$(x, y) \neq (0, 0) \\
    0 \quad $при$ (x, y) = (0,0)
\end{cases}\]
По любому вектору $v = (v_1, v_2)$ у функции есть производная в $(0, 0)$:
\[\lim_{t \to 0} \frac{f(tv) - f(0, 0)}{t} = \frac{t^3v_1^2v^2}{t^5v_1^4 + t^3v_2^2} = \frac{v_1^2v_2}{t^2v_1^2 v_2^2} 
    \underset{t \to 0}{=} 
    \begin{cases}
        0 \quad v_2 = 0\\
        \frac{v_1^2}{v_2} \quad v_2 \neq 0
    \end{cases}\]

\begin{statement}
    Если $f$ дифференциируема в $p$, то $f$ - непрерына в $p$.
\end{statement}

\begin{proof}
    \[f(p+h) - f(p) \underset{h \to 0}{=} df(p)<h>\] 
    Линейное отображение df(p)<> непрерывно, $o(h) \underset{h \to 0}{\to} 0$,
    т.е. $f(p+h) \underset{h \to 0}{\to} f(p)$.
\end{proof}

\textbf{Достаточный признак дифференциируемости:}\\
Все частные производные непрерырвны в $p$ ($f \in D(p)$).

Пример:
$f(x, y) = x^y$ дифференциируема во всех точках $(x_0, y_0)$, где $x_0 > 0$.
\[\frac{\partial f}{\partial x} = yx^{y-1} \qquad \frac{\partial f}{\partial y} = x^y\ln(x)\]
Частные производные непрерывны, значит и функция непрерывна.

\subsection{Матрица Якоби и градиент функции}
Контекст в котором работаем:
\[f:U \subset \mathbb{R}^{n} \to \mathbb{R}^{k}, \quad p \in U\]
\begin{definition}
Матрицей Якоби называют матрицу
\[D_f(p) = 
\begin{pmatrix}
    \frac{\partial f_1}{\partial x_1} & \ldots & \frac{\partial f_1}{\partial x_n}\\
    \vdots & \ddots & \vdots \\
    \frac{\partial f_k}{\partial x_1} & \dots & \frac{\partial f_k}{\partial x_n}
\end{pmatrix}\]
\end{definition}

В случае, если функция $f$ отображает $\mathbb{R}^n \to \mathbb{R}$, то матрица Якоби принимает вид $1 \times n$ и называется \textbf{градиентом функции}.
\begin{definition}
    Градиентом функции называется вектор 
    \[D_f = 
    \begin{pmatrix}
        \frac{\partial f}{\partial x_1}, & \dots, & \frac{\partial f}{\partial x_n}
    \end{pmatrix}
    \]
\end{definition}
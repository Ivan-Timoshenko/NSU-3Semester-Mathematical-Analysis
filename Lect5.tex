\section{Многообразия в $\mathbb{R}^n$}

\begin{definition}
    Пусть $M$ - метрическое пространство (необязательно подмножество в $\mathbb{R}^n$). $M$ является $k$-мерным мноогобразием без края,
    если $\forall p \in M$ у точки $p \ \exists $ окрестность $U_p \open M$ гомеоморфная открытому шару в $\mathbb{R}^k$.
\end{definition}

\begin{definition}
    Гомеоморфизм - непрерывное отображение, обратное к которому тоже непрерывно.
\end{definition}

\begin{theorem}[Брауэра об инвариантности области (без док-ва).]
    \par Пусть $U \open \mathbb{R}^k$ и $f:U \to \mathbb{R}^k$ - непрерывна и инъективна. \newline
    Тогда $f(U) \open \mathbb{R}^k$ и $f^{-1}: f(U) \to U$ - тоже непрерывна.
\end{theorem}

\begin{definition}
    $M$ - $k$-мерное $C^r$-мноогобразие в $R^n$, если:
    \[\forall p \in M \ \exists U \in \mathcal{N}(p), \ U \open M \text{ такая, что } U \overset{C^r}{\cong} \text{ открытому шару в } \mathbb{R}^k\]
\end{definition}

\begin{statement*}
    Предыдущее утверждение эквивалентно требованиям:
    \begin{enumerate}
        \item \[ \exists U \in \mathcal{N}(p) \ U \open M \text{ такая, что } U \cong \text{ открытому подмножеству } \omega \in \mathbb{R}^k\]
        \item \[ \exists U \in \mathcal{N}(p) \ U \open M \text{ такая, что } U \cong \mathbb{R}^k\]
    \end{enumerate}
\end{statement*}
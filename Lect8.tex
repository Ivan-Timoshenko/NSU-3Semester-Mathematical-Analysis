\section{Ряды Фурье}

Пусть $E$ - евклидово пространство (векторное пространство со скалярным произведением).\\
Скалярное произведение $\scalar{a}{b}$ - бинарная операция, удовлетворяющая аксиомам:
\begin{enumerate}
    \item $\scalar{a}{b} = \scalar{b}{a} \iff a, b \in \Real; \ \scalar{a}{b} = \overline{\scalar{b}{a}} \iff a, b \in \Compl$
    \item $\scalar{a}{a} \ge 0, \ \scalar{a}{a} = 0 \iff a = 0$
    \item $\scalar{\lambda u + \mu v}{w} = \lambda \scalar{u}{w} + \mu \scalar{v}{w}$
\end{enumerate}
Модуль вектора вычисляется с помощью $\left| v \right| = \sqrt{\scalar{v}{v}}$.

\begin{statement*}
    Пусть $\vec{e_1}, \hdots, \vec{e_n}$ - попарно ортогональные векторы ($\scalar{\vec{e_i}}{\vec{e_j}} = 0$). Пусть $\vec{v} = x_1\vec{e_1} + \hdots + x_n \vec{e_n}$, тогда
    \[x_i = \frac{\scalar{v}{e_i}}{\scalar{e_i}{e_i}} \ \forall i = 1, \hdots, n\]
    \begin{proof}
        \[k \neq i \implies \scalar{e_k}{e_i} = 0 \implies \scalar{v}{e_i} = \scalar{\sum_{k=1}^{n} x_k e_k}{e_i} = \sum_{k=1}^{n}x_k \scalar{e_k}{e_i} = x_i \scalar{e_i}{e_i}\]
    \end{proof}
\end{statement*}

Если $\{\vec{e_i}\}$ - попарно ортогональые ненулевые вектора, то они образуют линейно независимый набор.
\[\left| v \right|^2 = \scalar{v}{v} = \scalar{\sum x_i e_i}{\sum x_j e_j} = \sum x_i x_j \scalar{e_j}{e_i} =  \sum x_i^2 \left|e_i \right|^2\]
В частности, если все длины векторов $\vec{e_i}$ равны $1$, то $\left| v \right|^2 = \sum x_i^2$.

\par
Пусть $f, g$ - хорошие функции на отрезке $\left[0, 2\pi\right]$, то их скалярное произведение определено следующим образом:
\[\scalar{f}{g} := \int_{0}^{2\pi} f(t)g(t) \diff t\]

\begin{statement*}
    Функции $\sin(nx)$ и $\cos(mx)$ ортогональны при $\forall n, m \in \Nat$. А также:
    \begin{itemize}
        \item $\sin(nx) \bot \sin(mx) \ \forall n \neq m \ge 0$
        \item $\cos(nx) \bot \cos(mx) \ \forall n \neq m \ge 0$
        \item $|| \sin(nx) || = \pi \ n > 0$
        \item $|| \cos(nx) || = \pi \ n > 0$
    \end{itemize}
    \begin{proof}
        \[\cos(mx) = \frac{e^{imx}+e^{-imx}}{2}; \ \sin(nx) = \frac{e^{inx}-e^{-inx}}{2i}\]
        \[\int_{0}^{2\pi} e^{inx}\cdot e^{imx} \diff x,\  n, m \in \Nat = \int_{0}^{2\pi} e^{i(n+m)x} \diff x = \begin{cases*}
            \int_{0}^{2\pi} 1 \diff x = 2\pi \ n+m = 0\\
            \frac{e^{i(n+m)x}}{i(n+m)} \big|_{0}^{2\pi} = 0 \ n+m \neq 0
        \end{cases*}\]

        \[\int_{0}^{2\pi} \cos(nx)\sin(mx)\diff x = \int_{0}^{2\pi} \frac{e^{i(n+m)x} - e^{i(n-m)x} + e^{i(m-n)x} - e^{-i(n+m)x}}{2\cdot 2i} \diff x = 0 \ \forall n,m \in \Nat\]
        Аналогично остальные 
        \[\int_{0}^{2\pi} \sin(nx)^2 \diff x = \int_{0}^{2\pi} \frac{e^{2inx} - e^{-2inx}-2}{-4} \diff x = \int_{0}^{2\pi} \frac{-2}{-4} \diff x = \pi\]
    \end{proof}
\end{statement*}

\begin{statement*}[Следствие]
    Пусть $f(t)$ - линейная комбинация вида
    \[f(t) = \lambda + b_1\sin(t) + \hdots + b_n\sin(nt) + a_1\sin(t) + \hdots + a_n\cos(nt)\]
    Тогда 
    \[b_k = \frac{\int_{0}^{2\pi}f(t)\sin(kt)\diff t}{\pi}, \quad a_k = \frac{\int_{0}^{2\pi}f(t)\cos(kt)\diff t}{\pi}, \quad \lambda = \frac{\int_{0}^{2\pi} f(t)\cdot 1 \diff t}{2\pi}\]

\end{statement*}

\begin{definition}
    Ряд Фурье функции $f(t)$ (разложение обозначается знаком $\leftrightharpoons$)
    \[f(t) \leftrightharpoons \frac{a_0}{2} + \sum_{k=1}^{\infty} \left[a_k\sin(kt) + b_k\cos(kt)\right]\]
    \[a_k = \frac{1}{\pi} \int_{0}^{2\pi} f(t)\cos(kt)\diff t \quad b_k = \frac{1}{\pi}\int_{0}^{2\pi}f(t)\sin(kt) \diff t, \ k \ge 0\]
\end{definition}

Введем контекст:\par
Пусть $f$ - функция, интегрируемая на отрезке $[0, 2\pi]$. Разложим ее в ряд Фурье в соответствии с определением:
\[f(x) \leftrightharpoons \frac{a_0}{2} + \sum_{n=1}^{\infty} \left[ a_n\cos(nx) + b_n\sin(nx) \right]\]
\[a_n = \frac{1}{\pi} \int_{0}^{2\pi}f(t)\cos(nt)\diff t \quad b_n = \frac{1}{\pi} \int_{0}^{2\pi} f(t)\sin(nt)\diff t\]

\begin{theorem*}[Лемма об интеграле периодичной функции]
    Пусть $f$ - $T$-периодична, если $f$ интегрируема на $[0, T]$, то $f$ интегрируема на любом отрезке $[a, b]$, причем 
    \[\int_{0}^{T}f(x)\diff x = \int_{a}^{T+a}f(x) \diff x  \]
    \begin{proof}
        Функция $f$ интегрируема на любом отрезке $[T, (n+1)T]$, отсюда следует интегрируемость на любом $[a, b]$. Пусть $a \in [0, T]$, тогда 
        \[\int_{0}^{T+a} f(x)\diff x = \int_{0}^{T}f(x)\diff x + \int_{T}^{T+a}f(x)\diff x = \int_{0}^{a}f(x)\diff x + \int_{a}^{T+a}f(x)\diff x\]
        Разложим интеграл от $0$ до $T+a$ двумя способами, в силу периодичности $\int_{T}^{T+a} = \int_{0}^{a}$, значит другие два интеграла тоже равны.
        В общем случае $a>T$, тогда $\exists n \mid a \in [nT, (n+1)T]$ и $\int_{0}^{T} = \int_{nT}^{(n+1)T} = \int_{a}^{a+T}$.
    \end{proof}

    \textbf{Следствие}:\par
    Если $f$ - $2\pi$-периодична, то пределы интегрирования для коэффициентов $a_n, b_n$ будем брать $[-\pi, \pi]$.
\end{theorem*}

\subsection{Комплексная форма ряда Фурье}
\[f(x) \leftrightharpoons \sum_{k= -\infty}^{\infty} c_k e^{ikx}, \ c_k = \frac{1}{2\pi}\int_{-\pi}^{\pi} f(t)e^{-ikt}\diff t\]
Такая форма получается из $2\pi$-перодичной функции $f$, при $k \ge 0$, тогда
\[a_k + ib_k = \frac{1}{\pi} \int_{-\pi}^{\pi} f(t)(\cos(kt) + i\sin(kt)) \diff t = \frac{1}{\pi} \int_{-\pi}^{\pi}f(t)e^{ikt} \diff t = 2c_k\]

\textbf{Теорема Пифагора:}\par
Пусть $f = \sum x_n\vec{e_n}$. $e_n \bot e_n (n \neq k)$, $f$ - хорошая, тогда $\langle f, f, \rangle = \sum \left| x \right|^2 \langle e_n, e_n \rangle$. При этом $\langle e_n, e_n \rangle = \int_{-\pi}^{\pi} \sin^2(nx) \diff x = \pi$. 
\[\langle e_0, e_0 \rangle = \int_{-\pi}^{\pi} 1^2 \diff x = 2\pi\]
Если $f$ - хорошая и $f = \leftrightharpoons \frac{a_0}{2} + \sum_{n=1}^{\infty} (a_n\cos(nx) + b_n\sin(nx))$, то 
\[\frac{a_0^2}{2} + \sum_{n=1}^{\infty} \left| a_n \right|^2 + \left| b_n \right|^2 = \frac{1}{\pi} \int_{-\pi}^{\pi} \left| f(x) \right|^2 \diff x\]
Это равенство называется равенством Парсеваля.


\begin{definition}
    Формула Дирихле.
    \[D_n(t) \leftrightharpoons \sum_{k=-n}^{n}e^{ikt} = 1 + 2(\cos(t) + \cos(2t) + \hdots + \cos(nt)) = \frac{\sin((n+\frac{1}{2})t)}{\sin(\frac{t}{2})}\]
\end{definition}

\begin{definition}
    Если функция $f$ $2\pi$-периодична, кусочно дифференциируемая и ее первая производная ограничена, то $f$ - функция класса Фурье.
\end{definition}

\begin{statement*}
    Функция класса Фурье может иметь разрывы только 1-го рода, то есть существует конечные пределы:
    \[\forall x \ \exists f(x+0) := \lim_{n \to 0} f(x+h), \ \exists f(x-0):= \lim_{h \to 0} f(x-h)\]

    \begin{proof}
        Пусть $x_0$ - точка разрыва функции $f$. Выберем $a < x_0$ так, что на $[a, x_0)$ $f$ - дифференциируема. Тогда $\forall x \in [a, x_0)$:
        \[f(x) = f(a) + \int_{a}^{x}f'(t) \diff t\]
        Этот интеграл сходится, значит утверждение доказано.
    \end{proof}
\end{statement*}

\begin{statement*}
    Обозначим разложение функции $f$ в ряд Фурье через $S_n f$. Тогда
    \[S_n f(x) = \frac{1}{2\pi} \int_{-\pi}^{\pi} f(x+t) D_n(t) \diff t\]
    Функция Дирихле четна, $2\pi$-периодична.

    \begin{proof}
        \begin{align*}
            \int_{-\pi}^{\pi} f(x+t) D_n(t)\,dt 
            &= \int_{-\pi}^{\pi} f(x+t) D_n(-t)\,dt 
            \qquad (\text{т.к. } D_n(t) = D_n(-t)) \\[4pt]
            &= \int_{-\pi}^{\pi} f(x+t) \sum_{k=-n}^{n} e^{-ikt}\,dt
            = \sum_{k=-n}^{n} \int_{-\pi}^{\pi} f(x+t) e^{-ikt}\,dt \\[4pt]
            &= \sum_{k=-n}^{n} \int_{x-\pi}^{x+\pi} f(\tau) e^{-ik(\tau - x)}\,d\tau
            \qquad (\tau = x + t) \\[4pt]
            &= \sum_{k=-n}^{n} \int_{-\pi}^{\pi} f(\tau) e^{-ik\tau} e^{ikx}\,d\tau
            \qquad \text{(периодичность)} \\[4pt]
            &= \sum_{k=-n}^{n} e^{ikx} \int_{-\pi}^{\pi} f(\tau) e^{-ik\tau}\,d\tau
            = \sum_{k=-n}^{n} e^{ikx} \cdot 2\pi c_k.
        \end{align*}
    \end{proof}

    \textbf{Следствие:}\par
    Если функция класса Фурье, то
    \begin{align*}
    & 2\pi \left(S_n f(x) - \frac{f(x-0) + f(x+0)}{2}\right) = \\
    &= \int_{-\pi}^{0} \left( f(x+t) - f(x-0) D_n(t) \right) \diff t + \int_{0}^{\pi} \left(f(x+t) - f(x+0)D_n(t)\right) \diff t
    \end{align*}

    \begin{proof}
        
        \begin{align*}
            & 2\pi \left[ S_n f(x) - \frac{f(x-0)+f(x+0)}{2}\right] = \\
            &= \int_{-\pi}^{\pi} f(x+t)D_n(t) \diff t - \int_{-\pi}^{\pi} \frac{f(x-0) + f(x+0)}{2}\cdot D_n(t) \diff t = \\
            &= \int_{-\pi}^{\pi} \left[ f(x+t) - \frac{f(x-0)+f(x+0)}{2} \right] D_n(t) \diff t = \\
            &= \int_{-\pi}^{0} \left[ f(x+t) - f(x-0) \right]D_n(t) \diff t + \int_{0}^{\pi} \left[ f(x+t) - f(x+0) \right]D_n(t) \diff t
        \end{align*}
    \end{proof}
\end{statement*}


\begin{theorem*}[Теорема Фурье]
    Пусть $f$ - функция класса Фурье. Тогда
    \begin{enumerate}
        \item $\forall x \ S_n f(x) \underset{n \to \infty}{\to} \frac{f(x-0) - f(x+0)}{2}$.
        \item Если $a_0, \hdots, a_n$ - точки разрыва из отрезка $[-\pi, \pi]$, то для любого отрезка $I$, не содержащего ни одной точки разрыва $a_i$, сходимость на $I$ будет равномерной.
        
    \end{enumerate}

    \begin{proof}
        БУДЕТ ПОЗЖЕ
    \end{proof}
\end{theorem*}
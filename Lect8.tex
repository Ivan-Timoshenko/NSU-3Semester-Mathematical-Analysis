\section{Ряды Фурье}

Пусть $E$ - евклидово пространство (векторное пространство со скалярным произведением).\\
Скалярное произведение $\scalar{a}{b}$ - бинарная операция, удовлетворяющая аксиомам:
\begin{enumerate}
    \item $\scalar{a}{b} = \scalar{b}{a} \iff a, b \in \Real; \ \scalar{a}{b} = \overline{\scalar{b}{a}} \iff a, b \in \Compl$
    \item $\scalar{a}{a} \ge 0, \ \scalar{a}{a} = 0 \iff a = 0$
    \item $\scalar{\lambda u + \mu v}{w} = \lambda \scalar{u}{w} + \mu \scalar{v}{w}$
\end{enumerate}
Модуль вектора вычисляется с помощью $\left| v \right| = \sqrt{\scalar{v}{v}}$.

\begin{statement*}
    Пусть $\vec{e_1}, \hdots, \vec{e_n}$ - попарно ортогональные векторы ($\scalar{\vec{e_i}}{\vec{e_j}} = 0$). Пусть $\vec{v} = x_1\vec{e_1} + \hdots + x_n \vec{e_n}$, тогда
    \[x_i = \frac{\scalar{v}{e_i}}{\scalar{e_i}{e_i}} \ \forall i = 1, \hdots, n\]
    \begin{proof}
        \[k \neq i \implies \scalar{e_k}{e_i} = 0 \implies \scalar{v}{e_i} = \scalar{\sum_{k=1}^{n} x_k e_k}{e_i} = \sum_{k=1}^{n}x_k \scalar{e_k}{e_i} = x_i \scalar{e_i}{e_i}\]
    \end{proof}
\end{statement*}

Если $\{\vec{e_i}\}$ - попарно ортогональые ненулевые вектора, то они образуют линейно независимый набор.
\[\left| v \right|^2 = \scalar{v}{v} = \scalar{\sum x_i e_i}{\sum x_j e_j} = \sum x_i x_j \scalar{e_j}{e_i} =  \sum x_i^2 \left|e_i \right|^2\]
В частности, если все длины векторов $\vec{e_i}$ равны $1$, то $\left| v \right|^2 = \sum x_i^2$.

\par
Пусть $f, g$ - хорошие функции на отрезке $\left[0, 2\pi\right]$, то их скалярное произведение определено следующим образом:
\[\scalar{f}{g} := \int_{0}^{2\pi} f(t)g(t) \diff t\]

\begin{statement*}
    Функции $\sin(nx)$ и $\cos(mx)$ ортогональны при $\forall n, m \in \Nat$. А также:
    \begin{itemize}
        \item $\sin(nx) \bot \sin(mx) \ \forall n \neq m \ge 0$
        \item $\cos(nx) \bot \cos(mx) \ \forall n \neq m \ge 0$
        \item $|| \sin(nx) || = \pi \ n > 0$
        \item $|| \cos(nx) || = \pi \ n > 0$
    \end{itemize}
    \begin{proof}
        \[\cos(mx) = \frac{e^{imx}+e^{-imx}}{2}; \ \sin(nx) = \frac{e^{inx}-e^{-inx}}{2i}\]
        \[\int_{0}^{2\pi} e^{inx}\cdot e^{imx} \diff x,\  n, m \in \Nat = \int_{0}^{2\pi} e^{i(n+m)x} \diff x = \begin{cases*}
            \int_{0}^{2\pi} 1 \diff x = 2\pi \ n+m = 0\\
            \frac{e^{i(n+m)x}}{i(n+m)} \big|_{0}^{2\pi} = 0 \ n+m \neq 0
        \end{cases*}\]

        \[\int_{0}^{2\pi} \cos(nx)\sin(mx)\diff x = \int_{0}^{2\pi} \frac{e^{i(n+m)x} - e^{i(n-m)x} + e^{i(m-n)x} - e^{-i(n+m)x}}{2\cdot 2i} \diff x = 0 \ \forall n,m \in \Nat\]
        Аналогично остальные 
        \[\int_{0}^{2\pi} \sin(nx)^2 \diff x = \int_{0}^{2\pi} \frac{e^{2inx} - e^{-2inx}-2}{-4} \diff x = \int_{0}^{2\pi} \frac{-2}{-4} \diff x = \pi\]
    \end{proof}
\end{statement*}

\begin{statement*}[Следствие]
    Пусть $f(t)$ - линейная комбинация вида
    \[f(t) = \lambda + b_1\sin(t) + \hdots + b_n\sin(nt) + a_1\sin(t) + \hdots + a_n\cos(nt)\]
    Тогда 
    \[b_k = \frac{\int_{0}^{2\pi}f(t)\sin(kt)\diff t}{\pi}, \quad a_k = \frac{\int_{0}^{2\pi}f(t)\cos(kt)\diff t}{\pi}, \quad \lambda = \frac{\int_{0}^{2\pi} f(t)\cdot 1 \diff t}{2\pi}\]

\end{statement*}

\begin{definition}
    Ряд Фурье функции $f(t)$ (разложение обозначается знаком $\leftrightharpoons$)
    \[f(t) \leftrightharpoons \frac{a_0}{2} + \sum_{k=1}^{\infty} \left[a_k\sin(kt) + b_k\cos(kt)\right]\]
    \[a_k = \frac{1}{\pi} \int_{0}^{2\pi} f(t)\cos(kt)\diff t \quad b_k = \frac{1}{\pi}\int_{0}^{2\pi}f(t)\sin(kt) \diff t, \ k \ge 0\]
\end{definition}

Введем контекст:\par
Пусть $f$ - функция, интегрируемая на отрезке $[0, 2\pi]$. Разложим ее в ряд Фурье в соответствии с определением:
\[f(x) \leftrightharpoons \frac{a_0}{2} + \sum_{n=1}^{\infty} \left[ a_n\cos(nx) + b_n\sin(nx) \right]\]
\[a_n = \frac{1}{\pi} \int_{0}^{2\pi}f(t)\cos(nt)\diff t \quad b_n = \frac{1}{\pi} \int_{0}^{2\pi} f(t)\sin(nt)\diff t\]

\begin{theorem*}[Лемма об интеграле периодичной функции]
    Пусть $f$ - $T$-периодична, если $f$ интегрируема на $[0, T]$, то $f$ интегрируема на любом отрезке $[a, b]$, причем 
    \[\int_{0}^{T}f(x)\diff x = \int_{a}^{T+a}f(x) \diff x  \]
    \begin{proof}
        Функция $f$ интегрируема на любом отрезке $[T, (n+1)T]$, отсюда следует интегрируемость на любом $[a, b]$. Пусть $a \in [0, T]$, тогда 
        \[\int_{0}^{T+a} f(x)\diff x = \int_{0}^{T}f(x)\diff x + \int_{T}^{T+a}f(x)\diff x = \int_{0}^{a}f(x)\diff x + \int_{a}^{T+a}f(x)\diff x\]
        Разложим интеграл от $0$ до $T+a$ двумя способами, в силу периодичности $\int_{T}^{T+a} = \int_{0}^{a}$, значит другие два интеграла тоже равны.
        В общем случае $a>T$, тогда $\exists n \mid a \in [nT, (n+1)T]$ и $\int_{0}^{T} = \int_{nT}^{(n+1)T} = \int_{a}^{a+T}$.
    \end{proof}

    \textbf{Следствие}:\par
    Если $f$ - $2\pi$-периодична, то пределы интегрирования для коэффициентов $a_n, b_n$ будем брать $[-\pi, \pi]$.
\end{theorem*}

\subsection{Комплексная форма ряда Фурье}
\[f(x) \leftrightharpoons \sum_{k= -\infty}^{\infty} c_k e^{ikx}, \ c_k = \frac{1}{2\pi}\int_{-\pi}^{\pi} f(t)e^{-ikt}\diff t\]
Такая форма получается из $2\pi$-перодичной функции $f$, при $k \ge 0$, тогда
\[a_k + ib_k = \frac{1}{\pi} \int_{-\pi}^{\pi} f(t)(\cos(kt) + i\sin(kt)) \diff t = \frac{1}{\pi} \int_{-\pi}^{\pi}f(t)e^{ikt} \diff t = 2c_k\]

\textbf{Теорема Пифагора:}\par
Пусть $f = \sum x_n\vec{e_n}$. $e_n \bot e_n (n \neq k)$, $f$ - хорошая, тогда $\langle f, f, \rangle = \sum \left| x \right|^2 \langle e_n, e_n \rangle$. При этом $\langle e_n, e_n \rangle = \int_{-\pi}^{\pi} \sin^2(nx) \diff x = \pi$. 
\[\langle e_0, e_0 \rangle = \int_{-\pi}^{\pi} 1^2 \diff x = 2\pi\]
Если $f$ - хорошая и $f = \leftrightharpoons \frac{a_0}{2} + \sum_{n=1}^{\infty} (a_n\cos(nx) + b_n\sin(nx))$, то 
\[\frac{a_0^2}{2} + \sum_{n=1}^{\infty} \left| a_n \right|^2 + \left| b_n \right|^2 = \frac{1}{\pi} \int_{-\pi}^{\pi} \left| f(x) \right|^2 \diff x\]
Это равенство называется равенством Парсеваля.


\begin{definition}
    Формула Дирихле.
    \[D_n(t) \leftrightharpoons \sum_{k=-n}^{n}e^{ikt} = 1 + 2(\cos(t) + \cos(2t) + \hdots + \cos(nt)) = \frac{\sin((n+\frac{1}{2})t)}{\sin(\frac{t}{2})}\]
\end{definition}

\begin{definition}
    Если функция $f$ $2\pi$-периодична, кусочно дифференциируемая и ее первая производная ограничена, то $f$ - функция класса Фурье.
\end{definition}

\begin{statement*}
    Функция класса Фурье может иметь разрывы только 1-го рода, то есть существует конечные пределы:
    \[\forall x \ \exists f(x+0) := \lim_{n \to 0} f(x+h), \ \exists f(x-0):= \lim_{h \to 0} f(x-h)\]

    \begin{proof}
        Пусть $x_0$ - точка разрыва функции $f$. Выберем $a < x_0$ так, что на $[a, x_0)$ $f$ - дифференциируема. Тогда $\forall x \in [a, x_0)$:
        \[f(x) = f(a) + \int_{a}^{x}f'(t) \diff t\]
        Этот интеграл сходится, значит утверждение доказано.
    \end{proof}
\end{statement*}

\begin{statement*}
    Обозначим разложение функции $f$ в ряд Фурье через $S_n f$. Тогда
    \[S_n f(x) = \frac{1}{2\pi} \int_{-\pi}^{\pi} f(x+t) D_n(t) \diff t\]
    Функция Дирихле четна, $2\pi$-периодична.

    \begin{proof}
        \begin{align*}
            \int_{-\pi}^{\pi} f(x+t) D_n(t)\,dt 
            &= \int_{-\pi}^{\pi} f(x+t) D_n(-t)\,dt 
            \qquad (\text{т.к. } D_n(t) = D_n(-t)) \\[4pt]
            &= \int_{-\pi}^{\pi} f(x+t) \sum_{k=-n}^{n} e^{-ikt}\,dt
            = \sum_{k=-n}^{n} \int_{-\pi}^{\pi} f(x+t) e^{-ikt}\,dt \\[4pt]
            &= \sum_{k=-n}^{n} \int_{x-\pi}^{x+\pi} f(\tau) e^{-ik(\tau - x)}\,d\tau
            \qquad (\tau = x + t) \\[4pt]
            &= \sum_{k=-n}^{n} \int_{-\pi}^{\pi} f(\tau) e^{-ik\tau} e^{ikx}\,d\tau
            \qquad \text{(периодичность)} \\[4pt]
            &= \sum_{k=-n}^{n} e^{ikx} \int_{-\pi}^{\pi} f(\tau) e^{-ik\tau}\,d\tau
            = \sum_{k=-n}^{n} e^{ikx} \cdot 2\pi c_k.
        \end{align*}
    \end{proof}

    \textbf{Следствие:}\par
    Если функция класса Фурье, то
    \begin{align*}
    & 2\pi \left(S_n f(x) - \frac{f(x-0) + f(x+0)}{2}\right) = \\
    &= \int_{-\pi}^{0} \left( f(x+t) - f(x-0) D_n(t) \right) \diff t + \int_{0}^{\pi} \left(f(x+t) - f(x+0)D_n(t)\right) \diff t
    \end{align*}

    \begin{proof}
        
        \begin{align*}
            & 2\pi \left[ S_n f(x) - \frac{f(x-0)+f(x+0)}{2}\right] = \\
            &= \int_{-\pi}^{\pi} f(x+t)D_n(t) \diff t - \int_{-\pi}^{\pi} \frac{f(x-0) + f(x+0)}{2}\cdot D_n(t) \diff t = \\
            &= \int_{-\pi}^{\pi} \left[ f(x+t) - \frac{f(x-0)+f(x+0)}{2} \right] D_n(t) \diff t = \\
            &= \int_{-\pi}^{0} \left[ f(x+t) - f(x-0) \right]D_n(t) \diff t + \int_{0}^{\pi} \left[ f(x+t) - f(x+0) \right]D_n(t) \diff t
        \end{align*}
    \end{proof}
\end{statement*}


\begin{theorem*}[Теорема Фурье]
    Пусть $f$ - функция класса Фурье. Тогда
    \begin{enumerate}
        \item $\forall x \ S_n f(x) \underset{n \to \infty}{\to} \frac{f(x-0) - f(x+0)}{2}$.
        \item Если $a_0, \hdots, a_n$ - точки разрыва из отрезка $[-\pi, \pi]$, то для любого отрезка $I$, не содержащего ни одной точки разрыва $a_i$, сходимость на $I$ будет равномерной.
        
    \end{enumerate}

    \begin{proof}
        Надо доказать, что 
        \[\int_{0}^{\pi} \left[f(x+t) - f(x+0)\right]D_n(t) \diff t \underset{n \to \infty}{\to} 0\]
        Пусть $\varepsilon > 0$, выберем такое $h$, что $\int_{0}^{h} < \frac{\varepsilon}{2}$.

        \[
            \abs{ (f(x+t) - f(x+0))D_n(t)} = \abs{(f(x+t) - f(x+0)\frac{\sin(t(n+\frac{1}{2}))}{\sin(\frac{t}{2})})}
        \]
        Поскольку при $t \in [0, \pi]$ $\sin(\frac{t}{2}) \ge \frac{t}{\pi}$, то
        \[\abs{(f(x+t) - f(x+0)\frac{\sin(t(n+\frac{1}{2}))}{\sin(\frac{t}{2})})} \le \underbrace{\frac{\abs{f(x+t) - f(x+0)}\cdot \abs{\sin(t(n + \frac{1}{2}))}}{\abs{\frac{t}{\pi}}}}_{\star} \le \frac{\pi\abs{f(x+t) - f(x+0)}}{t}\]
        Если выбрать $h$ такое, чтобы на $(x, x+h)$ у функции $f(x)$ не было разрывов, то $\forall t \in (x, x+h)$ выполнена оценка 
        \[\abs{f(x+t) - f(x+0)} \le t\cdot \underbrace{\sup(\abs{f'(x)})}_{\le C \text{ по условию}}\]
        Продолжая цепочку неравенств $\star$ получим $\le  \pi \cdot \frac{C}{t} = \pi C$. Итак, если $h$ такое, что на $(x, x+h)$ нет точек разрыва, то 
        \begin{align*}
            \abs{\int_{0}^{h} \left[f(x+t) - f(x+0)\right] D_n(t) \diff t} \le \int_{0}^{h} \abs{\left[f(x+t) - f(x+0)\right]D_n(t)} \diff t \le \\
            \le \int_{0}^{h} \pi\cdot C \diff t = \pi \cdot Ch
        \end{align*}
        А если взять $h < \frac{\varepsilon}{2\pi c}$, то $\abs{ \int_{0}^{h}} \le \frac{\varepsilon}{2}$.


        \begin{statement*}[Лемма]
            Пусть $h \in (0, \pi)$. Тогда $\exists K < \infty$ такая, что 
            \[\abs{\int_{h}^{\pi}\left[f(x+t) - f(x+0)\right]D_n(t)\diff t} \le \frac{K}{h}\]
            \begin{proof}[Доказательство леммы]
                \[\left[f(x+t) - f(x+0)\right]D_n(t) = \underbrace{\frac{f(x+t)-f(x+0)}{\sin(\frac{t}{2})}}_{g(t)}\cdot \sin(t(n+\frac{1}{2}))\]
                Пусть $(a, b)$ - один из промежутков в $(h, \pi)$, на котором $f$ дифференциируема. Пусть всего таких промежутков $m$ штук.
                \begin{equation}
                    \label{lemma_fourier_eq}
                \abs{g'(t)} = \abs{\frac{f'(x+t)\cdot \sin(\frac{t}{2}) - \left[ f(x+t) - f(x+0) \right]\cdot \frac{1}{2}\cos(t)}{\sin^2(\frac{t}{2})}}
                \end{equation}
                Заметим, что $\sin^2(\frac{t}{2}) \ge \left(\frac{t}{\pi}\right)^2 \ge \left(\frac{h}{\pi}\right)^2$. Тогда продолжим \eqref{lemma_fourier_eq}:
                \[\eqref{lemma_fourier_eq} \le \frac{\abs{\text{числитель}}\pi^2}{h^2} \le \frac{C+M}{h^2} =: \xi\]
                где $M = \underset{x \in \left[-\pi, \pi\right]}{\sup}(\abs{f(x)}), \ C = \underset{x \in \left[-\pi, \pi\right]}{\sup}(\abs{f'(x)})$.
                \begin{align*}
                    I:= \abs{\int_{a}^{b} \left[f(x+t)-f(x+0)\right]D_n(t)\diff t} = \abs{\int_{a}^{b}g(t)\sin(t(n+\frac{1}{2}))\diff t} \overset{\text{ФИЧ}}{=} \underbrace{\abs{g(t)\cdot\frac{\cos(t(n+\frac{1}{2}))}{n + \frac{1}{2}}}^b_a}_{I_1} - \\
                    - \underbrace{\int_{a}^{b}g'(t)\frac{\cos(t(n+\frac{1}{2}))}{n+\frac{1}{2}}t \diff t}_{I_2} \le \frac{2\max(g(t))}{n+\frac{1}{2}} + \underbrace{\abs{b - a}}_{\le \pi}\cdot \xi \cdot 1 \cdot \frac{1}{n+\frac{1}{2}} = \frac{1}{n+\frac{1}{2}}\left[K\right], \\ K = 2\max(g(t)) + \pi \cdot \xi
                \end{align*}
                Итого $\abs{g(t)} \le M_1$, $\abs{g'(t)} \le M_2$. Тогда
                \[\abs{I_1} \le \frac{2M_1}{n + \frac{1}{2}}, \quad \abs{I_2} \le \frac{\abs{b-a}M_2}{n+\frac{1}{2}} \le \frac{\pi \cdot M_2}{n+\frac{1}{2}}\]
                \[\abs{I} = \abs{I_1 - I_2} \le \abs{I_1} + \abs{I_2} \le \frac{1}{n+\frac{1}{2}}\cdot const, \ const:= 2M_1 + \pi M_2 \cdot \text{[кол-во промежутков (a,b)]}\]
            \end{proof}
        \end{statement*}
        Продолжаем доказывать теорему Фурье:\\
        Получили, что $\exists n_0 \mid \forall n \ge n_0 \abs{\int_{h}^{\pi}\hdots} \le \frac{\varepsilon}{2}$. Таким образом:
        \begin{enumerate}
            \item Выбрали $h \in [0, \pi]$ такое, что у функции $f(x+t) - f(x+0)$ нет разрывов на $t \in (0, h)$ (то есть у $f$ нет разрывов на $(x, x+h)$) и при этом $h$ такое хитрое, что $\abs{\int_{0}^{h}\hdots} \le \frac{\varepsilon}{2}$. Это мы уже сделали.
            \item Вторая часть доказательства:
            \par Пусть теперь на $(a,b) \subset \left[-\pi, pi\right]$ нет точек разрыва функции $f: \Real \to \Real$. Покажем, что $S_n(f) \rightrightarrows f$ на этом $(a, b)$.\\
            $\exists h_0 > 0$ такое, что на $[a-h_0, b+h_0]$ функция $f$ непрерывна. Тогда в части 1 доказательства того, что уже было, можно выбрать $h = h(\varepsilon)$, а не $h = h(x, \varepsilon)$ (выбрать $h$ независимо от $x$). Таким образом выбор $h_0$ можно сделать независимым от выбора $x_0$.

        \end{enumerate}

    \end{proof}
\end{theorem*}

\begin{statement*}[Равенство Парсеваля]
    Пусть $f,g$ - непрерывные функции класса Фурье, $f,g: \Real \to \Compl$. Тогда
    \[\int_{-\pi}^{\pi}f(x)\overline{g(x)}\diff x = 2\pi \sum_{k=-\infty}^{\infty}C_k(f)\overline{C_k(g)}\]
    где 
    \[C_k(f) = \frac{1}{2\pi}\int_{-\pi}^{\pi}f(x)e^{-ikx}\diff x\]
    Сумма от минус бесконечности до бесконечности подразумевается в следующем смысле:
    \[f(x)\leftrightharpoons \sum_{-\infty}^{\infty}C_k(f)e^{ikx} = \lim_{n \to \infty} S_n(f), \quad S_n(f) = \sum_{k=-n}^{n}C_k(f)e^{ikx}\]

    \begin{proof}\par
        \begin{itemize}
            \item $S_n(f) \rightrightarrows f$ на $[-\pi, \pi]$
            \item $S_n(g) \rightrightarrows g$ на $[-\pi, \pi]$
        \end{itemize}
        $S_n(f) \cdot \overline{S_n(g)} \rightrightarrows f \cdot \overline{g}$, так как $f$ и $g$ ограничены (было оставлено как упражнение). 

        \begin{align*}
            \int_{-\pi}^{\pi} f \cdot \overline{g} &= \lim_{n \to \infty} \int_{-\pi}^{\pi} S_n(f(x)) \cdot \overline{S_n(g(x))} \, dx \\
            &= \lim_{n \to \infty} \int_{-\pi}^{\pi} \left( \sum_{k=-n}^{n} C_k(f) e^{ikx} \right) \left( \sum_{m=-n}^{n} \overline{C_m(g)} e^{-imx} \right) dx \\
            &= \lim_{n \to \infty} \sum_{k=-n}^{n} \sum_{m=-n}^{n} C_k(f) \overline{C_m(g)} \int_{-\pi}^{\pi} e^{i(k-m)x} \, dx \\
            &= \lim_{n \to \infty} \sum_{k=-n}^{n} C_k(f) \overline{C_k(g)} \int_{-\pi}^{\pi} e^{0} \, dx \\
            &= \sum_{k=-\infty}^{\infty} 2\pi C_k(f) \overline{C_k(g)}
\end{align*}
    \end{proof}

    Если $f=g$, то 
    \[\int f\cdot \overline{f} = \int_{-\pi}^{\pi} \abs{f(x)}^2 \diff x = 2\pi \sum_{n=-\infty}^{\infty}C_k \cdot \overline{C_k} = 2\pi\sum_{n=-\infty}^{\infty} \abs{C_k}^2\]
\end{statement*}

\begin{statement*}
    Равенство Парсеваля верно для всех $f,g\in L_2[-\pi, \pi]$, где $L_2[-\pi, \pi]$ - это множество всех функций, таких, что $\int_{-\pi}^{\pi}\abs{f(x)}^2 \diff x < \infty$.
\end{statement*}

\begin{theorem*}[Лемма о натуральной параметризации]
    Пусть $\Gamma$ - гладкая кривая в $\Real^n$ длины $L$. Тогда существует натуральная параметризация (т.е. модуль скорости) вида 
    \[\gamma(s):[0, L] \to \Gamma \mid \abs{\dot{\gamma}(s)} = 1\]

    \begin{proof}
        Возьмем какое-то отображение (путь) $f:[a, b] \to \Gamma$, $f \in C^1, f'(t) \neq 0 \ \forall t \in [a,b]$. Длина $L$ будет выражаться как $\int_{a}^{b}\abs{f'(t)} \diff t$. 
        Пройденное расстояние $S:[a,b] \to [0, L]$ представлено в виде $S(t) = \int_{a}^{t}\abs{\dot{f(s)}}\diff s$.
        \[\dot{S}(t) = \left( \int_{a}^{t} \abs{\dot{f}(s)}\diff s \right)' = \abs{\dot{f}(t)} > 0 \ \forall t\]
        При этом $S$ возрастает монотонно, а значит обратима, пусть $t(S)$ - обратная функция.
        \[t(S):[0, L] \to [a. b], \ t'(S) = \frac{1}{s'(t)}\]
        Значит $\gamma(s):=f(t(s))$ - искомая параметризация. 
        \[\dot{\gamma}(s) = \dot{f}(t(s))\cdot t'(s) = \dot{f}(t(s))\cdot \frac{1}{s'(t)} = \frac{\dot{f}(t(s))}{\abs{\dot{f}(t(s))}} \text{ и } \abs{\dot{\gamma}(s)} = 1\]
    \end{proof}
\end{theorem*}

\begin{theorem*}[Площадь сектора кривой]
    Пусть кривая $f(t) = \begin{cases*}
        x(t) \\ y(t) 
    \end{cases*}$, а $f$ и $\dot{f}$ имеют ориентацию. 
    Точка на кривой $f = (x, y)$, вектор скорости в точке $\dot{f} = (\dot{x}, \dot{y})$. Заметим, что 
    \[\forall t \quad  \left| \begin{array}{cc} x & y \\ \dot{x} & \dot{y}
        
    \end{array} \right| > 0\]
    То есть, если эта матрица положительно определена, то вектора $f$ и $\dot{f}$ положительно определены и точка $t_0$ всегда лежит справа от точки $t_1$ ($t_0$ и $t_1$ - точки начала и конца сегмента кривой, ограничивающего наш сектор. Мы показали, что при движении по кривой мы перемещаемся против часовой стрелки).
    \par    
    Площадь сектора:
    \[S = \frac{1}{2}\int_{t_0}^{t_1} \left| \begin{array}{cc} x & y \\ \dot{x} & \dot{y} \end{array} \right| \diff t = \frac{1}{2}\int_{t_0}^{t_1}(x\dot{y} - \dot{x}y)\diff t\]
    Пусть $\diff s$ - площадь треугольника, заметаемого за время $(t, t+\diff t)$, из формулы площади параллелограма через определитель:
    \[\diff s = \frac{1}{2}\left| \begin{array}{cc} x & y \\ \dot{x} & \dot{y} \end{array} \right| \diff t\]
    Тогда 
    \[S = \frac{1}{2}\int_{t_0}^{t_1}\diff s = \frac{1}{2}\int_{t_0}^{t_1}\left| \begin{array}{cc} x & y \\ \dot{x} & \dot{y} \end{array} \right| \diff t\]
\end{theorem*}

\begin{theorem*}[Изопериметрическое неравенство]
    Пусть $\Gamma $ - кусочно дважды гладкая замкнутая кривая длины $2\pi$. Тогда $S \le \pi$, причем если $S = \pi$, то $\Gamma$ - это окружность.
    \begin{proof}
        Пусть $\gamma = (x(s), y(s))$ - натуральная параметризация кривой $\Gamma$, $s \in [0, 2\pi]$.
        \[S = \frac{1}{2}\int_{0}^{2\pi}(x\dot{y} - \dot{x}y) \diff s\]
        
        $\begin{cases*}
            x:[0, 2\pi] \to \Real\\
            y:[0, 2\pi] \to \Real
        \end{cases*}$ - принадлежат классу Фурье и непрерывны.

        \begin{align*}x(s) \leftrightharpoons \frac{a_0}{2} + \sum_{n = 1}^{\infty}a_n\cos(ns) + b_n\sin(ns) \\ 
         \dot{x}(s) \leftrightharpoons \sum_{n = 1}^{\infty} n(-a_n\sin(ns) + b_n\cos(ns)) 
        \end{align*}
        \begin{align*}
            y(s) \leftrightharpoons \frac{\tilde{a_0}}{2} + \sum_{n=1}^{\infty} (\tilde{a_n}\cos(ns) + \tilde{b_n}\sin(ns)) \\
            \dot{y}(s) \leftrightharpoons \sum_{n=1}^{\infty} n(-\tilde{a_n}\sin(ns) + \tilde{b_n}\cos(ns))
        \end{align*}
        Вспомним равенство Парсеваля:
        \[\int_{0}^{2\pi} fg = \pi \left( \frac{a_0(f)a_0(g)}{2} + \sum [a_n(f)a_n(g) + b_n(f)b_n(g)] \right)\]
        Тогда:
        \[\int_{0}^{2\pi}x\dot{y} = \pi \cdot \sum_{n \ge 1} n\left[a_n \tilde{b_n} - \tilde{a_n} b_n \right]\]
        Утверждение о том, что $\int_{0}^{2\pi}x\dot{y} = \int_{0}^{2\pi}-\dot{x}y$ было упражнением (либо раньше, либо я это не вписал).
        Итак: 
        \[S = \pi\sum_{n \ge 1} n\cdot \left[ a_n\tilde{b_n} - \tilde{a_n}b_n \right] \le \pi \sum n\left[\abs{a_n\tilde{b_n}} + \abs{\tilde{a_n}b_n}\right]\]
        Из $(a\pm b)^2 \ge 0 \implies \abs{ab} \le \frac{a^2 + b^2}{2}$. Значит:
        \[S \le \frac{\pi}{2}\sum n(a_n^2 \tilde{b_n}^2 +\tilde{a_n}^2 b_n)\]
        Длина нашей кривой $\Gamma$:
        \[L(\Gamma) = 2\pi = \int_{0}^{2\pi}\underbrace{1}_{\abs{\dot{\gamma}(t)}}\diff t = \int_{0}^{2\pi}(\dot{x}^2 + \dot{y}^2)\diff t\]
        По равенству Парсеваля:
        \[\int_{0}^{2\pi} \dot{x}^2 + \dot{y}^2 \diff t = \pi \left( \sum_{n \ge 1} n^2(a_n^2 + b_n^2) + \sum_{n \ge 1} n^2 (\tilde{a_n}^2 + \tilde{b_n}^2) \right) \overset{\text{по условию}}{=} 2\pi\]
        Сократим на $2$:
        \begin{equation}
            \label{isoperimetric}
            \frac{\pi}{2} \sum n (a_n^2 + b_n^2 + \tilde{a_n}^2 + \tilde{b_n}^2) \underbrace{\le}_{\text{тут}}  \frac{\pi}{2}\sum_{n=1}^{\infty} n^2(a_n^2 + b_n^2 + \tilde{a_n}^2 + \tilde{b_n}^2) = \pi \implies S \le \frac{\pi}{2}
        \end{equation}
        Итого $S \le \pi$. Осталось доказать, что если $S = \pi$, то $\Gamma$ - окружность. Если $S = \pi$, то $\forall n$ в неравенстве "тут" в \eqref{isoperimetric} $= n^2(\hdots) = n(\hdots) \implies n = 1$.
        Тогда $\forall n \ge 2$ $a_n = 0, \ b_n = 0, \ \tilde{a_n}  = 0, \ \tilde{b_n} = 0$.
        \[x(s) = \frac{a_0}{2} + a_1\cos(s) + b_1\sin(s), \ y(S) = \frac{\tilde{a_0}}{2} + \tilde{a_1}\cos(s) + \tilde{b_1}\sin(s)\]
        \[\frac{\pi}{2}(a_1^2 \tilde{b_1}^2 - \tilde{a_1}b_1^2 = \pi), \ S = \pi(a_1^2 \tilde{b_1}^2 - \tilde{a_1^2}\tilde{b_1}^2)\]
        Из $\abs{ab} = \frac{a^2 + b^2}{2} \implies a_1 = \pm b_1$ и доказываем, что это окружность.
    \end{proof}
\end{theorem*}

\documentclass[a4paper]{article}

\usepackage{cmap}
\usepackage[T2A]{fontenc}
\usepackage[utf8]{inputenc}
\usepackage[english,russian]{babel}
\usepackage{amsthm}
\usepackage{amssymb}
\usepackage{amsmath}
\usepackage{mathtools}
\usepackage{indentfirst}
\usepackage{fullpage}
\usepackage{titlesec}
\usepackage{multicol}
\usepackage{parskip}
\usepackage{graphicx}
\usepackage{tikz}
\usepackage{wrapfig}
\usepackage{breqn}

\newcommand{\open}{\underset{op}{\subset}}
\newcommand{\Real}{\mathbb{R}}
\newcommand{\Compl}{\mathbb{C}}
\newcommand{\Nat}{\mathbb{N}}
\newcommand{\scalar}[2]{\langle #1, #2 \rangle}
\newcommand{\diff}{\mathop{}\!d}
\newcommand{\abs}[1]{\left| #1 \right|}

\newtheoremstyle{definition}{3pt}{3pt}{\upshape}{}{\bfseries}{.}{.5em}{}
\theoremstyle{definition}
\newtheorem{definition}{Опр.}

\newtheoremstyle{statement}{3pt}{3pt}{\upshape}{}{\bfseries}{}{.5em}{}
\theoremstyle{statement}
\newtheorem{statement}{Утв.}
\newtheorem*{statement*}{Утв.}

\newtheoremstyle{lemma}{3pt}{3pt}{\upshape}{}{\bfseries}{}{.5em}{}
\theoremstyle{lemma}
\newtheorem{lemma}{Лемма}
\newtheorem*{lemma*}{Лемма}

\newtheoremstyle{note}{3pt}{3pt}{\upshape}{}{\bfseries}{:}{.5em}{}
\theoremstyle{note}
\newtheorem*{note}{Замечание}

\newtheorem{theorem}{Теорема}
\newtheorem*{theorem*}{Теорема}

\newtheoremstyle{example}{3pt}{3pt}{\upshape}{}{\bfseries}{:}{.5em}{}
\theoremstyle{example}
\newtheorem*{example}{Пример}

\title{Лекции по математическому анализу, 3 семестр}
\date{}
\author{Тимошенко Иван, 24123}


\begin{document}
    \maketitle
    \include{Lect1}
    \begin{definition}
    Функция дифференциируема в точке, если 
    \begin{itemize}
        \item $f:U \to \mathbb{R}^k$ и $p \in Int(U)$
        \item $f(x) = f(p) + df(p)\langle x-p \rangle + \alpha(x),$ где $\alpha(x) = o(x-p).$
    \end{itemize}
\end{definition}
Если $k = 1$, то лин. отображение $df(p):\mathbb{R}^n \to \mathbb{R}$
можно задатьа как $df(p)\langle v \rangle = \langle \nabla f(p);  v \rangle$
 - скалярное произведение градиента функции на вектор, причем 
 $\nabla f(p) = \left(\frac{\partial f}{\partial x_1}(p), \dots, \frac{\partial f}{\partial x_n}(p)\right)$ - вектор частных производных в точке $p$.

\begin{statement}
    Градиент функции задает направление, при движении в котором функция растет быстрее всего.
    \begin{proof}
        Рассмотрим функцию $f$ в точке $p$, вектор $v$ единичной длины будет задавать произвольное направление.
        \begin{equation*}\frac{f(p+tv) - f(p)}{t} \underset{t \to 0}{\to}
            \frac{\partial f}{\partial v} = df(p)\langle v \rangle = \langle \nabla f(p); v \rangle 
            = |\nabla f(p)| \cdot |v| \cdot cos(\varphi), \text{где $\varphi $ - угол между $\nabla f$ и $v$.}
        \end{equation*}
        Поскольку $|\nabla f(p) = const, |v| = 1$, то для максимизации надо выбрать такое $\varphi$, 
        чтобы $cos(varphi)$ был максимален, т.е. вектора $v$ и $\nabla f$ параллельны и $\nabla f$ задает наибольшую скорость роста.
    \end{proof}
\end{statement}

\begin{statement}
    $\nabla f(p)$ ортогонален поверхности уровня $\Omega = \{x | f(x) = c\}$.
    \begin{proof}
        Пусть $f(p) = c \ (p \in \Omega)$. Пусть $x_n \in \Omega$, покажем, что $cos(\nabla f(p), \overrightarrow{x_n-p}) \underset{n \to \infty}{\to} 0:$
        \begin{eqnarray*}
            f(x_n) = f(p) = c \implies 0 = f(x_n) - f(p) = df(p)\langle x_n-p \rangle + o(x_n - p) = 
            \langle \nabla f(p); x_n - p \rangle + o(x_n - p).
        \end{eqnarray*}
        Значит $0 \underset{n \to \infty}{=} \langle \nabla f(p); \frac{x_n - p}{|x_n - p|} \rangle + o(1)$, т.е.
        $\langle \nabla f(p); \frac{x_n - p}{|x_n - p|} \rangle \to 0 $. Тогда:
        \begin{equation*}
            \langle \nabla f(p); \frac{x_n - p}{|x_n - p|} \rangle = |\nabla f(p)|\cdot \left|\frac{x_n-p}{|x_n-p|}\right|\cdot cos(\alpha) \to 0, 
            \ \text{т.е.} \alpha \underset{n \to \infty}{\to} \frac{\pi}{2}. 
        \end{equation*}
        
    \end{proof}
\end{statement}

\begin{definition}
    Функция $f:\mathbb{R}^n \to \mathbb{R}^n$ называется векторным полем.
\end{definition}
\begin{definition}
    Потенциалом векторного поля $F$ (если он есть) называется \textbf{скалярная} функция $U:W \to \mathbb{R}$, 
    такая, что $\nabla U = F$. Если потенциал существует, то F называется потенциальным полем.

\end{definition}

\begin{theorem*}
    Пусть $f:U \subset \mathbb{R}^n \to \mathbb{R}^k, \ g:V \subset \mathbb{R}^k \to \mathbb{R}^m, \ f \in C^1(p), \ g \in C^1(q), \ q = f(p)$.\\
    Тогда $g \circ f \in C^1(p), \, dg \circ f = dg(f(p)) \cdot df(p).$ В матрицах Якоби: $D_{g \circ f}(p) = D_g(f(p)) \cdot D_f(p).$
\end{theorem*}

\begin{example}
    \begin{equation*}
        \begin{cases}
            f(x, y, z) = (xy, xz): \mathbb{R}^3 \to \mathbb{R}^2\\
            g(a, b) = \cosh(ab): \mathbb{R}^2 \to \mathbb{R}
        \end{cases} \quad
        f =
        \begin{cases}
            f_1(x, y, z) = xy\\
            f_2(x, y, z) = xz.
        \end{cases}
    \end{equation*}

    \begin{equation*}
        h = g(f(x, y, z)) = \cosh(xy \cdot xz): \mbox{R}^3 \to \mathbb{R}, \quad (x, y, z) \in \mathbb{R}^3 \overset{f}{\to} \mathbb{R}^3 \overset{g}{\to} \mathbb{R}
    \end{equation*}
    \begin{equation*}
        \frac{\partial h}{\partial x} = \sinh(x^2yz)\cdot 2xyz \quad
        \frac{\partial h}{\partial y} = \sinh(x^2yz)\cdot x^2z \quad
        \frac{\partial h}{\partial z} = \sinh(x^2yz) \cdot x^2y
    \end{equation*}

    \[D_f(x, y, z) = \begin{pmatrix}
        \frac{\partial f_1}{\partial x} & \frac{\partial f_1}{\partial y} &  \frac{\partial f_1}{\partial z} \\
        \frac{\partial f_2}{\partial x} & \frac{\partial f_2}{\partial y} &  \frac{\partial f_2}{\partial z}  
    \end{pmatrix} =
    \begin{pmatrix}
        y & x & 0 \\
        z & 0 & x
    \end{pmatrix}
    \]

    \[D_f = \begin{pmatrix}
        \sinh(ab)\cdot a & \sinh(ab) \cdot b
    \end{pmatrix}\]

    \[D_g \cdot D_f = \begin{pmatrix}
        \sinh(x^2yz)\cdot xz & \sinh(x^2yz)\cdot xy
    \end{pmatrix} \cdot \begin{pmatrix}
        y & x & 0 \\
        z & 0 & x
    \end{pmatrix}
    \]
    Досчитывать я это не буду, поверим Сторожуку на слово.
\end{example}

\textbf{Правило дифференциирования обратного отображения:}
Если невырождено и $\exists$ обратное отображение $g:V \to U$, непрерывное в точке $q = f(p)$, тогда:
\[g \in D(q) \text{ и } dg(q) = (df(p))^{-1}\]


\subsection{Многократная дифференциируемость}

\begin{definition}
    $f: U \subset \mathbb{R}^n \to \mathbb{R}^m \ k$ раз дифференциируема в точке $p$ ($f \in D^k(p)$), если:
    \begin{enumerate}
        \item $f$ дифференциируема во всех точках некоторой окрестности точки $p$;
        \item Все частные производные $\frac{\partial f}{\partial x_1}, \dots, \frac{\partial f}{\partial x_n}$ дифференциируемы $k-1$ раз в точке $p$.
    \end{enumerate}  
\end{definition}

\begin{example}
    \[f \in D^2(p) \implies f \in D(x) \text{ и } \frac{\partial f}{\partial x}, \frac{\partial f}{\partial y} \in D(p)  \]
\end{example}

\begin{statement*}
    Если $\begin{cases}
        f \in D^k(p): \mathbb{R}^n \to \mathbb{R}^k\\
        g \in D^k(p): \mathbb{R}^n \to \mathbb{R}^k
    \end{cases}$ тогда $h(x) = f(x) \cdot g(x) \in D^k(p)$

    \begin{proof}
        \begin{equation*}
            \frac{\partial h}{\partial x_i}(x) = \frac{\partial f}{\partial x_i}(x)\cdot g(x) +
            f(x) \cdot \frac{\partial g}{\partial x_i}
        \end{equation*}
        Так как $\frac{\partial f}{\partial x_i}(x) \in D^{k-1}(p), \ g(x) \in D^k(p), \ f(x) \in D^k(p), \ \frac{\partial g}{\partial x_i} \in D^{k-1}(p)$,
        то $\frac{\partial h}{\partial x_i} \in D^{k-1}(p)$.
    \end{proof}
\end{statement*}


\begin{theorem}[о вторых производных]
    Пусть $f:U \subset \mathbb{R}^n \to \mathbb{R}, \ f \in D^2(p)$. Тогда $\frac{\partial^2 f}{\partial x \partial y} = \frac{\partial^2 g}{\partial y \partial x}$.
    \begin{proof}
        \textbf{ОЧЕНЬ ХОЧУ ДОКАТАЗАТЕЛЬСТВО ПРЯМ ЖЕСТЬ КАК}
    \end{proof}
\end{theorem}
    \textbf{Правило дифференциирования монома:\\}
\bigskip
Пусть $f(x)  = x_1^{i_1} \cdot x_2^{i_2} \cdot \dots \cdot x_m^{i_m}, \ x = (x_1, \dots, x_m)$. Тогда 
\[\frac{\partial^{i_1+i_2+ \dots + i_m} f}{\partial x_1^{i_1} \dots \partial x_m^{i_m}}(0) = i_1! \cdot \hdots \cdot i_m!\]
Любая другая производная любого порядка в точке $0$ равна $0$.

\subsection{Мульти-индексы}
Придумаем $\mu = (i_1, \hdots, i_m)$ - численный вектор, в котором $\forall j = \overline{1 \hdots m} \ i_j \geq 0$ и назовем его \textbf{мультииндексом}.
\par
Для мультииндексов определены операции:
\[\mu! := i_1! \cdot \hdots \cdot  i_m! \qquad \left| \mu \right| = \sum_{j = 1}^{m} i_j \text{ - порядок мультииндекса}\]
\[x \in \mathbb{R}^m, x = (x_1, \hdots, x_m), \quad x^{\mu} = x_1^{i_1} \cdot \hdots \cdot x_m^{i_m}\]
\[C^{\mu}_k = \frac{k!}{\mu!} = \frac{k!}{i_1!\cdot \hdots \cdot i_m!} \text{, где } k = \left|\mu\right|\]

    \section{Основы гладкого анализа}
Символ $\open$ обозначает "открыто в". Контекст:
\[U \open \mathbb{R}^m, \ f:U \to \mathbb{R}^k, \ f \in C^r(U), \ r \geq 0\]

\begin{definition}
    Отображение $f$ называется $r$-гладким, если все ее частные производные до порядка $r$ непрерывны на $U$.
\end{definition}

Пусть $X$ - не обязательно открыто в $\mathbb{R}^m$.
\begin{definition}
    $f \in C^r(X)$, если $f = \tilde{f}$ - сужение на $\tilde{X}$, $\tilde{f}: \tilde{X} \open \mathbb{R}^m \to \mathbb{R}^k$ - $C^r$-гладкая на $tilde{X}$. \textbf{НАДО УТОЧНИТЬ:} $X \subset \tilde{X}$ или наоборот.
\end{definition}

\begin{statement}
    Пусть $f:X \subset \mathbb{R}^m \to \mathbb{R}^k, \ g:X \to \mathbb{R}^k$ - $C^r$ отображения. Тогда $f+g \in C^r(X)$
    \begin{proof}
        Пусть $f = \tilde{f}, \ g = \tilde{g}$ и т.д. по определению $r$-гладкости:
        \[\tilde{f}:U \to \mathbb{R}^m, \tilde{g}: V \to \mathbb{R}^m, \ U, V \open \mathbb{R}^m, \ X \subset U, X \subset V\]
        Введем $U \cap V = W \open \mathbb{R}^m$. На $W$ заданы оба отображения и ясно, что $f+g = \tilde{f} + \tilde{g}$.
    \end{proof}
\end{statement}

\begin{statement}
    Композиция:
    \[X \overset{f}{\to} \mathbb{R}^k \supset Y \overset{g}{\to} \mathbb{R}^m\]
    Если $f \in C^r$ и $g \in C^r$, то $g \circ f \in C^r$.
    \begin{proof}
        Область определения $\mathrm{dom}(g \circ f) = \{x \in X | \ f(x) \in Y\} = X \cap f^{-1}(\mathrm{dom}(g))$
        \[\begin{cases}
            f \in C^r \implies f = \tilde{f}, \ \tilde{f}:\mathbb{R}^m \underset{op}{\supset}\tilde{X} \to \mathbb{R}^k -  C^r\text{-гладкое.} \\
            g \in C^r \implies g = \tilde{g}, \ \tilde{g}:\mathbb{R}^k \underset{op}{\supset}\tilde{Y} \to \mathbb{R}^m -  C^r\text{-гладкое.} \\
        \end{cases}\]
        \[\mathrm{dom(\tilde{g}\circ \tilde{f})} = \mathrm{dom}(\tilde{f}) \cap \tilde{f}^{-1}(\mathrm{dom}(\tilde{g})) = \tilde{X} \cap \tilde{f}^{-1}(\tilde{Y})\]
        $\tilde{X}$ - открытое, $\tilde{f}^{-1}(\tilde{Y})$ - открытое, как прообраз открытого множества $\tilde{Y}$ при непрерывном отображении.
        \newline
        Ясно, что $g \circ f = \tilde{g} \circ \tilde{f}$ - сужение $\mathrm{dom}(g \circ f)$.
    \end{proof}
\end{statement}

\begin{theorem}[Лемма о классе гладкости обратного отображения]
    Пусть $U, V \open \mathbb{R}^m$. \\
    $U \overunderset{f}{g}{\rightleftarrows} V$, $f, g$ - непрерывны и взаимно обратны. Если $f \in C^r(U)$ и $\forall x \in U \ df(x)$ - невырожден: $det(Df(x)) \neq 0$, то $g \in C^r(U)$.
    \begin{proof}
        При $r > 0$  $g$ дифференциируема в $\forall y \in V$ по правилу дифференциирования обратного отображения. В матрицах Якоби:
        \[Dg(f(x)) = (Df(x))^{-1}, \forall x \in U \text{, причем } f(x) = y, \ x = g(y)\]
        \[Dg(y) = (Df(g(y)))^{-1}, \quad \text{цепочка преобразований} y \to g(y) \to Df(g(y)) \to (Df(g(y)))^{-1}\]
        \[Dg(y) = w \circ Df \circ g(y), \ w \text{- отображение обращения матрицы.}\]
        \[Dg(y) = w \circ Df \circ g(y), \text{ причем } w - C^{\infty}, Df - C^{r-1}, g(y) - \text{дифференциируема}\]
        $g$ дифференциируема $\implies Dg$ тоже дифференциируема как композиция $\implies$ все производные $g$ дифференциируемы $\implies Dg \in C^1 \implies g \in C^2 \implies \hdots \implies g \in C^{r-1} \implies Dg \in C^{r-1} \implies g \in C^r$.
    \end{proof} 
\end{theorem}
    \section{Многообразия в $\mathbb{R}^n$}
\subsection{Многообразия}
\begin{definition}
    Пусть $M$ - метрическое пространство (необязательно подмножество в $\mathbb{R}^n$). $M$ является $k$-мерным многообразием без края,
    если $\forall p \in M$ у точки $p \ \exists $ окрестность $U_p \open M$ гомеоморфная открытому шару в $\mathbb{R}^k$.
\end{definition}

\begin{definition}
    Гомеоморфизм - непрерывное отображение, обратное к которому тоже непрерывно.
\end{definition}

\begin{theorem}[Брауэра об инвариантности области (без док-ва).]
    \par Пусть $U \open \mathbb{R}^k$ и $f:U \to \mathbb{R}^k$ - непрерывна и инъективна. \newline
    Тогда $f(U) \open \mathbb{R}^k$ и $f^{-1}: f(U) \to U$ - тоже непрерывна.
\end{theorem}

\begin{definition}
    $M$ - $k$-мерное $C^r$-многобразие в $R^n$, если:
    \[\forall p \in M \ \exists U \in \mathcal{N}(p), \ U \open M \text{ такая, что } U \overset{C^r}{\cong} \text{ открытому шару в } \mathbb{R}^k\]
\end{definition}

\begin{statement*}
    Предыдущее утверждение эквивалентно требованиям:
    \begin{enumerate}
        \item \[ \exists U \in \mathcal{N}(p) \ U \open M \text{ такая, что } U \cong \text{ открытому подмножеству } \omega \in \mathbb{R}^k\]
        \item \[ \exists U \in \mathcal{N}(p) \ U \open M \text{ такая, что } U \cong \mathbb{R}^k\]
    \end{enumerate}
\end{statement*}

\begin{theorem*}[1 о регулярных решениях]
    Пусть $\Omega \open \mathbb{R}^n, \ f_1, \hdots, f_k: \Omega \to \mathbb{R}$ - $C^r$ гладкие функции.
    Непустое множество $M$ задано как 
    \[M = \{\overline{x} \in \Omega \mid 
    \begin{array}{c}
        f_1(x) = 0 \\
        \vdots \\
        f_k(x) = 0
    \end{array}\}, \text{ где $\overline{x}$ - регулярная точка}\]
    То есть $\mathrm{rank} \frac{\partial f_{1 \hdots k}}{\partial x_{1 \hdots n}} = k$
    Тогда $M$ - $n-k$-мерное $C^r$-многобразие без края.

    \begin{proof}
        Пусть $p \in M$. Можно считать, что последние $k$ столбцов линейно независимы, то есть $\left| \frac{\partial f_{1\hdots k}}{\partial x_{n-k+1 \hdots n}}\right| \neq 0$ - определитель матрицы Якоби.  
        По теореме о неявной функции существует такая окрестность $\tilde{U}$ точки $p$, $\tilde{U} \open \mathbb{R}^n$, такая, что $\tilde{U} \cap M$ - график некоторой функции:
        \[x_{n-k+1 \hdots n} = g(x_{1 \hdots n-k})\]
        Осталось показать, что график функции является многобразием $U \subset \mathbb{R}^k$.
        \begin{lemma*}
            График любого $C^r$ отображения $g$, определенного на открытом подмножестве - это многобразие, гомеоморфное $U$.
            \begin{proof}
                \[X \overset{g}{\to} Y\]
                График $\Gamma_g = \{(x, g(x)) \mid x \in X\}$
                \[U \overset{g}{\to} \Gamma_g \ - C^r \text{ отображение}\]
                \[x \overset{g}{\to} (x, g(x)) \overset{g^-1 - \text{проекция на X}}{\to} X\]
            \end{proof}
        \end{lemma*}
        По лемме теорема доказана.
    \end{proof}
\end{theorem*}

\begin{note}
    Если у градиента функции в точке $p$ хотя бы одна координата не равна $0$, то $p$ - регулярная.
\end{note}

\begin{statement*}
    $X \overset{C^r}{\cong} Y$ - если $X$ - $C^r$-многобразие, то $Y$ - тоже.
    \begin{proof}
        Пусть $\psi: X \to Y$ - $C^r$-изоморфизм. Пусть $q\in Y, p = \psi^{-1}(q)$, по условию существует открытая
        окрестность $V$ точки $p$, $C^r$-изоморфная $U \open \mathbb{R}^k$. 
        \[U \overset{\varphi}{\cong} V \overset{\psi}{\cong} \psi(V) \open Y\]
        $\psi(V)$ - прообраз $V$ под действием $\psi^{-1}, V = \psi^{-1}(\psi(V))$\\
        $\psi \circ \varphi$ - $C^r$-изоморфизм $U$ на окрестность точки $q$ в $X$.
    \end{proof}
\end{statement*}

\begin{lemma*}[о локальном вложении]
    Пусть $f = \begin{array}{c}
        f_1(x_1, \hdots, x_k) \\
        \vdots \\
        f_n(x_1, \hdots, x_k)
    \end{array}$, $U \open \mathbb{R}^k, f: U \to \mathbb{R}^n$.\\
    Если точка $\overline{x_0} \in U$ такая, что $\frac{\partial f}{\partial x}(x_0) = k$, то у точки $x_0$ существует окрестность $\tilde{U}$ такая, что $f(\tilde{U})$ - $k$-мерное многобразие.

    \begin{proof}
        \[\begin{pmatrix}
            \frac{\partial f_1}{\partial x_1} & \hdots & \frac{\partial f_1}{\partial x_k} \\
            \vdots & & \vdots\\
            \frac{\partial f_k}{\partial x_1} & \hdots & \frac{\partial f_k}{\partial x_k} \\
            \vdots & & \vdots\\
            \frac{\partial f_n}{\partial x_1} & \hdots & \frac{\partial f_n}{\partial x_k}
        \end{pmatrix}\]
        Переставим $f$ (если надо) и считаем первые $k$ строк невырожденными в $x_0$.
        Рассмотрим урезанное отображение $\psi = (\begin{array}{c} f_1 \\ \vdots \\ f_k \end{array}): U \to \mathbb{R}^k$\\
        По теореме о локальной обратимости существует окрестность $\tilde{U} \ni x_0$, такая, что 
        $\psi(\tilde{U}) \open \mathbb{R}^k$ и $\psi|_{\tilde{U}}: \tilde{U} \to \psi(\tilde{U})$ - $C^r$-изоморфизм.
        \[\psi(\tilde{U}) \overset{\psi^{-1}}{\to} \tilde{U}\]
        $\tilde{U}$ в свою очередь, отображается в $f(\tilde{U})$ данным отображением:
        \[\begin{array}{c}
            f_1(\tilde{x_1}, \hdots, \tilde{x_k}) \\
            \vdots \\
            f_k(\tilde{x_1}, \hdots, \tilde{x_k}) \\
            f_{\text{остальные индексы}}(\tilde{x_1}, \hdots, \tilde{x_k})
        \end{array}\]
        Причем $f_1, \hdots, f_k = (x_1, \hdots, x_k)$. 
        \[f_{\text{остальные}}(\tilde{x_1}, \hdots, \tilde{x_k}) = f_{\text{остальные}}\circ \psi(x_1, \hdots, x_k)\]
        $f(\tilde{U})= \{(\overline{x}, \psi(\overline{x})) \mid \overline{x} \in \tilde{U}\}$ - график отображения $\psi$, то есть по соответствующей лемме о графике это многобразие.
    \end{proof}
\end{lemma*}

\newpage
\begin{definition}
    $M$ - многобразие (с краем, а может и без), если $\forall p \in M \ \exists U \open M \ C^r$-изоморфно 
    \begin{enumerate}
        \item $\tilde{U} \open \mathbb{R}^k$
        \item $\mathbb{R}^k_+ = \{x_1, \hdots, x_k \mid x_1 \geq 0\}$ 
    \end{enumerate}
\end{definition}

\begin{definition}
    Точка $p \in \partial M$ (принадлежит краю), если у точки $p$ $\nexists$ окрестности первого типа.
\end{definition}

\begin{lemma*}[о крае полупространства]
    $\mathbb{R}^k_+$ - многообразие с краем, $\partial \mathbb{R}^k_+ = \{(x_1, \hdots, x_k) \mid x_1 = 0\}$. При этом $\mathbb{R}^k_+$ называется полупространством.
    \begin{proof}
        $\mathbb{R}^k_+$ является многобразием по опрделению, ведь любая ее точка гарантированно имеет окрестность, открытую в $\mathbb{R}^k_+$, например, в качестве такой окрестности можно взять само $\mathbb{R}^k_+$.
        \par
        Пусть $x_1 > 0$, тогда очевидно, что множество $\mathbb{R}^k_+ \cap \{(x_1, \hdots, x_k) \mid x_1 > 0\}$ открытое в $\mathbb{R}^k_+$ является окрестностью точки $p$ и поэтому $p \in M \backslash \partial M$
        \par
        Пусть $x_1 = 0$. Ясно, что никакая окрестность точки $p$ в $\mathbb{R}^k_+$ не является открытым множеством в $\mathbb{R}^k_+$.
        Однако, нам было необходимо более сложное утверждение, а именно - нужно было показать, что $U$ не может быть $C^r$-изоморфно открытому подмножеству $\Omega \in \mathbb{R}^k_+$. 
        Пусть $r > 0$ в $C^r$, по теореме о локальной обратимости если $\Omega \open \mathbb{R}^k$ и $\psi: \Omega \to \psi(\Omega) \subset \mathbb{R}^k$ $C^r$-изоморфизм, то $\psi(\Omega) \open \mbox{R}^k$.
        \par
        И наконец, $r \neq 0$ по теореме Брауэра об инвариантности области.
    \end{proof}
\end{lemma*}

\begin{lemma*}[об изоморфизме многообразий]
    Пусть $X \subset \mathbb{R}^n$ - $C^r$-многобразие. Отображение $\varphi: X \overset{C^r}{\cong} Y \subset \mathbb{R}^n$ такое, что $\varphi(\partial X) = \partial Y$. Тогда $Y$ - тоже $C^r$-многообразие. 
    \par
    Более "стильная" \copyright \ формулировка: $\varphi(\partial X) = \partial(\varphi(X))$
    \begin{proof}
        Пусть $q \in Y$. Пусть $p = \varphi^{-1}(q) \in X$. Если $p \in X \backslash \partial X$, то $\exists U \open X \mid p \in U \overset{\psi}{\cong} W \open \mathbb{R}^k$, 
        тогда $\varphi(U) = \varphi \circ \psi^{-1}$ - $C^r$-изоморфизм как композиция. Т.е. мы нашли окрестность точки $q \in Y$, которая $C^r$-изоморфна открытому в $\mathbb{R}^k$ множеству.
        \par
        Пусть $p \in \partial X$, тогда \underline{не существует} окрестности $U \open X$, $C^r$-изоморфной подмножеству $W$, открытому в $\mathbb{R}^k$.\\
        Если предположить, что $q \in \partial Y$, то $\exists V \open \mathbb{R}^k$ и $\psi: V \underset{\cong}{\to} W \open \mathbb{R}^k$. Тогда $\varphi^{-1}\circ\psi^{-1}(W)\open X$ - открытая окрестность точки $p \in X$, а композиция - это $C^r$-изоморфизм $W$, чего не может быть по условию. Получаем противоречие.
    \end{proof}
\end{lemma*}

\begin{lemma*}[об открытых частях многообразия]
    $X$ - $C^r$-многообразие в $\mathbb{R}^n$, $U \open X$, тогда $U$ - тоже $C^r$-многообразие той же размерности, а $\partial U = U \cap \partial X$

    \begin{proof}
        Пусть $p \in X$, тогда у $p$ существует окрестность $V \open X$, такая, что $V \overset{\varphi}{\cong}$ открытому подмножеству $W$ в $\mathbb{R}^k$ или $\mathbb{R}^k_+$. \\
        Если $p \in U$, то множество $V \cap U \open U$ как пересечение открытых (лемма из 2 семестра 1 курса о пересечении открытых метрических пространств). 
        \[\varphi(U \cap V) \open W \open \mathbb{R}^k \text{ или } \mathbb{R}^k_+ \implies \varphi(V \cap U) \open \mathbb{R}^k \text{ или } \mathbb{R}^k_+\]
        Пусть $p \in U$. Если $p \notin \partial X$, то $W$ в предыдущих строчках можно выбрать первого типа ($W \in \mathbb{R}^k$). Значит $p \notin \partial U$. 
        \par
        Если $p \notin \partial U$, тогда (поскольку мы уже доказали, что $U$ - многообразие) $\exists \Omega \open U \mid p \in \Omega \overset{\psi}{\cong} A \open \mathbb{R}^k$. Но посольку $U \open X$, выполнено утверждение $\Omega \open X \implies p \notin \partial X$.
    \end{proof}
\end{lemma*}

\begin{theorem}[о крае многообразия]
    Пусть $X$ - $C^r$-многообразие размерности $k$. Тогда если его край $\partial X$ непуст, то он является $C^r$-многообразием без края размерности $k-1$ ($\partial \partial X = \varnothing$)
    \begin{proof}
        Пусть $p \in \partial X$. Надо показать, что $\exists U \open \partial X$, такое, что $U \cong $ открытому подмножеству $W$ в $\mathbb{R}^{k-1}$.
        \par
        По условию у точки $p$ существует открытая в $X$ окрестность $\tilde{U} \open X$, такая, что $\tilde{U} \overset{\varphi}{\cong} W \open \mathbb{R}^k_+$. 
        
        \begin{multline*}
                p \in \partial X \implies p \in \partial \tilde{U} (\text{ по предыдущей лемме}) \implies \\
                \implies \varphi(p) \in \partial \tilde{W} \implies \text{первая координата точки $\varphi(p)$ равна $0$}
            \end{multline*}

            \[\tilde{W} \open \mathbb{R}^k_+ \implies \tilde{W} \cap \{x_1, \hdots, x_k \mid x_1 = 0\} \open \mathbb{R}^k \text{ - по лемме об открытых частях подпространства}\]
            Заметим, что пересечение этих множест это $W$, а $\{x_1, \hdots, x_k \mid x_1 = 0\} = \mathbb{R}^{k-1}$
            $\varphi^{-1}(W) \open \partial X$ (как прообраз открытого), значит $\varphi^{-1}(W)$ - искомая окрестность $U$.
    \end{proof}
\end{theorem}


\begin{theorem}[2 о регулярных решениях]
    Пусть $\Omega \open \mathbb{R}^n, \ h, f_1, \hdots, f_k: \Omega \to \mathbb{R}$. 
    \par
    Множество $M$ регулярных решений системы 
    \begin{equation} 
        \begin{cases}
            \label{regular_solutions_system}  
            h(x) \geq 0 \\
            f_1(x) = 0 \\
            \vdots \\
            f_k(x) = 0
        \end{cases}
    \end{equation}
    является $n-k$-мерным $C^r$-гладким многообразием, край которого задается уравнением $\begin{cases*}
        h(x) = 0\\ f_i(x) = 0
    \end{cases*}$

    \begin{definition}
        Пусть $p$ - решение системы \eqref{regular_solutions_system}. 
        \begin{itemize}
            \item Если $h(p) > 0$ и $\nabla f_i(p)$ линейно независимы, то $p$ - регулярна 
            \item Если $h(p) = 0$ и $\nabla h, \nabla f_1, \hdots, \nabla f_k$ линейно независимы, то $p$ - регулярна 
        \end{itemize} 
    \end{definition}

    \begin{proof}
        Пусть $p \in M$
        \begin{enumerate}
            \item Если $h(p) > 0$, то у точки $p$ существует окрестность $W \open \mathbb{R}^n$, в которой $h(W) > 0$.
            Множество 
            \begin{equation}
                \begin{cases}
                    h(x) > 0\\
                    f_1(x) = 0 \\
                    \vdots \\
                    f_k(x) = 0
                \end{cases} = M \cap W \open M
            \end{equation}
            Итак, множество $M \cap W$ совпадает с множеством тех точек из $\Omega \cap W \open \mathbb{R}^n$, где $\begin{cases}
                f_1 = 0 \\ \vdots \\ f_k = 0
            \end{cases}$ - многообразие размерности $n-k$

            \item Если $h(p) = 0$,\\ то продолжим наш набор отображений и получим набор 
            $\varphi = (h, f_1, \hdots, f_k, f_{k+1}, \hdots, f_{n-1})$ так, чтобы $n$ штук функций 
            были регулярными в окрестности $U$ точки $p$ (это разрешается сделать в силу леммы о регулярном дополнении). 
            \[M \cap U = \{x \in M \mid x \in U\}, \text{ причем } x \in M \equiv \begin{cases}
                h(x) \geq = 0 \\ f_1(x) = 0 \\ \vdots \\ f_k(x) = 0
            \end{cases}\]
            \[\text{Отображение }\varphi = \begin{array}{c} 
                h \\ f_1 \\ \vdots \\ f_k \\ f_{k+1} \\ \vdots \\ f_{n-1}
            \end{array}: U \to \varphi(U) \text{$C^r$-изоморфизм}\]
            $\varphi(U) \open \mathbb{R}^n$ по лемме о локальной обратимости и 
            \begin{multline*}
                \varphi(M \cap U) = (y_1, \hdots, y_n) = \overline{y} \in \varphi(U) = \left(\begin{array}{c}
                y_1 \geq 0 \\ y_2 = 0 \\ \vdots \\ y_{k+1} \\ \vdots \\ y_{n-1}
            \end{array}\right) \text{, где на $y_{k+1} \hdots y_{n-1}$ не ограничений} \\ - \text{ это некоторое $n-k$ мерное подпространство $Q$, пересеченное с $\varphi(U) \open \mathbb{R}^n$}
            \end{multline*}
            Тогда $Q \cap \varphi(U) \open Q$
        \end{enumerate}
    \end{proof}
\end{theorem}

\subsection{Касательные пространства}

\begin{definition}
    Пусть $p \in X \subset \mathbb{R}^n$. Вектор $v$ называется касательным вектором в точке $p$ к множеству $X$ ($v \in T_p(X)$), если существует $ T \subset \left[ 0, \varepsilon\right]$, содержащая $0$, такая, что $0$ - предельная точка множества $T$
    и $\exists$ отображение $\gamma: T \to X$ такое, что $\gamma(0) = p, \gamma'_T(0) = v$. При этом $\gamma'_T = \lim_{t \to 0|_T}\frac{\gamma(t) - \gamma(0)}{t}$
    \par
    На бытовом уровне (человеческий перевод): \\
    $v$ - касательный вектор к в $p$ к $X$, если из точки $p$ возможно двигаться по $X$ с начальной скоростью $v$
    \par
    В таком случае $T_p(X)$ называется касательным пространством к $X$ в $p$ и \textbf{обязательно} проходит через нулевую точку пространства. А чтобы $T_p(X)$ проходило через точку $p$ - придумали специальное множество $K_p(X) = p + T_p(X)$, называемую контингенцией (чтобы сложение точки и пространства не смущало читателя - считайте $p$ вектором из нулевой точки пространства в $p$).
\end{definition}

\newpage
\begin{theorem}
    \underline{Свойства} $T_p(X)$:
    \begin{enumerate}
        \item Пусть $p \in X \subset \mathbb{R}^n, 0 \neq v \in \mathbb{R}^n$. $v$ - касательный вектор $\iff$ $\exists$ последовательность точек $x_n \to p$, такая, что $\frac{x_n - p}{\left| x_n - p \right|} \to \frac{v}{\left| v \right|}$ (угол между векторами $x_n - p$ и $v$ стремится к $0$).
        \item Ноль всегда $\in T_p(X)$
        \item $T_p(X)$ - замкнутое пространство
        \item $T_p(X)$ - конус с вершиной в 0, т.е. $v \in T_p(X) \implies \forall \lambda \geq 0 \ \lambda v \in T_p(X)$
        \item $p \in A \subset B \implies T_p(A) \subset T_p(B)$
        \item Локальность:
        \[p \in X, U - \text{ окрестность } p \text{ в } X \implies T_p(X) = T_p(X)\]
    \end{enumerate}
\end{theorem}

\begin{statement*}
    $v \in T_p(X) \iff \exists x_n \in X, \ x_n \to p$ и $t_n > 0, t_n \to 0$ так, что 
    \[\frac{x_n - p}{t_n} \underset{n \to \infty}{\to} v\]
    \begin{proof}
        Слева направо ($\implies$):
        \par
        Пусть $v \in T_p(X)$. Выберем $T$ и $\varphi$ как в определении. Т.к. $0$ - предельная точка $T$ (из определения), то $\exists t_n \in T, t_n \to 0$, положим $x_n = \varphi(t_n)$. Вот так вот раз-раз и готово.
        Справа налево ($\impliedby$):
        \par
        Пусть есть $x_n \in X, t_n \to 0$ и $\frac{x_n - p}{t_n} \to v$. Положим $T = \{0, t_1, t_2, \hdots\}, \varphi(t_n)$ приравняем к $x_n$.
    \end{proof}
\end{statement*}

\begin{definition}
    $K \subset \mathbb{R}^n$ называется полупространством размерности $k \leq n$, если $\exists e_1, \hdots, e_k$ - линейно независимый набор из $\mathbb{R}^n$, такой, что $K = \{(t_1e_1 + \hdots + t_ke_k) \mid t_1 \geq 0, t_{>1} - \text{любые}\}$
    \par
    Любое векторное полупространство в $\mathbb{R}^n$ - замкнутый конус с вершиной в любой точке границы.
\end{definition}

\subsection{Дифференциал гладкого отображения}
Дифференциал - отображение касательных пространств. Пусть $\mathbb{R}^n \supset X \overset{\varphi}{\to} Y \subset \mathbb{R}^n$.

\begin{definition}
    $p \in T_p(X), \varphi - C^r$-гладкое отображение. 
    \[d\varphi(p): T_p(X) \to T_{\varphi(p)}(Y) \text{ определим следующим образом:}\]
    \[d\varphi(p)\langle v \rangle = d\tilde{\varphi}(p) \langle v \rangle, \text{ где }\tilde{\varphi} - C^r\text{-продолжение отображения $\varphi$ на окрестность множества }\] 
    \par
    Покажем корректность определения:
    \par
    \begin{proof}
        
        Почему значение дифференциала не зависит от выбора $\varphi$, почему образ лежит в $T_q(Y), q = \varphi(p)$? \par
        
        Пусть $v \in T_p(X), v = \lim_{n \to \infty} \frac{x_n - p}{t_n}$. Покажем, что $d\varphi(p)\langle v \rangle = \lim_{n \to \infty} \frac{\varphi(x_n) - \varphi(p)}{t_n}$. Пусть $\tilde{\varphi}$ - некоторое $C^r$-продолжение отображения $\varphi$. 
        \[p, x_n \in X \implies \varphi(x_n) = \tilde{\varphi}(x_n), \tilde(\varphi)(p) = \varphi(p)\]    
        \[\varphi(x_n) = \tilde{\varphi}(x_n) = \tilde{\varphi}(p) + d\tilde{\varphi}(p)\langle x_n - p \rangle + o(\left| x_n - p \right|) \text{ - из определения } d\tilde{\varphi}\]
        \[\frac{\varphi(x_n) - \varphi(p)}{t_n} = \frac{\tilde{\varphi}(x_n) - \tilde{\varphi}(p)}{t_n} = \frac{d\tilde{\varphi}(p)\langle x_n - p \rangle}{t_n} + o(\frac{x_n - p}{t_n})\]
        Отображене $d\tilde{\varphi}(p):\mathbb{R}^n \to \mathbb{R}^m$ линейно, поэтому $\frac{d\tilde{\varphi}(p)\langle x_n - p \rangle}{t_n} = d\tilde{\varphi}(p)\langle \frac{x_n - p}{t_n} \rangle$
        \par
        Итак, $\frac{\varphi(x_n) - q}{t_n} \underset{n \to \infty}{=} d\tilde{\varphi}(p)\langle \frac{x_n - p}{t_n} \rangle + o(\frac{x_n - p}{t_n})$. Поскольку $\frac{x_n - p}{t_n} \underset{n \to \infty}{\to} v$, то 
        \[\lim_{n \to \infty} \frac{\varphi(x_n) - q}{t_n} = d\tilde{\varphi}(p)\langle v \rangle + o(\hdots)\]
    \end{proof}
\end{definition}

\textbf{Свойства дифференциала:}
\[p \in X \overset{\varphi}{\to} Y \overset{\psi}{\to} Z\]
\begin{enumerate}
    \item Композиция: \[d(\psi \circ \varphi)(p): T_p(X) \to T_{\psi \circ \varphi(p)}(Z)\]
    \[d(\psi \circ \varphi)(p) = d\psi(\varphi(p))\circ d\varphi(p), \underset{\forall v \in T_p(X)}{=} \forall v \in T_p(X) \ d\psi(\varphi(p))\langle d\varphi(p)\langle v \rangle \rangle\]
    \item Если $\varphi: X \to Y - C^r$-изоморфизм, то $T_p(X)$ и $T_q(Y)$ линейно изоморфны ($\exists$ линейный изоморфизм этих пространств).
\end{enumerate}

\begin{statement*}
    Пусть $M$ - $k$-мерное гладкое множество в $\mathbb{R}^n$. 
    \begin{enumerate}
        \item Если $p \in M\backslash \partial M$, то $T_p(M)$ - $k$-мерное векторное пространство.
        \item Если $p \in \partial M$, то $T_p(M)$ - $k$-мерное полупространство.
    \end{enumerate}

    \begin{proof}
        Было утверждение: $\varphi: (p \in) A \cong B$, тогда $T_p(A) \overset{d\varphi(p)}{\cong} T_{\varphi(p)}(B)$ - линейно изоморфны.

        \begin{enumerate}
            \item Пусть $\exists U \open M, p \in U$ и $\exists \varphi: U \cong \mathbb{R}^k$. Тогда $d\varphi(p):T_p(U) \to T_{\varphi(p)}(\mathbb{R}^k) = \mathbb{R}^k$. А по свойству локальности касательных пространств $T_p(U) = T_p(M)$.
            \item Пусть $\exists U \open M, p \in U$ и $\varphi: U \cong \mathbb{R}^k_+, \varphi(p) \in \partial \mathbb{R}^k_+$
            \[d\varphi(p): T_p(M) \overset{\text{линейно}}{\cong} T_{\varphi(p)}(\mathbb{R}^k_+) = \mathbb{R}^k_+\]
        \end{enumerate}
    \end{proof}
\end{statement*}

\begin{theorem}[о касательном пространстве к регулярному решению систему уравнений]
    \par
    Пусть $p$ - регулярное решение системы уравнений $\begin{array}{c} f_1(\overline{x}) = 0 \\ \vdots \\ f_k(\overline{x}) = 0 \end{array}, \ f_i$ - $C^r$-отображение $\Omega \open \mathbb{R}^n \to R$.
    Точка $p$ - регулярная $\iff$ $rank(df_1, \hdots, df_k) = k$ - матрица Якоби, в которой дифференциал - это вектор столбец матрицы.
    \par
    Тогда $T_p(M)$ ($M$ - множество всех регулярных решений) - это ортогональное дополнение к $\nabla f_1(p) \oplus \hdots \oplus \nabla f_k(p)$ = множеству решений линейной системы уравнений $\begin{array}{c} df_1(p)\langle v \rangle = 0 \\ \vdots \\ df_k(p)\langle v \rangle = 0
    \end{array}$
    
    Напоминание:
    Пусть $V \subseteq \mathbb{R}^n$ (не обязательно векторное пространство). $V^{\bot} \text{(орт. дополнение)} = \{ u \in \mathbb{R}^n \mid \forall v \in V \ u \bot v \}$
    
    \begin{statement*}
        $V^{\bot}$ - всегда векторное подпространство.
    \end{statement*}
    
    \begin{statement*}
        $V$ - векторно пространство $\implies V^{\bot}$ имеет дополнительную размерность (в смысле дополнения к пространству):
        \[\dim V = k \implies \dim V^{\bot} = n - k\]
    \end{statement*}

    \begin{proof}
        У точки $p$ имеется окрестность $U \open M$, в которой все решения регулярны $\implies$ $U$ - $(n-k)$-мерное многообразие по теореме о регулярных решениях. Тогда $T_p(M) = T_p(U)$ - $(n-k)$-мерное векторное подпространство в $\mathbb{R}^n$
        \par

        \begin{lemma*}
            Пусть $p \in X$ - произвольное множество $\subset \mathbb{R}^n, \ g|_X = const$ - постоянная на множестве гладкая функция. Тогда $dg(p):T_p(X) \to \mathbb{R}$ - нулевое (зануляющее) отображение.

            \begin{proof}
                Пусть $v \neq 0$ из $T_p(X)$. $\exists x_n \to p, \ t_n \searrow 0, \ \frac{x_n - p}{t_n} \to v$
                \[dg(p)\langle v \rangle = \lim_{n \to \infty} \frac{g(x_n) - g(p)}{t_n} = \frac{0}{t_n} = 0\]
            \end{proof}
        \end{lemma*}

        \[f_i|_u \equiv 0 \implies df_i(p)\langle v \rangle = 0 \ \forall v \in T_p(U) = T_p(M)\]
        Помним: $df(p)\langle v \rangle = (\nabla f, v) = 0$. Тогда ясно, что $\nabla f_i(p) \bot v \ \forall v \in T_p(U) = T_p(M)$.
        \par
        Осталось доказать, что если $v \bot$ всем градиентам $f_i$, то $v \in T_p(M)$: \\
        $\nabla f_1, \hdots, \nabla f_k(p)$ - линейно независимы как строки матрицы Якоби ранга $k$. $v$ - решене линеаризованной системы $(\nabla f, v) = 0$ (помним, что градиент $f$ это матрица из градиентов $f_i$)
        Эта система имеет ранг $k$ - значит множество ее решений это $(n-k)$-мерное векторное пространство $W$. Но мы знаем, что множество
        $T_p(M)$ - тоже $(n-k)$-мерное векторное подпространство и уже доказали, что $T_p(M) \subset W$. 
        \[\dim T_p(M) = \dim W \text{ и } T_p(M) \subset W \implies T_p(M) = W\]
    \end{proof}

\end{theorem}
    \subsection{Условные экстремумы}

\begin{lemma*}[о не-максимуме]
    Пусть $p \in X \subset \mathbb{R}^n, \ g:X \to \mathbb{R}$ - гладкая функция.\\
    Если $\exists v \in T_p(X)$ такое, что $df(p)\langle v \rangle > 0$, то $p$ - не локальный максимум.

    \begin{proof}
        Пусть $v$ - вектор из условия. Тогда $\exists x_n \in X,\ t_n \searrow 0,\ \frac{x_n - p}{t_n} \to v$\\
        Знаем, что $dg(p)\langle v \rangle = \lim_{n \to \infty }\frac{g(x_n) - g(p)}{t_n} > 0 \implies$ числитель больше $0$ при $n \to \infty$, значит $g(x_n) > g(p) \implies$ $p$ - не локальный максимум.
    \end{proof}
\end{lemma*}

\begin{statement*}[Следствие]
    Если $X$ - многообразие и $p \in X \backslash \partial X$, то $p$ - экстремум $g \implies$ $dg(P)|_{T_p(X)} \equiv 0$

    \begin{proof}
        Если $dg(p)|_{T_p(X)} \neq 0$, то $\exists v \in T_p(X) \ dg(p)\langle v \rangle \neq 0$, 
        при этом $-v \in T_p(X)$ и $dg(p)\langle v \rangle$ будет другого знака. Тогда по лемме о не-экстремуме $p$ - не экстремум.
    \end{proof}
\end{statement*}


\begin{theorem}[о градиентах]
    Пусть $p$ - регулярное решение гладкой системы $\begin{array}{c} f_1(\overline{x}) = 0 \\ \vdots \\ f_k(\overline{x}) = 0 \end{array}$. Пусть $g: M \to \mathbb{R}$ - гладкая функция.  \par
    Если $p$ - экстремум $g$ на $M$, то $\exists \lambda_1, \hdots, \lambda_k \in \mathbb{R}$, такие, что
    \[\nabla g(p) = \lambda_1\nabla f_1(p) + \hdots + \lambda_k \nabla f_k(p)\]
    \begin{proof}
        Мы уже знаем, что если $p$ - экстремум, то $\nabla g(p) \bot T_p(M)$ или, что то же самое: $\nabla g(p) \in T_p(M)^{\bot}$. \par
        Пространство $T_p(M)^{\bot}$ - линейная оболочка градиентов функций $f_1, \hdots, f_k$ в точке $p$  ($T_p(M)^{\bot}$ имеет размерность $k \implies \nabla f_1(p), \hdots, \nabla f_k(p)$ - его базис). 
    \end{proof}
\end{theorem}

\subsection{Достаточное условие экстремума}
\begin{theorem}
    Пусть $f_1, \hdots, f_k$ - $C^r$-дифференцируемы и точка $p$ - регулярное решение, $p \in M = \{\overline{x} \mid f_1(\overline{x}) = 0, \hdots, f_k(\overline{x}) = 0\}$.
    Пусть $g$ - $C^2$-гладкая функция, определенная в окрестности точки $p$.

    \par
    Необходимое условие:\\
    Если $p$ - экстремум, то $\exists \lambda_1, \hdots, \lambda_k$, а в $p$ $L(p)$ равна $0$:
    \[L(x) = g(x) - (\lambda_1 f_1(x) + \hdots + \lambda_k f_k(x))\]


    \par
    Достаточное условие:\\
    Если в такой точке матрица Гессе $\Gamma = \frac{\partial^2 L}{\partial x_i \partial x_j}(p)$ положительно определена на 
    $T_p(M)$, то $p$ - строгий локальный минимум, если $\frac{\partial^2 L}{\partial x_i \partial x_j}(p) < 0$ - локальный максимум, если же
    $\exists u, v \in T_p(M)$, такие, что $\Gamma\langle v \rangle > 0, \Gamma\langle u \rangle < 0$, то $p$ - не экстремум.

    \begin{proof}
        Если $p$ - экстремум, то необходимое условие выполнено по теореме о градиентах:
        \[\exists \lambda_i: \ \nabla g(p) = \sum \lambda_i\nabla f_i(p)\]
        Это на самом деле то же самое, что и $\frac{\partial L}{\partial x_1}(\overline{\lambda}, p) = 0, \hdots, \frac{\partial L}{\partial x_k}(\overline{\lambda}, p) = 0$, где $\frac{\partial L}{\partial x_i}(\overline{\lambda}, p) = \frac{\partial g}{\partial x_i} - (\lambda_1 \cdot \frac{\partial f_1}{\partial x_i} + \hdots + \lambda_k \cdot \frac{\partial f_k}{\partial x_i}) = 0$.

        При этом $\frac{\partial L}{\partial \lambda_i}(x) = -f_i(x) = 0$, так как $x \in M$.
        \par
        Докажем достаточное условие:\\
        Пусть $dL(p) = 0, \Gamma = \frac{\partial^2 L}{\partial x_i \partial x_j}(p) > 0$ (т.е. положительно определенная матрица). Покажем, что у точки $p$ существует окрестность в $M$ такая, что в не $g(x) > g(p)$.
        \\
        От противного: предположим, что это не так, тогда $\exists x_n \in M, \ x_n \to p, g(x_n) \leq g(p)$. 
        Можно выбрать такую подпоследовательность $x_{n_k}$ такую, что 
        \[\frac{x_{n_k} - p}{\left| x_{n_k} - p\right|} \to v \text{- единичный вектор}, v \in T_p(M)\]
        Считаем, что уже $x_n$ такая, что $\frac{x_n - p}{\left| x_n - p \right|} \to v \in T_p(M), \ g(x_n) = L(x_n)$ - на множестве $M$ функция $g = L$, ведь $x \in M \implies f_i(x) = 0$.\\
        Значит $L \in C^2$ (в окрестности $p$), мы знаем, что 
        \[L(x) = L(p) + dL(P)\langle x - p \rangle + \frac{\partial ^2L}{\partial x_i \partial x_j} + o(\left| x - p \right|^2)\]
        В частности
        \[L(x_n) - L(p) = dL(p)\langle x_n - p \rangle + \frac{\partial^2 L}{\partial x_i \partial x_j}(p)\langle x_n - p \rangle + o(\left| x_n - p\right|^2), \ dL(p)\langle x_n - p \rangle = 0 \text{ по предположению}\]
        \[g(x_n) - g(p) = L(x_n) - L(p) = \frac{\partial^2 L}{\partial x_i \partial x_j}(p)\langle x_n - p \rangle + o(\left| x_n - p\right|^2)\]
        По "противному" предположению это $\leq 0$, но с другой стороны, поскольку $\Gamma = \frac{\partial^2 L}{\partial x_i \partial x_j}(p) > 0$ для $p \in T_p(M)$, то $\exists \varepsilon > 0$ такой, что $\Gamma\langle v \rangle \geq \varepsilon \cdot \left| v\right|^2 \ \forall v \in T_p(M)$. Это самое сложное место доказательства, вот почему это так:
        \par
        $\Gamma: T_p(M) \to \mathbb{R}$ - 2-однородная функция, непрерывная на $T_p(M)$ (поскольку это многочлен).
        Значит на множестве $W$, составленном из множества единичных векторов из $T_p(M)$ эта функция имеет минимум, больший 0, поскольку множество $W$ - компакт. \\
        Итак, $\forall v \in T_p(M)$ если $\left| v \right| = 1 \implies \Gamma \langle v \rangle \geq \varepsilon$, тогда $\forall v \in T_p(M)$ при любой длине $v$ $\Gamma \langle v \rangle \geq \left| v \right|^2$ (вместо единичного $v$ взяли $v$ не единичной длины, вторая степень вылезла в силу 2-го порядка однородности).

        \par
        Итак, $0 \geq g(x_n) - g(p) = L(x_n) - L(P) = \Gamma\langle x_n - p \rangle + o(\left| x_n - p\right|^2)$. Поделим это равенство на $\left| x_n - p \right|^2$:
        \[0 \geq \frac{g(x_n) - g(p)}{\left| x_n - p\right|^2} = \frac{\Gamma(p) \langle x_n - p \rangle}{\left| x_n - p \right|^2} + o(1)\]
        Поскольку $o(1) \underset{x_n \to p}{\to} 0$, то:
        \[\frac{\Gamma\langle x_n - p \rangle}{\left| x_n - p\right|^2} = \Gamma\langle \frac{x_n - p}{\left| x_n - p\right|} \rangle \to \Gamma\langle v \rangle \geq \varepsilon\]
        Значит $0 \geq \frac{g(x_n)-g(p)}{\left| x_n - p \right|^2} \underset{n \to \infty}{>} \frac{\varepsilon}{2}$ - пополам для уверенности. Получаем противоречие, таким образом доказали достаточность.
        \par
        Пусть теперь $\exists u, v \in T_p(M), \ \Gamma\langle u \rangle > 0, \Gamma\langle v \rangle < 0$\\
        Заметим, что $u \in T_p(M) \implies \exists x_n \in M$ такая, что $\frac{x_n - p}{\left| x_n - p\right|} \to \frac{u}{\left| u \right|}$.
        \[g(x_n) - g(p) = \Gamma \langle x_n - p \rangle + o(\left| x_n - p\right|^2) \text{ - поделим уравнение на $\left| x_n - p\right|^2$}\]
        \[\frac{g(x_n) - g(p)}{\left| x_n - p \right|^2} = \Gamma \langle \frac{x_n - p}{\left| x_n - p\right|} \rangle + o(1)\]
        Первое слагаемое стремится к $\Gamma \langle \frac{u}{\left| u \right|} \rangle > 0$, а второе - к нулю. Значит при $n \to \infty \ \frac{g(x_n) - g(p)}{\left| x_n - p \right|^2} > 0$. \\
        Значит, $p$ - не максимум функции $g$ на множестве $M$. Аналогично расписав для вектора $v$ получаем, что $p$ - не минимум.


    \end{proof}
\end{theorem}
    \section{Поточечная и равномерная сходимость}


Пусть $X$ - некое множество, $Y$ - $\mathbb{R}$ или $\mathbb{R}^n$ или вообще любое метрическое пространство.
Даны последовательность функций $f_n: X \to Y, n = 1, 2, \hdots$, и функция $f:X \to Y$.

\begin{definition}
    Последовательность функций сходится на $X$ к функции $f$, поточечно, если $\forall x \in X$ последовательность значений $f_n(x)$ сходится к $f$ при $n \to \infty$.
    \[\forall x \in X \quad f_n(x) \underset{n \to \infty}{\to} f(x)\]
\end{definition}

\begin{statement}
    Из непрерывной функций $f_n$ не всегда следует непрерывность функции $f$
    \begin{proof}
        Пусть $X = \left[0, 1\right]$, $f_n(x) = x^n$. Тогда $f_1(x) = x, f_2(x) = x^2, f_3(x) = x^3$ и так далее.
        В это случае    
        \[\underset{x \in \left[0, 1\right]}{\lim_{n \to \infty}} = \begin{cases}
            0 \ x < 1 \\ 
            1 \ x = 1
        \end{cases}\]
        Видно, что $f(x)$ разрывна, хотя все $f_n$ непрерывны.
    \end{proof}
\end{statement}

\begin{definition}
    Пусть $f_n: X \to \mathbb{R}$, $f:X \to \mathbb{R}$.\\
    Последовательность $f_n$ сходится к $f$ \textbf{равномерно} на множестве $X$, если 
    \begin{equation}
        \label{uniform_convergence}
        \underset{x \in X}{\sup} \left|f_n(x) - f(x)\right| \underset{n \to \infty}{\to} 0
    \end{equation}

    \par
    "Оригинальное"  определение равномерной сходимости:
    
    \begin{equation}
        \label{original_uniform_convergence}
        \forall \varepsilon > 0 \ \exists n_0 \mid \forall x \in X \ \forall n \geq n_0 \quad \left| f_n(x) - f(x) \right| \leq \varepsilon
    \end{equation}

    \begin{proof}
        Докажем эквивалентность двух определений:\par
        \eqref{uniform_convergence} $\implies$ \eqref{original_uniform_convergence}:\\
        Пусть $\varepsilon > 0$, надо подобрать $n_0$. По условию \eqref{uniform_convergence}:
        \[\exists n_0 \mid \forall n \geq n_0 \quad \underset{x \in X}{\sup} \left| f_n(x) - f(x) \right| \leq \varepsilon\]
        Но $\forall x \in X \ \sup \geq \left| f_n(x) - f(x) \right|$, значит 
        \[\forall x \in X \ \left|f_n(x) - f(x) \right| \leq \sup \left| f_n(x) - f(x) \right| \leq \varepsilon\]
    \end{proof}
\end{definition}

\begin{definition}
    Последовательность $f_n(x)$ поточечно сходится к $f(x)$ при любых $x$, если 
    \[\forall \varepsilon > 0 \ \forall x \in X \ \exists n_0 \in \mathbb{N} \ \forall n \geq n_0 \ \left| f_n(x) - f(x)\right| \leq \varepsilon\]

    \par
    \eqref{original_uniform_convergence} $\implies$ \eqref{uniform_convergence}: Надо доказать, что $\forall \varepsilon \ \exists n_0 \mid \forall n \geq n_0 \ \underset{x \in X}{\sup} \left| f_n(x) - f(x) \right| \leq \varepsilon$.
    
    По условию \eqref{original_uniform_convergence}:
    \[\exists n_0 \mid \forall n \geq n_0 \ \forall x \in X \ \left| f_n(x) - f(x) \right| \leq \varepsilon \implies sup \left| f_n(x) - f(x) \right| \leq \varepsilon\]
\end{definition}

Равномерную сходимость $f_n(x)$ к $f(x)$ будем обозначать $f_n \underset{n \to \infty}{\rightrightarrows} f$


\begin{theorem}[Критерий отсутствия равномерной сходимости]
    $f_n$ не сходится равномерно к $f$ $\iff$ $\exists x_n \in X$ такая, что $f_n(x_n) - f(x_n) \nrightarrow 0$
\end{theorem}

\begin{theorem}[о пределе пределов]
    Пусть $X \subseteq \mathbb{R}$ (или $\forall$ метрического пространства). \\ Пусть $\begin{cases} f: X \to \mathbb{R} \\ f_n: X \to \mathbb{R} \end{cases}$, $p$ - предельная точка $X$.

    Предположим, что $f_n(x) \underset{x \to p}{\to} y_n$ и $y_n \underset{n \to \infty}{\to} y$. 
    Если $f_n$ \textbf{равномерно} сходится к $f$ ($f_n \rightrightarrows f$) на $X$, то $f(x) \to y$ при $x \to p$.
    \[\lim_{n \to \infty} (\lim_{x \to p} f_n(x)) = \lim_{x \to p}(\lim_{n \to \infty} f_n(x))\] 

    \begin{proof}
        Пусть $\varepsilon > 0$, надо доказать, что у точки $p$ имеется окрестность $U$, такая, что $\forall x \in X \cap U \ \left| f(x) - y \right| \leq \varepsilon$.
        \[\left| f(x) - y \right| = \left| f(x) - f_n(x) + f_n(x) - y_n + y_n - y \right| \leq \left| f(x) - f_n(x) \right| + \left| f_n(x) - y_n \right| + \left| y_n - y \right|\] 
        Рассмотрим каждое слагаемое в отдельности:\\
        1:
         \[\exists n_0 \mid \forall n \geq n_0 \quad \left| f(x) - f_n(x) \right| \leq \frac{\varepsilon}{3}\]
        
        3:
            \item \[\exists n_1 \mid \forall n \geq n_1 \quad \left| y_n - y \right| \leq \frac{\varepsilon}{3}\]
            Пусть $n_2 = \max{n_0, n_1}$. Тогда $\forall n \geq n_2$ выполнено и 1, и 3.\\
        2:
            \[f_{n_2}(x) \underset{x \to p}{\to} y_{n_2} \implies \exists U \text{ - окрестность точки $p$, такая, что } \forall x \in U \cap X \ \left| f_{n_2}(x) - y_{n_2} \right| \leq \frac{\varepsilon}{3}\]
        
        Для таких $x$ из $U \cap X$ выполнено:
        \[\left| f(x) - y\right| \leq \left| f(x) - f_{n_2}(x) \right| + \left| f_{n_2}(x) - y_{n_2} \right| + \left| y_{n_2} - y \right| \leq \frac{\varepsilon}{3} + \frac{\varepsilon}{3} + \frac{\varepsilon}{3} = \varepsilon\]
    \end{proof}
\end{theorem}

\begin{theorem*}[Следствие]
    Если все $f_n$ непрерывны в $p \in X$ и $f_n \rightrightarrows f$ на $X$, то $f$ тоже непрерывна в точке $p$.\\
    Нет, тот контрпример показывал, что просто непрерывности $f_n$ недостаточно, а следствие утверждает, что непрерывности + равномерной сходимости уже достаточно.
    \begin{proof}
        Надо показать, что $f(x) \underset{x \to p}{\to}f(p)$. Заметим, что $f_n(x) \to f(x)$ при $n \to \infty$ в силу обычной поточечной сходимости. Теперь по теореме:
        \[f(x) = \lim_{n \to \infty} f_n(x) \implies \lim_{x \to p} f(x) = \lim_{x \to p} (\lim_{n \to \infty} f_n(x)) \overset{\text{по теореме}}{=} \lim_{n \to \infty} (\lim_{x \to p} f_n(x)) = \lim_{n \to \infty} f_n(p) = f(p)\]
    \end{proof}

\end{theorem*}

\begin{theorem*}[Еще одно следствие]
    Если все $f_n$ непрерывны \textbf{на $X$} и $f_n \rightrightarrows$, то $f$ тоже непрерывна на $X$.
    \begin{proof}
        Нечего тут доказывать: определение непрерывности в точке расширяем до определения непрерывности на множестве.
    \end{proof}
\end{theorem*}

\begin{theorem*}[Критерий Коши равномерной сходимости последовательности функций]
    Пусть $X$ - множество, $f_n: X \to \mathbb{R}$. 
    \[\exists \text{функция } f: X \to \mathbb{R} \text{ такая, что } f_n \rightrightarrows f \text{ на $X$} \implies \forall \varepsilon \ \exists n_0 \mid \begin{array}{c} \forall x \in X \\ \forall k, l \geq n_0\end{array} \ \left| f_k(x) - f_l(x) \right| \leq \varepsilon\]
\end{theorem*}
    \section{Ряды Фурье}

Пусть $E$ - евклидово пространство (векторное пространство со скалярным произведением).\\
Скалярное произведение $\scalar{a}{b}$ - бинарная операция, удовлетворяющая аксиомам:
\begin{enumerate}
    \item $\scalar{a}{b} = \scalar{b}{a} \iff a, b \in \Real; \ \scalar{a}{b} = \overline{\scalar{b}{a}} \iff a, b \in \Compl$
    \item $\scalar{a}{a} \ge 0, \ \scalar{a}{a} = 0 \iff a = 0$
    \item $\scalar{\lambda u + \mu v}{w} = \lambda \scalar{u}{w} + \mu \scalar{v}{w}$
\end{enumerate}
Модуль вектора вычисляется с помощью $\left| v \right| = \sqrt{\scalar{v}{v}}$.

\begin{statement*}
    Пусть $\vec{e_1}, \hdots, \vec{e_n}$ - попарно ортогональные векторы ($\scalar{\vec{e_i}}{\vec{e_j}} = 0$). Пусть $\vec{v} = x_1\vec{e_1} + \hdots + x_n \vec{e_n}$, тогда
    \[x_i = \frac{\scalar{v}{e_i}}{\scalar{e_i}{e_i}} \ \forall i = 1, \hdots, n\]
    \begin{proof}
        \[k \neq i \implies \scalar{e_k}{e_i} = 0 \implies \scalar{v}{e_i} = \scalar{\sum_{k=1}^{n} x_k e_k}{e_i} = \sum_{k=1}^{n}x_k \scalar{e_k}{e_i} = x_i \scalar{e_i}{e_i}\]
    \end{proof}
\end{statement*}

Если $\{\vec{e_i}\}$ - попарно ортогональые ненулевые вектора, то они образуют линейно независимый набор.
\[\left| v \right|^2 = \scalar{v}{v} = \scalar{\sum x_i e_i}{\sum x_j e_j} = \sum x_i x_j \scalar{e_j}{e_i} =  \sum x_i^2 \left|e_i \right|^2\]
В частности, если все длины векторов $\vec{e_i}$ равны $1$, то $\left| v \right|^2 = \sum x_i^2$.

\par
Пусть $f, g$ - хорошие функции на отрезке $\left[0, 2\pi\right]$, то их скалярное произведение определено следующим образом:
\[\scalar{f}{g} := \int_{0}^{2\pi} f(t)g(t) \diff t\]

\begin{statement*}
    Функции $\sin(nx)$ и $\cos(mx)$ ортогональны при $\forall n, m \in \Nat$. А также:
    \begin{itemize}
        \item $\sin(nx) \bot \sin(mx) \ \forall n \neq m \ge 0$
        \item $\cos(nx) \bot \cos(mx) \ \forall n \neq m \ge 0$
        \item $|| \sin(nx) || = \pi \ n > 0$
        \item $|| \cos(nx) || = \pi \ n > 0$
    \end{itemize}
    \begin{proof}
        \[\cos(mx) = \frac{e^{imx}+e^{-imx}}{2}; \ \sin(nx) = \frac{e^{inx}-e^{-inx}}{2i}\]
        \[\int_{0}^{2\pi} e^{inx}\cdot e^{imx} \diff x,\  n, m \in \Nat = \int_{0}^{2\pi} e^{i(n+m)x} \diff x = \begin{cases*}
            \int_{0}^{2\pi} 1 \diff x = 2\pi \ n+m = 0\\
            \frac{e^{i(n+m)x}}{i(n+m)} \big|_{0}^{2\pi} = 0 \ n+m \neq 0
        \end{cases*}\]

        \[\int_{0}^{2\pi} \cos(nx)\sin(mx)\diff x = \int_{0}^{2\pi} \frac{e^{i(n+m)x} - e^{i(n-m)x} + e^{i(m-n)x} - e^{-i(n+m)x}}{2\cdot 2i} \diff x = 0 \ \forall n,m \in \Nat\]
        Аналогично остальные 
        \[\int_{0}^{2\pi} \sin(nx)^2 \diff x = \int_{0}^{2\pi} \frac{e^{2inx} - e^{-2inx}-2}{-4} \diff x = \int_{0}^{2\pi} \frac{-2}{-4} \diff x = \pi\]
    \end{proof}
\end{statement*}

\begin{statement*}[Следствие]
    Пусть $f(t)$ - линейная комбинация вида
    \[f(t) = \lambda + b_1\sin(t) + \hdots + b_n\sin(nt) + a_1\sin(t) + \hdots + a_n\cos(nt)\]
    Тогда 
    \[b_k = \frac{\int_{0}^{2\pi}f(t)\sin(kt)\diff t}{\pi}, \quad a_k = \frac{\int_{0}^{2\pi}f(t)\cos(kt)\diff t}{\pi}, \quad \lambda = \frac{\int_{0}^{2\pi} f(t)\cdot 1 \diff t}{2\pi}\]

\end{statement*}

\begin{definition}
    Ряд Фурье функции $f(t)$ (разложение обозначается знаком $\leftrightharpoons$)
    \[f(t) \leftrightharpoons \frac{a_0}{2} + \sum_{k=1}^{\infty} \left[a_k\sin(kt) + b_k\cos(kt)\right]\]
    \[a_k = \frac{1}{\pi} \int_{0}^{2\pi} f(t)\cos(kt)\diff t \quad b_k = \frac{1}{\pi}\int_{0}^{2\pi}f(t)\sin(kt) \diff t, \ k \ge 0\]
\end{definition}
    \begin{theorem*}[Теорема о тригонометрической аппроксимации]
    Пусть $f$ - непрерывная функция на $[0, 2\pi]$, $f(0) = f(2\pi)$. Тогда $\forall \varepsilon$ $\exists$ тригонометрический многочлен $P(x)$, такой, что 
    \\ $\forall x \in [0, 2\pi] \ \abs{f(x) - P(x)} \le \varepsilon$.
    \begin{proof}
        \begin{enumerate}
            \item Заменим $f$ кусочно гладкой линейной функцией $\tilde{f}$, так, что \\ $\forall x \ \abs{f(x) - \tilde{f}(x)} \le \frac{\varepsilon}{2}$.\par
            По теореме Кантора-Гейне (функция, непрерывная на компакте, равномерно непрерывна на нем) непрерывная функция $f$ на отрезке $[0, 2\pi]$ равномерно непрерывна, то есть
            \[\forall \varepsilon > 0 \ \exists \delta > 0 \ \forall x_1, x_2 \in [0, 2\pi] \quad \abs{x_1 - x_2} < \delta \implies \abs{f(x_1) - f(x_2)} \le \varepsilon\]
            Пользуясь этим разобьем $[0, 2\pi]$ на конечное число кусочков $[a_k, a_{k+1}]$, настолько коротких, что $\forall x_1, x_2  \in [a_k, a_{k+1}] \ \abs{f(x_1) - f(x_2)} \le \frac{\varepsilon}{4}$.
            \par
            Пусть $\tilde{f}(a_i) = f(a_i) \ \forall i$. На промежутках $[a_i, a_{i+1}]$ $\tilde{f}$ - прямая между точками $(a_i, f(a_i))$ и $(a_{i+1}, f(a_{i+1}))$.
            \begin{align*}
                \forall x \in [a_i, a_{i+1}] \\
                \abs{\tilde{f}(x) - f(x)} = \abs{\tilde{f}(x) - \tilde{f}(a_i) + \tilde{f}(a_i) - f(x)} \le \\
                \le \abs{\tilde{f}(x) - \tilde{f}(a_i)} + \abs{\tilde{f}(a_i) - f(x)}
            \end{align*}
            \begin{align*}
                \abs{\tilde{f}(x) - \tilde{f}(a_i)} \le \abs{\tilde{f}(a_{i+1}) - f(a_i)} = \abs{f(a_{i+1}) - f(a_i)} \le \frac{\varepsilon}{4}
            \end{align*}
            \[\abs{\tilde{f}(a_i) - f(x)} = \abs{f(a_i) - f(a_i)} \le \frac{\varepsilon}{4}\]

            \item Функция $\tilde{f}$ принадлежит классу Фурье, $\tilde{f}(0) = f(2\pi)$. По теореме Фурье $S_n(\tilde{f}) \rightrightarrows \tilde{f}$ на $[0, 2\pi]$. Тогда $\exists n_0 \ \underset{x \in [0, 2\pi]}{\sup}\abs{S_{n_0}(\tilde{f}(x)) - \tilde{f}(x)} \le \frac{\varepsilon}{2}$. А значит $\abs{S_{n_0}(\tilde{f}(x)) - f(x)} \le \frac{\varepsilon}{2} + \frac{\varepsilon}{2}$ (доказывается хитрым движением $\pm \tilde{f}(x)$ под модуль).
        \end{enumerate}
    \end{proof}
\end{theorem*}

\begin{theorem*}[Теорема Стоуна-Вейерштрасса]
    Непрерывную на отрезке функцию можно равномерно приблизить многочленом:
    \[f:[a,b] \to \Compl(\Real) \implies \forall \varepsilon \ \exists \text{многочлен } P(x) \ \forall x \in [a,b] \ \abs{U(x) - f(x)} \le \varepsilon\]

    \begin{proof}
        \begin{enumerate}
            \item Пусть $\lambda(x):[a,b] \underset{\text{линейное}}{\to} [0, 2\pi]$. 
            \[g(t) = f(\lambda^{-1}(t)):[0, 2\pi] \to \Compl\]
            Существует тригонометрический многочлен $Q(t)$ (например, форма Фурье) такой, что $\forall t \in [0, 2\pi] \ \abs{Q(t) - g(t)} \le \varepsilon$. Теперь осталось приблизить $Q(t)$ обычным многочленом.
            \par
            \[Q(t) = q_0 + q_1\cos(t) + \hdots + q_N\cos(Nt) + q_1\sin(t) + \hdots + q_N\sin(Nt)\]
            Для $\forall k_0$ форма $\cos(k_0 t)$ равномерно приближаемая на $[0, 2\pi]$ частичными суммами своего ряда Тейлора. 
            \[\cos(k_0 t) = 1 - \frac{(k_0 t)^2}{2!} + \frac{(k_0 t)^4}{4!} + \hdots\]
            Радиус сходимости равен бесконечности, поэтому ряд сходится равномерно на любом отрезке конечной длины.
            \par
            Пусть $n_0$ настолько большое, что все частичные суммы для \\ $\cos(t), \hdots, \cos(Nt), \sin(t), \hdots, \sin(Nt)$ были $\delta$-близки к своим функциям. Т.е.
            \[\delta := \frac{\varepsilon}{
                2\Bigl(
                    \sum_{i=1}^k \abs{q_i} + \sum_{i=1}^k \abs{\tilde{q}_i}
                \Bigr)
            }\]
            Соответствующая сумма имеет вид 
            \[q_0 + \sum_{k=1}^{n_0} q_k \cos(kt) + \sum_{k=1}^{n_0} \tilde{q}_k \sin(kt),\]
            где первая сумма — это частичная сумма ряда Тейлора (до $n_0$) для $\cos(kt)$, 
            а вторая — для $\sin(kt)$. Поскольку каждая из функций $\cos(kt)$, $\sin(kt)$
            аппроксимируется многочленом от $t$, вся сумма является многочленом;
            следовательно, можно взять $P(t)$ в указанном виде.
        \end{enumerate}
    \end{proof}
\end{theorem*}
\end{document}
